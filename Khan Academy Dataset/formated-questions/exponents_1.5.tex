% Auto-converted from khan-exercises
\documentclass{article}
\usepackage{amsmath,amssymb}
\usepackage[T1]{fontenc}
\usepackage{textcomp}
\newcommand{\abs}[1]{\lvert #1\rvert}

\begin{document}
\section*{Positive exponents with positive and negative bases}
\textbf{Question.} \textbackslash\{\}Large\{5\textasciicircum{}\{3\} = \{?\}\}

\textbf{Answer.} [

\textbf{Hints.}
\begin{itemize}
  \item The base in this expression is 5.
  \item The exponent in this expression is 3.
  \item This expression means 5 multiplied by itself 3 times.
  \item [[negParens(BASE)]]\textasciicircum{}\{3\} = [
  \item [ is 5 multiplied by itself 3 times.
  \item The base is 5.
  \item The exponent is 3.
  \item [ is 5 to the power of 3.
  \item [ = [[negParens(BASE)]]\textasciicircum{}\{3\}
  \item = [[v]]
\end{itemize}
\end{document}
