% Auto-converted from khan-exercises
\documentclass{article}
\usepackage{amsmath,amssymb}
\usepackage[T1]{fontenc}
\usepackage{textcomp}
\newcommand{\abs}[1]{\lvert #1\rvert}

\begin{document}
\section*{Powers of complex numbers}
\textbf{Question.} \textbackslash\{\}blue\{x\} is plotted in blue below.
                Change the angle and radius to plot \textbackslash\{\}orange\{y\} in orange.

\textbf{Answer.} Radius: 1
                        
                            
                            
                        
                    
                    
                        Angle: 0
                        
                            
                            
                        
                    
                    1
                
                [
                    graph.currComplexPolar.getAngleNumerator(),
                    graph.currComplexPolar.getRadius()
                    ]
                
                    var angle = guess[0];
                    var radius = guess[1];

                    if (angle === 0 \&\& radius === 1) \{
                        return "";
                    \}

                    return angle === ANSWER\_ANGLE\_NUMERATOR \&\& radius === ANSWER\_RADIUS;
                
                
                    redrawComplexPolarForm(guess[0], guess[1]);
                
                
                    redrawComplexPolarForm(guess[0], guess[1]);

\textbf{Hints.}
\begin{itemize}
  \item First express \textbackslash\{\}blue\{x\} in Euler form.
                        \textbackslash\{\}large \{\textbackslash\{\}blue\{[[polarForm( BASE\_RADIUS, BASE\_ANGLE, EULER\_FORM )]]\} = \textbackslash\{\}blue\{[[polarForm( BASE\_RADIUS, BASE\_ANGLE, true )]]\}\}
                    

                    
                        
                            \textbackslash\{\}large\{(\textbackslash\{\}blue\{[[polarForm( BASE\_RADIUS, BASE\_ANGLE, true )]]\}) \textasciicircum{} \{2\} =
                            
                            \textbackslash\{\}orange\{[[coefficient(ANSWER\_RADIUS)]]e \textasciicircum{} \{2 $\cdot$ [[eulerFormExponent( BASE\_ANGLE )]]\}\}\}
                        
                        
                    

                    
                        The angle of the result is \textbackslash\{\}large\{2 $\cdot$ [[piFraction( BASE\_ANGLE, true )]]\},
                        which is \textbackslash\{\}large\{[[piFraction( ANGLE\_MULTIPLE, true )]]\}.
  \item \textbackslash\{\}large\{\textbackslash\{\}orange\{y = [[polarForm( ANSWER\_RADIUS, ANSWER\_ANGLE, true )]]\}\}
                    
                        Converting this back from Euler form, we get
                        \textbackslash\{\}large\{y = [[polarForm( ANSWER\_RADIUS, ANSWER\_ANGLE, EULER\_FORM )]]\}.
\end{itemize}
\end{document}
