% Auto-converted from khan-exercises
\documentclass{article}
\usepackage{amsmath,amssymb}
\usepackage[T1]{fontenc}
\usepackage{textcomp}
\newcommand{\abs}[1]{\lvert #1\rvert}

\begin{document}
\section*{Reading bar charts 2}
\textbf{Question.} Which student's score improved the most between the midterm and final exams?

\textbf{Answer.} [[MOST\_IMPROVED]]

\textbf{Hints.}
\begin{itemize}
  \item [[person( INDEX + 1 )]]'s final exam bar is taller than his
                            midterm bar, so [[person( INDEX + 1 )]] improved his score. His midterm score was
                            [[MIDTERM[ INDEX ]]] and his final exam score was [[FINAL[ INDEX ]]], so
                            he improved by [[IMPROVEMENT[ INDEX ]]] points.[[person( INDEX + 1 )]]'s final exam bar is taller than her
                            midterm bar, so [[person( INDEX + 1 )]] improved her score. Her midterm score was
                            [[MIDTERM[ INDEX ]]] and her final exam score was [[FINAL[ INDEX ]]], so
                            she improved by [[IMPROVEMENT[ INDEX ]]] points.
                            
                        
                        
                            [[person( INDEX + 1 )]]'s final exam bar is shorter than his
                            midterm bar, so [[person( INDEX + 1 )]] did not improve his score.[[person( INDEX + 1 )]]'s final exam bar is shorter than her
                            midterm bar, so [[person( INDEX + 1 )]] did not improve her score.
                            
                        
                        
                            [[person( INDEX + 1 )]]'s final exam bar is the same height as his
                            midterm bar, so [[person( INDEX + 1 )]] did not improve his score.[[person( INDEX + 1 )]]'s final exam bar is the same height as her
                            midterm bar, so [[person( INDEX + 1 )]] did not improve her score.
  \item [[MOST\_IMPROVED]] improved his score the most, scoring [[BEST\_IMPROVEMENT]] more points
                        on his final exam than on his midterm.
  \item [[MOST\_IMPROVED]] improved her score the most, scoring [[BEST\_IMPROVEMENT]] more points
                        on her final exam than on her midterm.
  \item Find the two bars for [[STUDENT]].
  \item Compare the height of [[STUDENT]]'s blue bar to the scale on the left to find his midterm score.
                            
                                The bar's height is halfway between [[MIDTERM[ INDEX ] - 5]] and [[MIDTERM[ INDEX ] + 5]], so
                            
                            [[STUDENT]] earned [[MIDTERM[ INDEX ]]] points on the midterm.
                        
                            Compare the height of [[STUDENT]]'s blue bar to the scale on the left to find her midterm score.
                            
                                The bar's height is halfway between [[MIDTERM[ INDEX ] - 5]] and [[MIDTERM[ INDEX ] + 5]], so
                            
                            [[STUDENT]] earned [[MIDTERM[ INDEX ]]] points on the midterm.
  \item Compare the height of [[STUDENT]]'s orange bar to the scale on the left to find his final exam score.
                            
                                The bar's height is halfway between [[FINAL[ INDEX ] - 5]] and [[FINAL[ INDEX ] + 5]], so
                            
                            [[STUDENT]] earned [[FINAL[ INDEX ]]] points on the final exam.
                        
                            Compare the height of [[STUDENT]]'s orange bar to the scale on the left to find her final exam score.
                            
                                The bar's height is halfway between [[FINAL[ INDEX ] - 5]] and [[FINAL[ INDEX ] + 5]], so
                            
                            [[STUDENT]] earned [[FINAL[ INDEX ]]] points on the final exam.
  \item Subtract the midterm score from the final exam score to find out how much [[STUDENT]] improved.
  \item \textbackslash\{\}color\{ORANGE\}\{[[FINAL[ INDEX ]]]\} - \textbackslash\{\}color\{\#6495ED\}\{[[MIDTERM[ INDEX ]]]\} = [[IMPROVEMENT[ INDEX ]]], so
                        [[STUDENT]] improved by [[IMPROVEMENT[ INDEX ]]] points from the midterm to the final exam.
  \item Find the two bars for [[STUDENT]].
  \item Use the key to figure out which of the two bars shows the score for the [[TEST]].
  \item Compare the height of [[STUDENT]]'s blueorange bar to the scale on the left.
  \item The bar's height is halfway between [[ANSWER - 5]] and [[ANSWER + 5]], so
                        [[STUDENT]] earned [[ANSWER]] points on the [[TEST]].
  \item [[STUDENT]] earned [[ANSWER]] points on the [[TEST]].
  \item [[person( INDEX + 1 )]]'s final exam bar is taller than his
                            midterm bar, so [[person( INDEX + 1 )]] improved his score.[[person( INDEX + 1 )]]'s final exam bar is taller than her
                            midterm bar, so [[person( INDEX + 1 )]] improved her score.
                            
                        
                        
                            [[person( INDEX + 1 )]]'s final exam bar is shorter than his
                            midterm bar, so [[person( INDEX + 1 )]] did not improve his score.[[person( INDEX + 1 )]]'s final exam bar is shorter than her
                            midterm bar, so [[person( INDEX + 1 )]] did not improve her score.
                        
                        
                            [[person( INDEX + 1 )]]'s final exam bar is the same height as his
                            midterm bar, so [[person( INDEX + 1 )]] did not improve his score.[[person( INDEX + 1 )]]'s final exam bar is the same height as her
                            midterm bar, so [[person( INDEX + 1 )]] did not improve her score.
  \item Count the number of students who improved their scores.
  \item [[NUM\_IMPROVED]] students improved their scores from the midterm to the final exam.
\end{itemize}
\end{document}
