% Auto-converted from khan-exercises
\documentclass{article}
\usepackage{amsmath,amssymb}
\usepackage[T1]{fontenc}
\usepackage{textcomp}
\newcommand{\abs}[1]{\lvert #1\rvert}

\begin{document}
\section*{Volume 1}
\textbf{Question.} What is the volume of this rectangular prism? Drag on the rectangular prism to rotate it.

\textbf{Answer.} [[LENGTH*WIDTH*HEIGHT]] cubic [[randFromArray(metricUnits)]]

\textbf{Hints.}
\begin{itemize}
  \item The volume of a shape is measured by the number of one \textbackslash\{\}text\{ [[randFromArray(metricUnits)]]\} cubes which make up the shape.
  \item To make counting the cubes easier, break the rectangular prism into [[max(max(LENGTH, WIDTH), HEIGHT)]] slabs which are each
                    [[DIMENSIONS[2]]] 6, 8, 7 wide,  [[DIMENSIONS[1]]] 6, 8, 7 long, and
                    1 6, 8, 7 high.
  \item Each slab has [[DIMENSIONS[2]]] rows containing [[DIMENSIONS[1]]] cubes, so there are
                       [[DIMENSIONS[2]]] $\times$ [[DIMENSIONS[1]]] = [[WIDTH*LENGTH]] cubes in each slab.
  \item Since there are [[WIDTH*LENGTH]] cubes in each slab, there are a total of
                [[max(max(LENGTH, WIDTH), HEIGHT)]] $\times$ [[WIDTH*LENGTH]] = [[HEIGHT*WIDTH*LENGTH]] cubes in the whole rectangular prism.
  \item Since each cube is a cubic \textbackslash\{\}text\{ [[randFromArray(metricUnits)]]\}, the volume of the rectangular prism is [[HEIGHT*WIDTH*LENGTH]] 6, 8, 7\textasciicircum{}3
  \item The volume of the 
                            [[DIMENSIONS[1]]] 6, 8, 7 $\times$
                            [[DIMENSIONS[2]]] 6, 8, 7 $\times$
                            1 6, 8, 7
                         slab pictured above is 
                            [[LENGTH*WIDTH]] 6, 8, 7\textasciicircum{}3
                         since there are
                        [[DIMENSIONS[1]]] rows of
                        [[DIMENSIONS[2]]] cubic
                        [[plural\_form(UNIT)]].
  \item Since the slabs have a height of 1 \textbackslash\{\}text\{ [[randFromArray(metricUnits)]]\}, x is the number of these slabs which fill up the whole rectangular prism.
  \item We can figure out the number of slabs needed by seeing how many times [[LENGTH*WIDTH]]
                     goes into [[LENGTH*WIDTH*HEIGHT]]
  \item So x = [[LENGTH*WIDTH*HEIGHT]] \textbackslash\{\}div [[LENGTH*WIDTH]]
  \item Thus x = [[max(max(LENGTH, WIDTH), HEIGHT)]]
\end{itemize}
\end{document}
