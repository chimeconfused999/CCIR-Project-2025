% Auto-converted from khan-exercises
\documentclass{article}
\usepackage{amsmath,amssymb}
\usepackage[T1]{fontenc}
\usepackage{textcomp}
\newcommand{\abs}[1]{\lvert #1\rvert}

\begin{document}
\section*{Imaginary unit powers}
\textbf{Question.} \textbackslash\{\}large\{i \textasciicircum{} \{9\} = ?\}

\textbf{Answer.} i

\textbf{Hints.}
\begin{itemize}
  \item The most important property of the imaginary unit i is
                    that \textbackslash\{\}blue\{i \textasciicircum{} 2\} = \textbackslash\{\}pink\{-1\}.
  \item \textbackslash\{\}qquad i \textasciicircum{} 9 = i
  \item The most important property of the imaginary unit i is:
                    \textbackslash\{\}qquad \textbackslash\{\}blue\{i \textasciicircum{} 2\} = \textbackslash\{\}pink\{-1\}
  \item Therefore:
                    \textbackslash\{\}qquad i \textasciicircum{} 4 = (\textbackslash\{\}blue\{i \textasciicircum{} 2\}) \textasciicircum{} 2 = (\textbackslash\{\}pink\{-1\}) \textasciicircum{} 2 = 1
  \item So, we can simplify the expression by rewriting it in terms of i\textasciicircum{}4.
  \item Because 9 \textbackslash\{\}div \textbackslash\{\}blue\{4\} = \textbackslash\{\}green\{2\}
                        \textbackslash\{\}text\{ remainder \} \textbackslash\{\}red\{1\},
                    
                    
                        \textbackslash\{\}qquad $\begin{aligned}
                        i ^ {9} &=& (i^\blue{4})^\green{2}
                         $\textbackslash\{\}cdot$ i^\red{1} \\
                        &=& (1)^\green{2}
                        $\textbackslash\{\}cdot$ i^\red{1} \\
                        &=& i^\red{1}
                        1
                        \end{aligned}$
  \item Anything to the first power is the number itself.
                    \textbackslash\{\}qquad i \textasciicircum{} \textbackslash\{\}red\{1\} = i
  \item i \textasciicircum{} \{9\} = i \textasciicircum{} \{1\} = i
\end{itemize}
\end{document}
