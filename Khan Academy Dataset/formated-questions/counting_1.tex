% Auto-converted from khan-exercises
\documentclass{article}
\usepackage{amsmath,amssymb}
\usepackage[T1]{fontenc}
\usepackage{textcomp}
\newcommand{\abs}[1]{\lvert #1\rvert}

\begin{document}
\section*{Counting 1}
\textbf{Question.} If Alex read all of the articles he was assigned, how many articles did he read?

\textbf{Answer.} 23 articles

\textbf{Hints.}
\begin{itemize}
  \item Instead of counting articles 41 through 63,
                        we can subtract 40 from each number.
  \item Now the articles are 1 through 23.
  \item So Alex read 23 articles.
  \item Notice that he read 23 and not 22 articles.
  \item Notice that she read 23 and not 22 articles.
  \item One cut will make two slices, two cuts will make three slices, and so on.
  \item Therefore, we need 12 cuts to make 13 slices.
  \item If the fence is one meter long, he needs two posts (one for each end).
  \item If the fence is one meter long, she needs two posts (one for each end).
  \item If the fence is two meters long, then he needs three posts, and so on.
  \item If the fence is two meters long, then she needs three posts, and so on.
  \item Therefore, he needs 14 posts for a 13 meter fence.
  \item Therefore, she needs 14 posts for a 13 meter fence.
\end{itemize}
\end{document}
