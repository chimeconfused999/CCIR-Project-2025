% Auto-converted from khan-exercises
\documentclass{article}
\usepackage{amsmath,amssymb}
\usepackage[T1]{fontenc}
\usepackage{textcomp}
\newcommand{\abs}[1]{\lvert #1\rvert}

\begin{document}
\section*{Volume word problems}
\textbf{Question.} What could be the dimensions of my box?

\textbf{Answer.} Length =  cm 
                   Width \textbackslash\{\}hphantom\{ \}=  cm 
                   Height  =  cm
                
                
                [$("input#response1").val(),$("input\#response2").val(),$("input#response3").val()]
                
                    var lengthMessage = null, lengthEmpty = false;

                    var LengthValidator = Khan.answerTypes.predicate.createValidatorFunctional(function(length, error) {
                        if (length > 10) {
                            return false;
                        }

                        var heightMessage = null, heightEmpty = false;

                        var WidthValidator = Khan.answerTypes.predicate.createValidatorFunctional(function(width, error) {
                            if (width > 10) {
                                return false;
                            }

                            var HeightValidator = Khan.answerTypes.predicate.createValidatorFunctional(function(height, error) {
                                if (height > 10) {
                                    return false;
                                }
                                return Math.abs(length * width * height - VOL) < Math.pow(2, -42);
                            }, {forms: 'integer, proper, improper, mixed, decimal'});

                            var heightResult = HeightValidator(guess[2]);

                            if (heightResult.empty) {
                                heightEmpty = true;
                            }
                            if (heightResult.message !== null) {
                                heightMessage = heightResult.message;
                            }
                            if (heightResult.correct) {
                                heightEmpty = false;
                                heightMessage = null;
                            }

                            return heightResult.correct;
                        }, {forms: 'integer, proper, improper, mixed, decimal'});

                        var widthResult = WidthValidator(guess[1]);

                        if (widthResult.empty |$| heightEmpty) \{
                            lengthEmpty = true;
                        \}
                        if (widthResult.message !== null |$| heightMessage !== null) {
                            lengthMessage = widthResult.message |$| heightMessage;
                        \}
                        if (widthResult.correct) \{
                            lengthEmpty = false;
                            lengthMessage = null;
                        \}

                        return widthResult.correct;
                    \}, \{forms: 'integer, proper, improper, mixed, decimal'\});

                    var lengthResult = LengthValidator(guess[0]);

                    // TODO(emily): In the future, when validator-functions can return empty and
                    // message separately, make this actually work
                    if (lengthResult.empty |$| lengthEmpty) {
                        return "";
                    } else if (lengthResult.message !== null) {
                        return lengthResult.message;
                    } else if (lengthMessage !== null) {
                        return lengthMessage;
                    }

                    return lengthResult.correct;$

\textbf{Hints.}
\begin{itemize}
  \item We can use a factor tree to break [[VOL]] into its prime
                    factorization. Which of the prime numbers divides into [[VOL]]?
  \item [[REMAINING[I]]] is divisible by \textbackslash\{\}blue\{[[FACTOR]]\},
                        leaving us with [[REMAINING[I] / FACTOR]].
  \item \textbackslash\{\}blue\{[[FACTORIZATION[FACTORIZATION.length - 1]]]\} is prime, so we're done factoring.
  \item So the prime factors of [[VOL]] are:
                    
                        \textbackslash\{\}qquad[[FACTORIZATION.join("\textbackslash\{\}\textbackslash\{\}space\textbackslash\{\}\textbackslash\{\}color\{black\}\{\textbackslash\{\}$\times$\}\textbackslash\{\}\textbackslash\{\}space")]]
  \item So the dimensions of the package should be
                    [[FACTORIZATION[0]]],
                    [[FACTORIZATION[1]]] and
                    [[FACTORIZATION[2]]].
  \item We can find the dimensions of the package by arranging the factors into three groups of products, all less than 10.
                    
                        One solution would be:
                        ([[getPrimeFactorization(WIDTH).join("\textbackslash\{\}\textbackslash\{\}space\textbackslash\{\}\textbackslash\{\}color\{black\}\{\textbackslash\{\}$\times$\}\textbackslash\{\}\textbackslash\{\}space")]]),
                        ([[getPrimeFactorization(LENGTH).join("\textbackslash\{\}\textbackslash\{\}space\textbackslash\{\}\textbackslash\{\}color\{black\}\{\textbackslash\{\}$\times$\}\textbackslash\{\}\textbackslash\{\}space")]]) and
                        ([[getPrimeFactorization(HEIGHT).join("\textbackslash\{\}\textbackslash\{\}space\textbackslash\{\}\textbackslash\{\}color\{black\}\{\textbackslash\{\}$\times$\}\textbackslash\{\}\textbackslash\{\}space")]])
                    
                    
                        So the dimensions of the box could be [[WIDTH]],
                        [[LENGTH]] and [[HEIGHT]].
  \item The tank has a volume of [[H1]] $\times$ [[L1]] $\times$ [[W1]] = 
                        [[H1 * L1 * W1]] cubic [[plural\_form(METERS, H1 * L1 * W1)]],
                        and the metal box has a volume of [[H2]] $\times$ [[L2]] $\times$ [[W2]] = 
                        [[H2 * W2 * L2]] cubic [[plural\_form(METERS, H2 * L2 * W2)]].
  \item Since there is no water in the box, the volume of the water in the tank is the volume of the tank minus the volume of the metal box.
  \item The volume of the water in the tank is 
                    
                        [[H1 * L1 * W1]] \textbackslash\{\}text\{[[UNIT]]\}\textasciicircum{}3 -
                        [[H2 * L2 * W2]] \textbackslash\{\}text\{[[UNIT]]\}\textasciicircum{}3 = 
                        [[H1 * L1 * W1 - H2 * L2 * W2]]\textbackslash\{\}text\{[[UNIT]]\}\textasciicircum{}3
  \item The first box has a volume of [[H1]] $\times$ [[L1]] $\times$ [[W1]] = 
                        [[VOL1]] cubic [[plural\_form(METERS, VOL1)]].
                        The second box has a volume of [[H2]] $\times$ [[L2]] $\times$ [[W2]] = 
                        [[VOL2]] cubic [[plural\_form(METERS, VOL2)]]
  \item Since my gorilla can play in both boxes, we need to add the volumes of the two boxes.
  \item The total amount of space my gorilla has to play is 
                    
                        [[VOL1]] \textbackslash\{\}text\{[[UNIT]]\}\textasciicircum{}3 +
                        [[VOL2]] \textbackslash\{\}text\{[[UNIT]]\}\textasciicircum{}3 = 
                        [[VOL1 + VOL2]]\textbackslash\{\}text\{[[UNIT]]\}\textasciicircum{}3
  \item First we will figure out how many cubic [[plural\_form(TO)]] fit in one cubic [[FROM]].
  \item There are [[CONVERSION]] [[plural\_form(TO, CONVERSION)]] in every [[FROM]].
  \item So a cubic [[FROM]] is the same as a
                        
                            [[CONVERSION]] \textbackslash\{\}text\{ [[TO\_TEXT]]\}
                            $\times$ [[CONVERSION]] \textbackslash\{\}text\{ [[TO\_TEXT]]\}
                            $\times$[[CONVERSION]] \textbackslash\{\}text\{ [[TO\_TEXT]]\}
                        
                        cube.
  \item There are [[CONVERSION]] $\times$ [[CONVERSION]]
                    $\times$ [[CONVERSION]] = [[CONVERSION\_CUBED]] \textbackslash\{\}text\{ [[TO\_TEXT]]\}\textasciicircum{}3
                   in each cubic [[FROM]], by the volume formula.
  \item Since we have [[VOL]] cubic [[plural\_form(FROM, VOL)]],
                   and each cubic [[FROM]] holds [[CONVERSION\_CUBED]] cubic
                   [[plural\_form(TO, CONVERSION\_CUBED)]], we have a total of
                   [[VOL]] $\times$ [[CONVERSION\_CUBED]]
                   cubic [[plural\_form(TO, VOL * CONVERSION\_CUBED)]].
  \item So [[VOL * CONVERSION\_CUBED]] cubic [[plural\_form(TO, VOL * CONVERSION\_CUBED)]]
                    fit in [[VOL]] cubic [[plural\_form(FROM, VOL)]].
\end{itemize}
\end{document}
