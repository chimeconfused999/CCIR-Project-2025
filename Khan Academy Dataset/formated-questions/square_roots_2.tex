% Auto-converted from khan-exercises
\documentclass{article}
\usepackage{amsmath,amssymb}
\usepackage[T1]{fontenc}
\usepackage{textcomp}
\newcommand{\abs}[1]{\lvert #1\rvert}

\begin{document}
\section*{Estimating square roots}
\textbf{Question.} The value of $\sqrt{14}$ lies between which two consecutive integers?

\textbf{Answer.} 3 < $\sqrt{14}$ < 4

\textbf{Hints.}
\begin{itemize}
  \item Consider the perfect squares near 14.
                            [What are perfect squares?]
                        
                        
                            
                                Perfect squares are integers which can be obtained by squaring an integer.
                            
                            
                                The first 13 perfect squares are:
                            
                            \textbackslash\{\}qquad 1,4,9,16,25,36,49,64,81,100,121,144,169
  \item 9 is the nearest perfect square less than 14.
  \item 16 is the nearest perfect square more than 14.
  \item So, 9 < 14 < 16.
  \item $\sqrt{9}$ < $\sqrt{14}$ < $\sqrt{16}$
                        3 < $\sqrt{14}$ < 4
\end{itemize}
\end{document}
