% Auto-converted from khan-exercises
\documentclass{article}
\usepackage{amsmath,amssymb}
\usepackage[T1]{fontenc}
\usepackage{textcomp}
\newcommand{\abs}[1]{\lvert #1\rvert}

\begin{document}
\section*{Fractions on the number line 1}
\textbf{Question.} Move the \textbackslash\{\}orange\{\textbackslash\{\}text\{orange dot\}\} to
                \textbackslash\{\}orange\{$\frac{4}{3}$\} on the number line.

\textbf{Answer.} graph.movablePoint.coord[0] 
                
                    if ( guess === 0 ) \{
                        return "";
                    \}
                    return abs( SOLUTION - guess ) < 0.001;
                
                
                    graph.movablePoint.setCoord( [ guess, 0 ] );

\textbf{Hints.}
\begin{itemize}
  \item Above we've drawn the number line from 0 to 2, divided into equal pieces.
  \item Between 0 and 1, the number line is divided into 6 equal pieces,
                    so each piece represents $\frac{1}{6}$.
  \item We move the orange dot \textbackslash\{\}blue\{8\} tick marks because $\frac{4}{3}$
                        is the same as 8 copies of $\frac{1}{6}$.
  \item The orange number shows where \textbackslash\{\}orange\{$\frac{4}{3}$\} is on the number line.
\end{itemize}
\end{document}
