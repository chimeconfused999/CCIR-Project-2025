% Auto-converted from khan-exercises
\documentclass{article}
\usepackage{amsmath,amssymb}
\usepackage[T1]{fontenc}
\usepackage{textcomp}
\newcommand{\abs}[1]{\lvert #1\rvert}

\begin{document}
\section*{Discount, tax, and tip word problems}
\textbf{Question.} Alex buys a basket of [[plural\_form(fruit(1))]] on sale for \textbackslash\{\}$15 before tax.  The sales tax is 15\%.  What is the total price Alex pays for the basket of [[plural_form(fruit(1))]]?$

\textbf{Answer.} \textbackslash\{\}$\ 2$

\textbf{Hints.}
\begin{itemize}
  \item To find the total price, first find the amount of sales tax paid by multiplying the sales tax by the original price of the basket of [[plural\_form(fruit(1))]].
                    \textbackslash\{\}blue\{15\textbackslash\{\}\%\} $\times$ \textbackslash\{\}green\{$15} = \text{?}$
  \item Percent means "out of one hundred," so \textbackslash\{\}blue\{15\textbackslash\{\}\%\} is equivalent to \textbackslash\{\}blue\{$\frac{15}{100}$\} which is also equal to \textbackslash\{\}blue\{15 \textbackslash\{\}div 100\}.
                    \textbackslash\{\}blue\{15 \textbackslash\{\}div 100 = 0.15\}
  \item To find the amount of sales tax that must be paid, multiply \textbackslash\{\}blue\{0.15\} by the original price.
                    \textbackslash\{\}blue\{0.15\} $\times$ \textbackslash\{\}green\{$15} = \purple{$2.25\}
  \item To find the final price Alex paid, add the sales tax you just found to the original price.
                    \textbackslash\{\}purple\{$2.25} + \green{$15.00\} = $2.00$
  \item To find the amount of sales tax you would pay, multiply the sales tax by the original price of the item.
                    \textbackslash\{\}blue\{1.0\textbackslash\{\}\%\} $\times$ \textbackslash\{\}green\{\textbackslash\{\}$12} = \text{?}$
  \item Percent means "out of one hundred," so \textbackslash\{\}blue\{1.0\textbackslash\{\}\%\} is equivalent to \textbackslash\{\}blue\{1.0 \textbackslash\{\}div 100\}.
                    \textbackslash\{\}blue\{1.0 \textbackslash\{\}div 100 = 0.010\}
  \item To find the amount of sales tax that must be paid, multiply \textbackslash\{\}blue\{0.010\} by the original price. Round to the nearest cent.
                    \textbackslash\{\}blue\{0.010\} $\times$ \textbackslash\{\}green\{\textbackslash\{\}$12} = \$2.00
  \item First, find the amount of the discount by multiplying the original price of the of the item by the discount.
                    \textbackslash\{\}blue\{\textbackslash\{\}$12} $\textbackslash\{\}times$ \green{55\%} = \text{?}$
  \item Percent means "out of one hundred," so \textbackslash\{\}green\{55\textbackslash\{\}\%\} is equivalent to \textbackslash\{\}green\{$\frac{55}{100}$\} which is also equal to \textbackslash\{\}green\{55 \textbackslash\{\}div 100\}.
                    \textbackslash\{\}green\{55 \textbackslash\{\}div 100 = 0.55\}
  \item To find the amount of money saved, multiply \textbackslash\{\}green\{0.55\} by the original price.
                    \textbackslash\{\}green\{0.55\} $\times$ \textbackslash\{\}blue\{\textbackslash\{\}$12} = \purple{\$6.60\}
  \item To find the final price Alex paid, subtract \textbackslash\{\}purple\{\textbackslash\{\}$6.60} from the original price.
                    \blue{\$12\} - \textbackslash\{\}purple\{\textbackslash\{\}$6.60} = $2.00
  \item To find the amount saved with the discount, multiply the discount by the original price.
                    \textbackslash\{\}blue\{55\textbackslash\{\}\%\} $\times$ \textbackslash\{\}green\{\textbackslash\{\}$12} = \text{?}$
  \item Percent means "out of one hundred," so \textbackslash\{\}blue\{55\textbackslash\{\}\%\} is equivalent to \textbackslash\{\}blue\{$\frac{55}{100}$\} which is also equal to \textbackslash\{\}blue\{55 \textbackslash\{\}div 100\}.
                    \textbackslash\{\}blue\{55 \textbackslash\{\}div 100 = 0.55\}
  \item To find the amount of money you saved, multiply \textbackslash\{\}blue\{0.55\} by the original price.
                    \textbackslash\{\}blue\{0.55\} $\times$ \textbackslash\{\}green\{\textbackslash\{\}$12} = \$2.00
  \item The tip amount is equal to 20\textbackslash\{\}\% $\times$ \textbackslash\{\}green\{\textbackslash\{\}$23.00}.
                    
                        We can find the tip by first calculating a \purple{10\%} tip
                        
                         and multiplying it by two.
                         and a \pink{5\%} tip, and then adding those two numbers together.$
  \item To calculate a \textbackslash\{\}purple\{10\textbackslash\{\}\%\} tip, move the decimal point in \textbackslash\{\}green\{\textbackslash\{\}$23.00} one place to the left.
                    \purple{10\%} $\textbackslash\{\}times$ \green{\$23.00\} = \textbackslash\{\}purple\{\textbackslash\{\}$2.30}$
  \item To calculate a \textbackslash\{\}pink\{5\textbackslash\{\}\%\} tip, divide the \textbackslash\{\}purple\{10\textbackslash\{\}\%\} tip amount in half.
                        \textbackslash\{\}pink\{5\textbackslash\{\}\%\} $\times$ \textbackslash\{\}green\{\textbackslash\{\}$23.00} = \purple{\$2.30\} \textbackslash\{\}div 2 = \textbackslash\{\}pink\{\textbackslash\{\}$1.15}.
                    
                    
                        To calculate a \blue{20\%} tip, multiply the 10\% tip amount by two.
                        \blue{20\%} $\textbackslash\{\}times$ \green{\$23.00\} = \textbackslash\{\}purple\{\textbackslash\{\}$2.30} $\textbackslash\{\}times$ 2 = \blue{\$4.60\}.
  \item The total bill is the cost of the meal plus the tip.
                    
                        \textbackslash\{\}green\{\textbackslash\{\}$23.00} + 
                        \blue{\$4.60\}
                        \textbackslash\{\}purple\{\textbackslash\{\}$4.60}
                        = \$2.00.
\end{itemize}
\end{document}
