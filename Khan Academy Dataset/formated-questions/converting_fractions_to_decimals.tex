% Auto-converted from khan-exercises
\documentclass{article}
\usepackage{amsmath,amssymb}
\usepackage[T1]{fontenc}
\usepackage{textcomp}
\newcommand{\abs}[1]{\lvert #1\rvert}

\begin{document}
\section*{Converting fractions to decimals}
\textbf{Question.} $\frac{61}{3}$

\textbf{Answer.} 4

\textbf{Hints.}
\begin{itemize}
  \item $\frac{61}{3}$ represents 61 \textbackslash\{\}div 3
                .
  \item $\begin{aligned}
                    61 \div 3
                    &=& (-1 $\textbackslash\{\}times$ -61) \div 3 \\
                    &=& -1 $\textbackslash\{\}times$ (-61 \div 3)
                    \end{aligned}$
  \item $\begin{aligned}
                    61 \div 3
                    &=& 61 \div (-1 $\textbackslash\{\}times$ -3) \\
                    &=& -1 $\textbackslash\{\}times$ (61 \div -3)
                    \end{aligned}$
  \item $\frac{61}{3}$ = 4
  \item $\frac{61}{3}$ represents 61 \textbackslash\{\}div 3
                .
  \item Notice how the decimal is repeating and will continue to repeat as we bring down more zeros.
  \item So the answer is 4 to 4 decimal places.
\end{itemize}
\end{document}
