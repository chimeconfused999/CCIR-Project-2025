% Auto-converted from khan-exercises
\documentclass{article}
\usepackage{amsmath,amssymb}
\usepackage[T1]{fontenc}
\usepackage{textcomp}
\newcommand{\abs}[1]{\lvert #1\rvert}

\begin{document}
\section*{Arithmetic sequences 1}
\textbf{Question.} The first [[cardinalThrough20(N)]] terms of an arithmetic sequence are given:
                [[GIVEN.join(",")]], \textbackslash\{\}ldots
                What is the [[ordinalThrough20(N + 1)]] term in the sequence?

\textbf{Answer.} [[A + D * N]]

\textbf{Hints.}
\begin{itemize}
  \item In any arithmetic sequence, each term is equal to the previous term plus the common difference.
  \item Thus, the second term is equal to the first term plus the common difference. In this sequence, the second term, [[A + D]], is [[abs(D)]] more than the first term, [[A]].
  \item Thus, the second term is equal to the first term plus the common difference. In this sequence, the second term, [[A + D]], is [[abs(D)]] less than the first term, [[A]].
  \item Therefore, the common difference is [[D]].
  \item The [[ordinalThrough20(N + 1)]] term in the sequence is equal to the [[ordinalThrough20(N)]] term plus the common difference, or [[A + D * (N - 1)]] + [[D]] = [[A + D * N]].
\end{itemize}
\end{document}
