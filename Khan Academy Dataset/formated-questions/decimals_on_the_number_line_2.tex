% Auto-converted from khan-exercises
\documentclass{article}
\usepackage{amsmath,amssymb}
\usepackage[T1]{fontenc}
\usepackage{textcomp}
\newcommand{\abs}[1]{\lvert #1\rvert}

\begin{document}
\section*{Decimals on the number line 2}
\textbf{Question.} Move the orange dot to \textbackslash\{\}pink\{3.97\} on the number line.

\textbf{Answer.} graph.movablePoint.coord[0] 
                
                    if (guess === 0) \{
                        return "";
                    \}
                    return abs(SOLUTION - guess) < 0.001;
                
                
                    graph.movablePoint.setCoord([guess, 0]);

\textbf{Hints.}
\begin{itemize}
  \item The length between \textbackslash\{\}blue\{4.5\} and \textbackslash\{\}blue\{4.0\}
                    on the number line is \textbackslash\{\}blue\{0.1\},
                    or \textbackslash\{\}blue\{1 \textbackslash\{\}text\{ [[decimalPlaceNames[1]]]\}\}.
  \item This \textbackslash\{\}blue\{0.1\} is divided into 10 equal pieces.
                    Each piece has a length of \textbackslash\{\}green\{0.01\},
                    or \textbackslash\{\}green\{1 \textbackslash\{\}text\{ [[decimalPlaceNames[2]]]\}\}.
  \item From \textbackslash\{\}blue\{4.5\}, we need to move 
                        \textbackslash\{\}green\{10 \textbackslash\{\}text\{ [[plural\_form(decimalPlaceNames[2], HUNDREDTHS)]]\}\}
                        to the right to get to \textbackslash\{\}pink\{3.97\}.
  \item The highlighted number shows where \textbackslash\{\}pink\{3.97\} is on the number line.
  \item The length between the blue labeled tick marks is \textbackslash\{\}blue\{0.1\},
                    or \textbackslash\{\}blue\{1 \textbackslash\{\}text\{ [[decimalPlaceNames[1]]]\}\}.
  \item Each \textbackslash\{\}blue\{0.1\} is divided into 10 equal pieces
                    that have a length of \textbackslash\{\}green\{0.01\},
                    or \textbackslash\{\}green\{1 \textbackslash\{\}text\{ [[decimalPlaceNames[2]]]\}\}.
  \item From \textbackslash\{\}blue\{3.9\}, we need to move
                        \textbackslash\{\}green\{0 \textbackslash\{\}text\{ [[plural\_form(decimalPlaceNames[2], HUNDREDTHS \% 10)]]\}\}
                        to the right to get to \textbackslash\{\}pink\{3.97\}.
  \item The highlighted number shows where \textbackslash\{\}pink\{3.97\} is on the number line.
  \item The length between the blue labeled tick marks is \textbackslash\{\}blue\{0.1\},
                    or \textbackslash\{\}blue\{1 \textbackslash\{\}text\{ [[decimalPlaceNames[1]]]\}\}.
  \item Each \textbackslash\{\}blue\{0.1\} is divided into 10 equal pieces
                    that have a length of \textbackslash\{\}green\{0.01\},
                    or \textbackslash\{\}green\{1 \textbackslash\{\}text\{ [[decimalPlaceNames[2]]]\}\}.
  \item We can rewrite the labels as \textbackslash\{\}green\{\textbackslash\{\}text\{[[plural\_form(decimalPlaceNames[2], 2)]]\}\}.
  \item The highlighted number shows where \textbackslash\{\}pink\{3.97\} is on the number line.
\end{itemize}
\end{document}
