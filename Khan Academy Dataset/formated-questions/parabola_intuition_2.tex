% Auto-converted from khan-exercises
\documentclass{article}
\usepackage{amsmath,amssymb}
\usepackage[T1]{fontenc}
\usepackage{textcomp}
\newcommand{\abs}[1]{\lvert #1\rvert}

\begin{document}
\section*{Parabola intuition 2}
\textbf{Question.} Drag the focus and directrix to define a
                    parabola with the equation:

\textbf{Answer.} Drag the focus and directrix on the graph to match the equation.
                        
                            Equation of the parabola:
                            y\{\}+0
                            \{\}=\{\}$\frac{1}{4}$
                            (x\{\}+0)\textasciicircum{}2
                        
                    

                    [
                        graph.currParabola.getVertexX(),
                        graph.currParabola.getVertexY(),
                        graph.currParabola.getLeadingCoefficient()
                    ]

                    
                        if (guess[0] === 0 \&\& guess[1] === 0 \&\& guess[2] === 0.25) \{
                            return "";
                        \}
                        return (abs(guess[0] - X1) < 0.001) \&\& (abs(guess[1] - Y1) < 0.001) \&\& (abs(guess[2] - A) < 0.001);

\textbf{Hints.}
\begin{itemize}
  \item The equation is in the form y - y\_1 = a(x - x\_1)\textasciicircum{}2.
  \item The leading coefficient, a, in the equation is [[fractionReduce.apply(KhanUtil, toFraction(A, 0.001))]], which is positive.
                        Therefore the parabola is upward opening.
  \item The leading coefficient, a, in the equation is [[fractionReduce.apply(KhanUtil, toFraction(A, 0.001))]], which is negative.
                        Therefore the parabola is downward opening.
  \item The value of x\_1 is [[fractionReduce.apply(KhanUtil, toFraction(X1, 0.001))]], what does this number mean for the parabola?
  \item The minimum value of [[fractionReduce.apply(KhanUtil, toFraction(A, 0.001))]](x - [[fractionReduce.apply(KhanUtil, toFraction(X1, 0.001))]])\textasciicircum{}2 occurs when
                        (x - [[fractionReduce.apply(KhanUtil, toFraction(X1, 0.001))]])\textasciicircum{}2 = 0.
                        Therefore the vertex of the parabola is at x = [[fractionReduce.apply(KhanUtil, toFraction(X1, 0.001))]].
  \item The maximum value of [[fractionReduce.apply(KhanUtil, toFraction(A, 0.001))]](x - [[fractionReduce.apply(KhanUtil, toFraction(X1, 0.001))]])\textasciicircum{}2 occurs when
                        (x - [[fractionReduce.apply(KhanUtil, toFraction(X1, 0.001))]])\textasciicircum{}2 = 0.
                        Therefore the vertex of the parabola is at x = [[fractionReduce.apply(KhanUtil, toFraction(X1, 0.001))]].
  \item The focus of the parabola has the same x-coordinate as the vertex,
                            so move the focus horizontally, so its x-coordinate is [[fractionReduce.apply(KhanUtil, toFraction(X1, 0.001))]].
                        
                        
                            Show me
  \item At x = [[fractionReduce.apply(KhanUtil, toFraction(X1, 0.001))]], then y - [[fractionReduce.apply(KhanUtil, toFraction(Y1, 0.001))]] = 0.
                        
                            Therefore, the vertex of the parabola is at y = [[fractionReduce.apply(KhanUtil, toFraction(Y1, 0.001))]].
                            The focus and directrix should be an equal distance above and below this value.
  \item The number in the denominator of a is twice the distance between the directrix to the focus.
                        Therefore, the focus and directrix are separated by 10 units.
  \item The focus is therefore 5 units above the vertex,
                        and the directrix is 5 units below the vertex.
  \item The focus is therefore -5 units below the vertex,
                        and the directrix is -5 units above the vertex.
  \item The focus is at ([[fractionReduce.apply(KhanUtil, toFraction(X1, 0.001))]], [[fractionReduce.apply(KhanUtil, toFraction(VERTEX\_Y, 0.001))]])
                            and the directrix is at y = [[fractionReduce.apply(KhanUtil, toFraction(DIR\_Y, 0.001))]].
                        
                        
                            Show me
\end{itemize}
\end{document}
