% Auto-converted from khan-exercises
\documentclass{article}
\usepackage{amsmath,amssymb}
\usepackage[T1]{fontenc}
\usepackage{textcomp}
\newcommand{\abs}[1]{\lvert #1\rvert}

\begin{document}
\section*{Permutations}
\textbf{Question.} How many unique ways are there to arrange the letters in the word FEAT?

\textbf{Answer.} [[factorial(NUM\_NAMES) / factorial(NUM\_NAMES - SLOTS)]]

\textbf{Hints.}
\begin{itemize}
  \item Let's try building the re-arrangements (or permutations) letter by letter.
                            The word is 4 letters long:
                        \_ \_ \_ \_ 
                        For the first blank, we have 4 choices of letters.
  \item After we put in the first letter, let's say it's \textbackslash\{\}text\{C\},
                            we have 3 blanks left.
                        
                        \textbackslash\{\}text\{C\} \_ \_ \_ 
                        
                            For the second blank, we only have 3 choices of letters left.
                            So far, there are 4 $\cdot$ 3 unique choices we can make.
  \item We can continue in this fashion to put in a third letter, and so on.
                        At each step, we have one fewer unique choice to make, until we get to the last letter, and there's only one we can put in.
  \item So, the total number of unique re-arrangements must be
                        4$\cdot$3$\cdot$2$\cdot$1.
                        Another way of writing this is 4!,
                        or 4 factorial, which is [[factorial(NUM\_NAMES) / factorial(NUM\_NAMES - SLOTS)]].
  \item We can build our line of reindeer one by one: there are 3 slots,
                        and we have 5 different reindeer we can put in the first slot.
  \item Once we fill the first slot, we only have 4 reindeer left,
                        so we only have 4 choices for the second slot.
                        So far, there are 5 $\cdot$ 4 = 20
                        unique choices we can make.
  \item We can continue in this way for the third reindeer, where we will have 3 choices.
  \item We can continue in this way for the third reindeer, and so on, until we reach the last slot, where we will
                        have 3 choices for the last reindeer.
  \item So, the total number of unique choices we could make to get to an arrangement of reindeer is
                        5$\cdot$4$\cdot$3.
                        Another way of writing this is
                        $\frac{5!}{(5-3)!}$ = [[factorial(NUM\_NAMES) / factorial(NUM\_NAMES - SLOTS)]]
\end{itemize}
\end{document}
