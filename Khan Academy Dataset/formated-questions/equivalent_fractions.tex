% Auto-converted from khan-exercises
\documentclass{article}
\usepackage{amsmath,amssymb}
\usepackage[T1]{fontenc}
\usepackage{textcomp}
\newcommand{\abs}[1]{\lvert #1\rvert}

\begin{document}
\section*{Equivalent fractions}
\textbf{Question.} What number could replace [[SYMBOL]] below?

\textbf{Answer.} [[C]]

\textbf{Hints.}
\begin{itemize}
  \item The fraction on the left represents [[A]] out of [[B]] slices of a rectangular [[pizza( 1 )]].
  \item What if we cut the [[pizza( 1 )]] into [[D]] slices instead? How many slices would result in the same amount of [[pizza( 1 )]]?
  \item We would need [[C]] slices.
  \item $\frac{[[A]]}{[[B]]}$ = $\frac{[[C]]}{[[D]]}$ and so the answer is [[C]].
  \item Another way to get the answer is to multiply by $\frac{[[M]]}{[[M]]}$.
                    $\frac{[[M]]}{[[M]]}$ = $\frac{1}{1}$ = 1 so really we are multiplying by 1.
  \item The final equation is: $\frac{[[A]]}{[[B]]}$ $\times$ $\frac{[[M]]}{[[M]]}$ = $\frac{[[C]]}{[[D]]}$  so our answer is [[C]].
  \item The fraction on the left represents [[A]] out of [[B]] slices of a rectangular [[pizza( 1 )]].
  \item How many total slices would we need if we want the same amount of [[pizza( 1 )]] in [[C]] slices?
  \item We would need to cut the [[pizza( 1 )]] into [[D]] slices.
  \item $\frac{[[A]]}{[[B]]}$ = $\frac{[[C]]}{[[D]]}$ and so the answer is [[D]].
  \item Another way to get the answer is to multiply by $\frac{[[M]]}{[[M]]}$.
                    $\frac{[[M]]}{[[M]]}$ = $\frac{1}{1}$ = 1 so really we are multiplying by 1.
  \item The final equation is: $\frac{[[A]]}{[[B]]}$ $\times$ $\frac{[[M]]}{[[M]]}$ = $\frac{[[C]]}{[[D]]}$  so our answer is [[D]].
\end{itemize}
\end{document}
