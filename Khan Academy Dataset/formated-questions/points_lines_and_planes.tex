% Auto-converted from khan-exercises
\documentclass{article}
\usepackage{amsmath,amssymb}
\usepackage[T1]{fontenc}
\usepackage{textcomp}
\newcommand{\abs}[1]{\lvert #1\rvert}

\begin{document}
\section*{Points, lines, and planes}
\textbf{Question.} 

\textbf{Answer.} [[A + B + E]]
                [[A + C + E]]
                [[A + E + B]]
                [[A + E + C]]
                [[B + A + E]]
                [[B + C + E]]
                [[B + E + A]]
                [[B + E + C]]
                [[C + A + E]]
                [[C + B + E]]
                [[C + E + A]]
                [[C + E + B]]
                [[E + A + B]]
                [[E + A + C]]
                [[E + B + A]]
                [[E + B + C]]
                [[E + C + A]]
                [[E + C + B]]

                
                    Plane

\textbf{Hints.}
\begin{itemize}
  \item Planes can be named with three noncollinear points.
                    
                        Noncollinear points are points that are not on the same line.
  \item Find any three points in the plane \textbackslash\{\}mathcal\{[[R]]\} that are
                    not on the same line and list them in any order.
  \item For example, we can write \textbackslash\{\}mathcal\{[[R]]\} as plane
                    [[A + B + E]], plane [[A + C + E]],
                    or plane [[B + E + C]].
  \item Lines are named using any two points on the line. The order doesn't matter.
  \item The points must have the \textbackslash\{\}leftrightarrow above because we're
                naming a line, not a ray or a segment.
  \item Another way to name line \textbackslash\{\}ell is [[SOLUTION]].
  \item Collinear means that they lie on the same line.
  \item Can you draw a straight line through points [[toSentence(POINTS)]]?
  \item Yes, points [[toSentence(POINTS)]] are collinear.
  \item No, points [[toSentence(POINTS)]] are not collinear.
  \item Through any two points, there is exactly one line.
  \item Points can be collinear even if the line isn't drawn in the figure.
  \item Can you draw a straight line through points [[toSentence(POINTS)]]?
                Actually, can you draw a straight line through any two points?
  \item Yes, points [[toSentence(POINTS)]] are collinear.
  \item Coplanar points are points that all lie on the same plane.
  \item Can a flat surface pass through all the points without bending?
  \item No, any flat surface that includes three of the points won't include the fourth.
                    For example, points [[toSentence(COPLANAR)]] are in plane
                    \textbackslash\{\}mathcal\{[[R]]\}, but point [[D]] is not.
  \item Yes, points [[toSentence(POINTS)]] all lie within a single
                    flat surface. In this case, plane \textbackslash\{\}mathcal\{[[R]]\}.
  \item Yes, there is always at least one flat surface that passes through
                    any three points.
\end{itemize}
\end{document}
