% Auto-converted from khan-exercises
\documentclass{article}
\usepackage{amsmath,amssymb}
\usepackage[T1]{fontenc}
\usepackage{textcomp}
\newcommand{\abs}[1]{\lvert #1\rvert}

\begin{document}
\section*{Limiting reagent stoichiometry}
\textbf{Question.} How many grams of \textbackslash\{\}text\{NaCl\} will be produced from
        1 \textbackslash\{\}text\{g\} of \textbackslash\{\}text\{Na\} and
        [[randRange(1, 40 * (R2\_RATIO * R2\_MOLAR\_MASS) / (R1\_RATIO * R1\_MOLAR\_MASS))]] \textbackslash\{\}text\{g\} of \textbackslash\{\}text\{Cl\}\_2?

\textbf{Answer.} [[roundTo(3, P1\_MOL * P1\_MOLAR\_MASS)]] grams (you can round to the nearest gram)

\textbf{Hints.}
\begin{itemize}
  \item \textbackslash\{\}dfrac\{1 \textbackslash\{\}cancel\{\textbackslash\{\}text\{g\}\}\}\{[[roundTo(3, molarMass("Na"))]] \textbackslash\{\}cancel\{\textbackslash\{\}text\{g\}\} / \textbackslash\{\}text\{mol\}\} =
                \textbackslash\{\}blue\{\textbackslash\{\}text\{ [[roundTo(3, R1\_MASS / R1\_MOLAR\_MASS)]] [[plural\_form(MOLE, R1\_MOL)]]\}\} \textbackslash\{\}text\{ of \}\textbackslash\{\}text\{Na\}
            
            [Explain]
            
                
                    First we want to convert the given amount of \textbackslash\{\}text\{Na\} from grams to moles. To do this, we divide
                    the given amount of \textbackslash\{\}text\{Na\} by the molar mass of \textbackslash\{\}text\{Na\}.
                
                     \textbackslash\{\}dfrac\{\textbackslash\{\}text\{grams of \}\textbackslash\{\}text\{Na\}\}\{\textbackslash\{\}text\{molar mass of \}\textbackslash\{\}text\{Na\}\} = \textbackslash\{\}text\{moles of \}\textbackslash\{\}text\{Na\}
                
                    To find the molar mass of \textbackslash\{\}text\{Na\}, we look up the atomic weight of each atom in a molecule of
                    \textbackslash\{\}text\{Na\} in the periodic table and add them together.
                    In this case, it's [[roundTo(3, molarMass("Na"))]] \textbackslash\{\}text\{g/mol\}.
                
                    Dividing the given 1 \textbackslash\{\}text\{g\} of \textbackslash\{\}text\{Na\} by the molar mass of
                    [[roundTo(3, molarMass("Na"))]] \textbackslash\{\}text\{g/mol\} tells us we're starting with
                    [[roundTo(3, R1\_MASS / R1\_MOLAR\_MASS)]]\textbackslash\{\}text\{ [[plural\_form(MOLE, R1\_MOL)]] of \}\textbackslash\{\}text\{Na\}.
  \item \textbackslash\{\}dfrac\{[[randRange(1, 40 * (R2\_RATIO * R2\_MOLAR\_MASS) / (R1\_RATIO * R1\_MOLAR\_MASS))]] \textbackslash\{\}cancel\{\textbackslash\{\}text\{g\}\}\}\{[[roundTo(3, molarMass("Cl") * 2)]] \textbackslash\{\}cancel\{\textbackslash\{\}text\{g\}\} / \textbackslash\{\}text\{mol\}\} =
                \textbackslash\{\}green\{\textbackslash\{\}text\{ [[plural(R2\_MOL, "mole")]]\}\} \textbackslash\{\}text\{ of \}\textbackslash\{\}text\{Cl\}\_2
            
            [Explain]
            
                
                    We want to convert the given amount of \textbackslash\{\}text\{Cl\}\_2 from grams to moles. To do this, we divide
                    the given amount of \textbackslash\{\}text\{Cl\}\_2 by the molar mass of \textbackslash\{\}text\{Cl\}\_2.
                
                     \textbackslash\{\}dfrac\{\textbackslash\{\}text\{grams of \}\textbackslash\{\}text\{Cl\}\_2\}\{\textbackslash\{\}text\{molar mass of \}\textbackslash\{\}text\{Cl\}\_2\} = \textbackslash\{\}text\{moles of \}\textbackslash\{\}text\{Cl\}\_2
                
                    To find the molar mass of \textbackslash\{\}text\{Cl\}\_2, we look up the atomic weight of each atom in a molecule of
                    \textbackslash\{\}text\{Cl\}\_2 in the periodic table and add them together.
                    In this case, it's [[roundTo(3, molarMass("Cl") * 2)]] \textbackslash\{\}text\{g/mol\}.
                
                    Dividing the given [[randRange(1, 40 * (R2\_RATIO * R2\_MOLAR\_MASS) / (R1\_RATIO * R1\_MOLAR\_MASS))]] \textbackslash\{\}text\{g\} of \textbackslash\{\}text\{Cl\}\_2 by the molar mass of
                    [[roundTo(3, molarMass("Cl") * 2)]] \textbackslash\{\}text\{g/mol\} tells us we're starting with
                    \textbackslash\{\}text\{ [[roundTo(3, R2\_MASS / R2\_MOLAR\_MASS)]] [[plural\_form(MOLE, R2\_MOL)]]\} \textbackslash\{\}text\{ of \}\textbackslash\{\}text\{Cl\}\_2.
  \item The mole ratio of \textbackslash\{\}dfrac\{\textbackslash\{\}text\{Na\}\}\{\textbackslash\{\}text\{Cl\}\_2\} in the reaction is
                    $\frac{2}{1}$.
                    [Explain]
                
                
                    
                        The reaction is \textbackslash\{\}blue\{2\}\textbackslash\{\}text\{Na\} +
                        \textbackslash\{\}red\{1\}\textbackslash\{\}text\{Cl\}\_2 \textbackslash\{\}rightarrow
                        2\textbackslash\{\}text\{NaCl\}.
                        The coefficients in front of each molecule tell us in what ratios the molecules react. In this case
                        [[cardinalThrough20(R1\_RATIO)]] \textbackslash\{\}text\{Na\} for every
                        [[cardinalThrough20(R2\_RATIO)]] \textbackslash\{\}text\{Cl\}\_2 molecule.
                    
                        The reaction is \textbackslash\{\}blue\{2\}\textbackslash\{\}text\{Na\} +
                        \textbackslash\{\}red\{1\}\textbackslash\{\}text\{Cl\}\_2 \textbackslash\{\}rightarrow
                        2\textbackslash\{\}text\{NaCl\}.
                        The coefficients in front of each molecule tell us in what ratios the molecules react. In this case
                        [[cardinalThrough20(R1\_RATIO)]] \textbackslash\{\}text\{Na\} for every
                        [[cardinalThrough20(R2\_RATIO)]] \textbackslash\{\}text\{Cl\}\_2 molecules.
                    
                
                \textbackslash\{\}qquad
                    \textbackslash\{\}dfrac\{\textbackslash\{\}text\{Na\}\}\{\textbackslash\{\}text\{Cl\}\_2\} = $\frac{2}{1}$ =
                    $\frac{x}{\green{\text{ [[roundTo(3, R2_MASS / R2_MOLAR_MASS)]] [[plural_form(MOLE, R2_MOL)]]}$\}\}
                \textbackslash\{\}qquad
                
                [Show alternate approach]
                
                
                    
                        
                        \textbackslash\{\}dfrac\{\textbackslash\{\}text\{Na\}\}\{\textbackslash\{\}text\{Cl\}\_2\} = $\frac{2}{1}$ =
                        \textbackslash\{\}dfrac\{\textbackslash\{\}blue\{\textbackslash\{\}text\{ [[roundTo(3, R1\_MASS / R1\_MOLAR\_MASS)]] [[plural\_form(MOLE, R1\_MOL)]]\}\}\}\{x\}
                        
                    
                    
                        Instead of finding out how much \textbackslash\{\}text\{Na\} we need to react with all of our \textbackslash\{\}text\{Cl\}\_2, we could
                        figure out how much \textbackslash\{\}text\{Cl\}\_2 we need to react with all of our \textbackslash\{\}text\{Na\}. In this case,
                        x = \textbackslash\{\}text\{ [[roundTo(3, R1\_MOL * R2\_RATIO / R1\_RATIO)]] [[plural\_form(MOLE, roundTo(3, R1\_MOL * R2\_RATIO / R1\_RATIO))]]\} of
                        \textbackslash\{\}text\{Cl\}\_2 needed, which
                        is more than we have. Therefore \textbackslash\{\}text\{Cl\}\_2 is the limiting reagent.
                    
                
                
                    x = \textbackslash\{\}text\{ [[roundTo(3, R2\_MOL * R1\_RATIO / R2\_RATIO)]] [[plural\_form(MOLE, roundTo(3, R2\_MOL * R1\_RATIO / R2\_RATIO))]]\} of
                    \textbackslash\{\}text\{Na\} needed.
                    We have \textbackslash\{\}text\{ [[roundTo(3, R1\_MASS / R1\_MOLAR\_MASS)]] [[plural\_form(MOLE, R1\_MOL)]]\} of \textbackslash\{\}text\{Na\}, which is more
                    than we need. Therefore \textbackslash\{\}text\{Cl\}\_2 is the limiting reagent.
                
            

            
                
                    The mole ratio of \textbackslash\{\}dfrac\{\textbackslash\{\}text\{Cl\}\_2\}\{\textbackslash\{\}text\{NaCl\}\} in the reaction is
                    $\frac{1}{2}$.
                    [Explain]
                
                
                    
                        The reaction is 2\textbackslash\{\}text\{Na\} +
                        \textbackslash\{\}blue\{1\}\textbackslash\{\}text\{Cl\}\_2 \textbackslash\{\}rightarrow
                        \textbackslash\{\}red\{2\}\textbackslash\{\}text\{NaCl\}
                        .
                        The coefficients in front of each molecule tell us in what ratios the molecules react. In this case
                        [[cardinalThrough20(R2\_RATIO)]] \textbackslash\{\}text\{Cl\}\_2 for every
                        [[cardinalThrough20(P1\_RATIO)]] \textbackslash\{\}text\{NaCl\} molecules.
                    
                
                \textbackslash\{\}qquad
                    \textbackslash\{\}dfrac\{\textbackslash\{\}text\{Cl\}\_2\}\{\textbackslash\{\}text\{NaCl\}\} = $\frac{1}{2}$ =
                    \textbackslash\{\}dfrac\{\textbackslash\{\}green\{\textbackslash\{\}text\{ [[roundTo(3, R2\_MASS / R2\_MOLAR\_MASS)]] [[plural\_form(MOLE, R2\_MOL)]]\}\}\}\{x\}
                
                
                    x = \textbackslash\{\}text\{ [[roundTo(3, R1\_LIMIT ? R1\_MOL * P1\_RATIO / R1\_RATIO : R2\_MOL * P1\_RATIO / R2\_RATIO)]] [[plural\_form(MOLE, P1\_MOL)]]\} of \textbackslash\{\}text\{NaCl\} produced.
  \item \textbackslash\{\}cancel\{\textbackslash\{\}text\{[[roundTo(3, R1\_LIMIT ? R1\_MOL * P1\_RATIO / R1\_RATIO : R2\_MOL * P1\_RATIO / R2\_RATIO)]] [[plural\_form(MOLE, P1\_MOL)]]\}\}
            \textbackslash\{\}text\{NaCl\} $\times$ \textbackslash\{\}dfrac\{[[roundTo(3, molarMass("Na") + molarMass("Cl"))]] \textbackslash\{\}text\{g\}\}\{\textbackslash\{\}cancel\{\textbackslash\{\}text\{[[plural\_form(MOLE, 1)]]\}\}\} =
            \textbackslash\{\}text\{ [[roundTo(3, P1\_MOL * P1\_MOLAR\_MASS)]] [[plural\_form(GRAM, P1\_MASS)]]\} \textbackslash\{\}text\{ of \}\textbackslash\{\}text\{NaCl\}
\end{itemize}
\end{document}
