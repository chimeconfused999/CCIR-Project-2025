% Auto-converted from khan-exercises
\documentclass{article}
\usepackage{amsmath,amssymb}
\usepackage[T1]{fontenc}
\usepackage{textcomp}
\newcommand{\abs}[1]{\lvert #1\rvert}

\begin{document}
\section*{Caesar cipher frequency analysis}
\textbf{Question.} You intercept the following message, which you know to be encrypted with a Caesar cipher:
                    [[applyCaesar(M,SHIFT)]]
                    What shift was applied to the original message?

                    

                    You can use the graph and slider below to help solve this problem.
                    Slide the orange dot to adjust the left shift of this message.

\textbf{Answer.} 5

\textbf{Hints.}
\begin{itemize}
  \item The letter frequency of the shifted cipher message should
            be similar to the letter frequency of English in the graph.
  \item Since the original message was shifted right by some value, we want to shift the
                cipher message frequency
                to the left, until the frequencies are similarly matched.
  \item Try to find patterns of frequencies. For instance the letters a, e, and
                h show up frequently in English, and appear close together in the alphabet.
                The letters j, q, and z hardly show up at all.
  \item You can use the left shift slider to see what the original message would look like when decrypting it with a specific shift. Drag the
                orange dot to where you think the encrypted letter 'a' would be
                in the letter frequency graph.
  \item Adjust the shift to the letter f to see the message
  \item f is 5 letters to the
                right of a. So the original message was shifted by 5
\end{itemize}
\end{document}
