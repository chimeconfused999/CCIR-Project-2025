% Auto-converted from khan-exercises
\documentclass{article}
\usepackage{amsmath,amssymb}
\usepackage[T1]{fontenc}
\usepackage{textcomp}
\newcommand{\abs}[1]{\lvert #1\rvert}

\begin{document}
\section*{Adding and subtracting decimals word problems}
\textbf{Question.} \{On a sunny morning|On a beautiful afternoon\}, [[person(1)]]
                    rode his bicycle to a farm that sold baskets of [[plural\_form(fruit(1))]] for $[[fruit_1]]
                    each and baskets of [[plural_form(fruit(2))]] for $[[fruit\_2]] each.
                
                    \{On a sunny morning|On a beautiful afternoon\}, [[person(1)]]
                    rode her bicycle to a farm that sold baskets of [[plural\_form(fruit(1))]] for $[[fruit_1]]
                    each and baskets of [[plural_form(fruit(2))]] for $[[fruit\_2]] each.
                
                
                    [[person(1)]] decided to buy a basket of [[plural\_form(fruit(1))]] and a basket of [[plural\_form(fruit(2))]] \{before heading home$| because those were his favorite kinds of fruit |$\}.
                
                    [[person(1)]] decided to buy a basket of [[plural\_form(fruit(1))]] and a basket of [[plural\_form(fruit(2))]] \{before heading home$| because those were her favorite kinds of fruit |$\}.
                
                How much did [[person(1)]] need to pay for his produce?
                How much did [[person(1)]] need to pay for her produce?

\textbf{Answer.} $[[fruit_1 + fruit_2]]$

\textbf{Hints.}
\begin{itemize}
  \item To find the total amount [[person(1)]] needs to pay,
                    we need to add the price of the [[plural\_form(fruit(1))]] and the price of the [[plural\_form(fruit(2))]].
  \item Price of [[plural\_form(fruit(1))]] + price of [[plural\_form(fruit(2))]] = total price.
  \item [[person(1)]] needs to pay $[[solution]].$
  \item To find how much faster [[person(2)]] was than [[person(1)]],
                    we need to find the difference between their times in seconds.
  \item [[person(1)]]'s time - [[person(2)]]'s time = difference in times.
  \item [[person(2)]] was [[solution]] seconds faster than [[person(1)]].
  \item To find the weights of the two babies, we need to add their weights together.
  \item [[person(2)]]'s weight + [[person(3)]]'s weight = total weight.
  \item Together, the babies weigh [[solution]] kilograms.
  \item To find out how much change [[person(1)]] received, we can subtract the price of the
                    [[storeItem(1,1)]] from the amount of money he paid.
  \item To find out how much change [[person(1)]] received, we can subtract the price of the
                    [[storeItem(1,1)]] from the amount of money she paid.
  \item The amount [[person(1)]] paid - the price of the [[storeItem(1,1)]] = 
                    the amount of change [[person(1)]] received.
  \item [[person(1)]] received $[[solution]] in change.$
  \item To find the difference in rainfall, we can subtract the amount of rain in [[person(1)]]'s
                    town from the amount of rain in [[person(2)]]'s town.
  \item Rain in [[person(2)]]'s town - rain in [[person(1)]]'s town = the difference in rain between the two towns.
  \item [[person(2)]]'s town received [[solution]] centimeters more rain than [[person(1)]]'s town.
  \item To find the total distance [[person(1)]] travels, we need to add the two distances together.
  \item Distance on [[vehicle(1)]] + distance on [[vehicle(2)]] = total distance.
  \item [[person(1)]] travels [[solution]] [[plural\_form(distance(1), solution)]] in total.
\end{itemize}
\end{document}
