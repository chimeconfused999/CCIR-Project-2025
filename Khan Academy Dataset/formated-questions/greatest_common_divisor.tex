% Auto-converted from khan-exercises
\documentclass{article}
\usepackage{amsmath,amssymb}
\usepackage[T1]{fontenc}
\usepackage{textcomp}
\newcommand{\abs}[1]{\lvert #1\rvert}

\begin{document}
\section*{Greatest common factor}
\textbf{Question.} What is the greatest common factor of [[A]] and [[B]]?
                    Another way to say this is: 
                    \textbackslash\{\}operatorname\{[[GCF\_TEXT]]\}([[A]], [[B]]) = \{?\}

\textbf{Answer.} [[GCD]]

\textbf{Hints.}
\begin{itemize}
  \item The greatest common factor is the largest number that is a factor of both [[A]] and [[B]].
  \item The only factor of 1 is 1.
                The factors of [[A]] are [[A\_FACTORS]].
  \item The only factor of 1 is 1.
                The factors of [[B]] are [[B\_FACTORS]].
  \item Thus, the greatest common factor of [[A]] and [[B]] is [[GCD]].
                \textbackslash\{\}operatorname\{[[GCF\_TEXT]]\}([[A]], [[B]]) = [[GCD]]
\end{itemize}
\end{document}
