% Auto-converted from khan-exercises
\documentclass{article}
\usepackage{amsmath,amssymb}
\usepackage[T1]{fontenc}
\usepackage{textcomp}
\newcommand{\abs}[1]{\lvert #1\rvert}

\begin{document}
\section*{Basic set notation}
\textbf{Question.} What is the set X \textbackslash\{\}cup Y?

\textbf{Answer.} Enter the answer as a set, with members separated by commas.
                        \textbackslash\{\}\{  \textbackslash\{\}\}
                    
                    $('#solutionarea input').val() 
                    
                        var guessTrim = guess.replace(/\s/g, "");
                        var guessArray = guessTrim.length ? _.map(guessTrim.split(","), function(n) { return parseInt(n); }) : [];
                        return _.isEqual(sortNumbers(guessArray), ANSWER);
                    
                    $('\#solutionarea input').val(guess)
                
                
                    
                        
                            [[EMPTY\_SET]]
                        
                        Empty set
                    
                
                
                
                    
                        $("#solutionarea input").eq(0).val() === "" ^
                        !$("\#solutionarea input").eq(1).is(":checked")
                    
                    
                        return guess ? true : "";

\textbf{Hints.}
\begin{itemize}
  \item Remember that \textbackslash\{\}cup refers to the union of sets.
  \item The union of two sets X and Y is the collection of elements which are in X or in Y or in both X and Y.
  \item The members of a set must be unique, and the order doesn't matter.
  \item X \textbackslash\{\}cup Y = \textbackslash\{\}\{[[ANSWER]]\textbackslash\{\}\}
  \item Remember that \textbackslash\{\}backslash refers to the difference between sets.
  \item The difference of two sets X and Y is the collection of elements which are in X but not in Y.
  \item The members of a set must be unique, and the order doesn't matter.
  \item X \textbackslash\{\}setminus Y = \textbackslash\{\}\{[[ANSWER]]\textbackslash\{\}\}
  \item Remember that \textbackslash\{\}cap refers to the intersection of sets.
  \item The intersection of two sets X and Y is the collection of elements which are in X and also in Y.
  \item The members of a set must be unique, and the order doesn't matter.
  \item X \textbackslash\{\}cap Y = \textbackslash\{\}\{[[ANSWER]]\textbackslash\{\}\}
\end{itemize}
\end{document}
