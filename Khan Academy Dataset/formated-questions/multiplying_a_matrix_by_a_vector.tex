% Auto-converted from khan-exercises
\documentclass{article}
\usepackage{amsmath,amssymb}
\usepackage[T1]{fontenc}
\usepackage{textcomp}
\newcommand{\abs}[1]{\lvert #1\rvert}

\begin{document}
\section*{Multiplying a matrix by a vector}
\textbf{Question.} What is [[PRETTY\_MAT\_1\_ID + PRETTY\_MAT\_2\_ID]]?

\textbf{Answer.} [[elem]]
                
                
                    [[elem]]

\textbf{Hints.}
\begin{itemize}
  \item Because [[PRETTY\_MAT\_1\_ID]] has dimensions ([[DIM\_1 + "\textbackslash\{\}$\times$" + DIM\_2]]) and [[PRETTY\_MAT\_2\_ID]] has dimensions ([[DIM\_3 + "\textbackslash\{\}$\times$" + DIM\_4]]), the answer matrix will have dimensions ([[DIM\_1 + "\textbackslash\{\}$\times$" + DIM\_4]]).
            
            
                
                    [[PRETTY\_MAT\_1\_ID + PRETTY\_MAT\_2\_ID]]
                    =
                    [[printColoredDimMatrix(MAT\_1, ROW\_COLORS, true)]]
                    [[printColoredDimMatrix(MAT\_2, COL\_COLORS, false)]]
                    =
                    [[printSimpleMatrix(maskMatrix(FINAL\_HINT\_MAT, []))]]
  \item To find the element at any row i, column j of the answer matrix, multiply the elements in row i of the first matrix, [[PRETTY\_MAT\_1\_ID]], with the corresponding elements in column j of the second matrix, [[PRETTY\_MAT\_2\_ID]], and add the products together.
  \item So, to find the element at row 1, column 1 of the answer matrix, multiply the first element in [[colorMarkup("\textbackslash\{\}\textbackslash\{\}text\{" + ROW + " \}1", ROW\_COLORS[0])]] of [[PRETTY\_MAT\_1\_ID]] with the first element in [[colorMarkup("\textbackslash\{\}\textbackslash\{\}text\{" + COLUMN + " \}1", COL\_COLORS[0])]] of [[PRETTY\_MAT\_2\_ID]], then multiply the second element in [[colorMarkup("\textbackslash\{\}\textbackslash\{\}text\{" + ROW + " \}1", ROW\_COLORS[0])]] of [[PRETTY\_MAT\_1\_ID]] with the second element in [[colorMarkup("\textbackslash\{\}\textbackslash\{\}text\{" + COLUMN + " \}1", COL\_COLORS[0])]] of [[PRETTY\_MAT\_2\_ID]], and so on. Add the products together.
            
            
                
                    [[printSimpleMatrix(
                            maskMatrix(FINAL\_HINT\_MAT, [[1, 1]])
                        )]]
  \item Likewise, to find the element at row 2, column 1 of the answer matrix, multiply the elements in [[colorMarkup("\textbackslash\{\}\textbackslash\{\}text\{" + ROW + " \}2", ROW\_COLORS[1])]] of [[PRETTY\_MAT\_1\_ID]] with the corresponding elements in [[colorMarkup("\textbackslash\{\}\textbackslash\{\}text\{" + COLUMN + " \}1", COL\_COLORS[0])]] of [[PRETTY\_MAT\_2\_ID]] and add the products together.
            
            
                
                    [[printSimpleMatrix(
                            maskMatrix(FINAL\_HINT\_MAT, [[1, 1], [2, 1]])
                        )]]
  \item Fill out the rest:
            
            
                
                    [[printSimpleMatrix(FINAL\_HINT\_MAT)]]
  \item After simplifying, we end up with:
            
                
                    [[printSimpleMatrix(SOLN\_MAT)]]
\end{itemize}
\end{document}
