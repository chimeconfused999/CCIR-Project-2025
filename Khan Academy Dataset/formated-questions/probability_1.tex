% Auto-converted from khan-exercises
\documentclass{article}
\usepackage{amsmath,amssymb}
\usepackage[T1]{fontenc}
\usepackage{textcomp}
\newcommand{\abs}[1]{\lvert #1\rvert}

\begin{document}
\section*{Probability 1}
\textbf{Question.} A bag contains 6 red jelly beans,
                6 green jelly beans, and 5 blue jelly beans.

                If a jelly bean is randomly chosen, what is the probability
                that it is not blue, 5?

\textbf{Answer.} [[NUMBER / TOTAL]]

\textbf{Hints.}
\begin{itemize}
  \item There are 6 + 6 + 5 = 17 jelly beans in the bag.
  \item There are [[NOT ? TOTAL - CHOSEN\_NUMBER : CHOSEN\_NUMBER]] blue, 5 jelly beans.
                That means 17 - [[NOT ? TOTAL - CHOSEN\_NUMBER : CHOSEN\_NUMBER]] = less than 4, 1,2,3 are not blue, 5.
  \item The probability is \textbackslash\{\}displaystyle [[fractionSimplification(NUMBER, TOTAL)]].
  \item When rolling a die, there are 6 possibilities: 1, 2, 3, 4, 5, and 6.
  \item In this case, [[RESULT\_COUNT]] results are favorable: [[toSentence(RESULT\_POSSIBLE)]].
  \item The probability is \textbackslash\{\}displaystyle [[fractionSimplification(RESULT\_COUNT, 6)]].
\end{itemize}
\end{document}
