% Auto-converted from khan-exercises
\documentclass{article}
\usepackage{amsmath,amssymb}
\usepackage[T1]{fontenc}
\usepackage{textcomp}
\newcommand{\abs}[1]{\lvert #1\rvert}

\begin{document}
\section*{Equivalent fractions 2}
\textbf{Question.} What number could replace [[SYMBOL]] below?

\textbf{Answer.} [[D]]

\textbf{Hints.}
\begin{itemize}
  \item The fraction on the left represents dividing some rectangular [[pizza(1).plural(2)]] into [[B]] slices,
                            then taking [[A]] slices.
  \item How many slices would we need to cut each [[pizza(1)]] into so that [[C]]
                            slices would give us the same amount of [[pizza(1)]]?
  \item Each of the original [[A]] slices must be divided into [[M]]
                            slices to get [[C]] slices in total.
  \item If we divide all the original slices into [[M]] slices, then one
                            [[pizza(1)]] will have a total of [[D]] slices.
  \item $\frac{[[A]]}{[[B]]}$ = $\frac{[[C]]}{[[D]]}$ and so the answer is [[D]].
  \item Another way to get the answer is to multiply by $\frac{[[M]]}{[[M]]}$.
                        $\frac{[[M]]}{[[M]]}$ = $\frac{1}{1}$ = 1 so really we are multiplying by 1.
  \item The final equation is: $\frac{[[A]]}{[[B]]}$ $\times$ $\frac{[[M]]}{[[M]]}$ =
                        $\frac{[[C]]}{[[D]]}$  so our answer is [[D]].
  \item The fraction on the left represents dividing some rectangular [[pizza(1).plural(2)]] into [[B]] slices,
                            then taking [[A]] slices.
  \item What if we cut each [[pizza(1)]] into [[D]] slices instead?
  \item In order to take the same amount of [[pizza(1)]] as before,
                            we now need to take [[C]] slices.
  \item $\frac{[[A]]}{[[B]]}$ = $\frac{[[C]]}{[[D]]}$ and so the answer is [[C]].
  \item Another way to get the answer is to multiply by $\frac{[[M]]}{[[M]]}$.
                        $\frac{[[M]]}{[[M]]}$ = $\frac{1}{1}$ = 1 so really we are multiplying by 1.
  \item The final equation is: $\frac{[[A]]}{[[B]]}$ $\times$ $\frac{[[M]]}{[[M]]}$ =
                        $\frac{[[C]]}{[[D]]}$  so our answer is [[C]].
  \item The fraction on the left represents dividing some rectangular [[pizza(1).plural(2)]] into [[D]] slices,
                            then taking [[C]] slices.
  \item If we share those [[C]] slices equally between [[A]] person,
                            how many slices does each person get?
                        
                            If we share those [[C]] slices equally between [[A]] people,
                            how many slices does each person get?
  \item Sharing [[C]] slices equally between [[A]] person means each person gets [[M]] slices.
                        
                            Sharing [[C]] slices equally between [[A]] people means each person gets [[M]] slices.
                        
                        
                        If we give each person [[M]] slices, how many people can we feed with one [[pizza(1)]]?
  \item One [[pizza(1)]] has [[D]] slices, so if we give each person
                            [[M]] slices, we could feed [[B]] people.
  \item $\frac{[[C]]}{[[D]]}$ = $\frac{[[A]]}{[[B]]}$ and so the answer is [[B]].
  \item Another way to get the answer is to divide by $\frac{[[M]]}{[[M]]}$.
                        $\frac{[[M]]}{[[M]]}$ = $\frac{1}{1}$ = 1 so really we are dividing by 1.
  \item The final equation is: $\frac{[[C]]}{[[D]]}$ \textbackslash\{\}div $\frac{[[M]]}{[[M]]}$ =
                        $\frac{[[A]]}{[[B]]}$ so our answer is [[B]].
  \item The fraction on the left represents dividing some rectangular [[pizza(1).plural(2)]] into [[D]] slices,
                            then taking [[C]] slices.
  \item What if we cut each [[pizza(1)]] into [[B]] slices instead?
  \item In order to take the same amount of [[pizza(1)]] as before,
                            we now need to take only [[A]] slices.
  \item $\frac{[[C]]}{[[D]]}$ = $\frac{[[A]]}{[[B]]}$ and so the answer is [[A]].
  \item Another way to get the answer is to divide by $\frac{[[M]]}{[[M]]}$.
                        $\frac{[[M]]}{[[M]]}$ = $\frac{1}{1}$ = 1 so really we are dividing by 1.
  \item The final equation is: $\frac{[[C]]}{[[D]]}$ \textbackslash\{\}div $\frac{[[M]]}{[[M]]}$ =
                        $\frac{[[A]]}{[[B]]}$ so our answer is [[A]].
\end{itemize}
\end{document}
