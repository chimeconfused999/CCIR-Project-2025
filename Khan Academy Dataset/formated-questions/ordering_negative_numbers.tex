% Auto-converted from khan-exercises
\documentclass{article}
\usepackage{amsmath,amssymb}
\usepackage[T1]{fontenc}
\usepackage{textcomp}
\newcommand{\abs}[1]{\lvert #1\rvert}

\begin{document}
\section*{Ordering negative numbers}
\textbf{Question.} Order the following integers from least to greatest:
                    
                        
                            [[NUM]]
                            [[NUM]]
                        
                    
                    [[SORTER.init("sortable")]]

\textbf{Answer.} SORTER.getContent()
                    
                        if (SORTER.hasAttempted) \{
                            return guess.join(",") === NUMS\_SORT.join(",");
                        \} else \{
                            return "";
                        \}
                    
                    
                        SORTER.setContent(guess);

\textbf{Hints.}
\begin{itemize}
  \item Let's use different colors for each number.
                        [[$.map(NUMS, function(n) {
                return "\\color{" + NUM_COLORS[n] + "}{" + n + "}";
            } ).join( "," )]]$
  \item Plot these numbers on a number line. Then we can see which ones are lower and which ones are higher.
  \item Now just read the numbers from left to right on the number line.
                        The leftmost numbers are least, and the rightmost numbers are greatest.
  \item [[$.map(NUMS_SORT, function(n) {
                return "\\color{" + NUM_COLORS[n] + "}{" + n + "}";
            } ).join( "," )]]$
\end{itemize}
\end{document}
