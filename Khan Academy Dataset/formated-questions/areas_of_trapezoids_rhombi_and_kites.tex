% Auto-converted from khan-exercises
\documentclass{article}
\usepackage{amsmath,amssymb}
\usepackage[T1]{fontenc}
\usepackage{textcomp}
\newcommand{\abs}[1]{\lvert #1\rvert}

\begin{document}
\section*{Area of trapezoids}
\textbf{Question.} What is the area of this figure?

\textbf{Answer.} [[K]]
                    square [[plural\_form(UNIT\_TEXT)]]

\textbf{Hints.}
\begin{itemize}
  \item This figure is a quadrilateral with a pair of parallel sides (the top and bottom sides), so it's a trapezoid.
  \item area of a trapezoid = \textbackslash\{\}dfrac12 $\cdot$ (b\_1 + b\_2) $\cdot$ h
                            [Show me why]
                        
                        
                            Let's draw a line between the opposite ends of the two bases.
                            
                            Notice that the line divides the trapezoid into two triangles: one triangle with base b\_1 = 4, and another triangle with base b\_2 = 3. Both triangles have height h = -1.
                            The area of the trapezoid is equal to the sum of the areas of the two triangles.
                            A = \textbackslash\{\}dfrac12 $\cdot$ b\_1 $\cdot$ h + \textbackslash\{\}dfrac12 $\cdot$ b\_2 $\cdot$ h
                            Factor out \textbackslash\{\}dfrac12 $\cdot$ h to get the formula for the area of a trapezoid:
                            A = \textbackslash\{\}dfrac12 $\cdot$ h $\cdot$ (b\_1 + b\_2) = \textbackslash\{\}dfrac12 $\cdot$ (b\_1 + b\_2) $\cdot$ h
  \item Now use this formula to calculate the trapezoid's area.
                        b\_1 = 4
                        b\_2 = 3
                        h = -1
                        A = \textbackslash\{\}dfrac12 $\cdot$ (4 + 3) $\cdot$ -1 = [[K]]
\end{itemize}
\end{document}
