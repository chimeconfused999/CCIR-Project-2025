% Auto-converted from khan-exercises
\documentclass{article}
\usepackage{amsmath,amssymb}
\usepackage[T1]{fontenc}
\usepackage{textcomp}
\newcommand{\abs}[1]{\lvert #1\rvert}

\begin{document}
\section*{Converting decimals to fractions 1}
\textbf{Question.} Express [[localeToFixed(D, 1)]] as a fraction.

\textbf{Answer.} [[D]]

\textbf{Hints.}
\begin{itemize}
  \item The number [[T]] is in the tenths place,
                    so we have [[cardinalThrough20(T)]] [[plural\_form(decimalPlaceNames[1], T)]].
  \item [[CardinalThrough20(T)]] [[plural\_form(decimalPlaceNames[1], T)]]
                    can be written as [[fraction(T, 10)]].
  \item The number [[H]] is in the hundredths place,
                    so we have [[cardinalThrough20(H)]] [[plural\_form(decimalPlaceNames[2], H)]].
  \item [[CardinalThrough20(H)]] [[plural\_form(decimalPlaceNames[2], H)]]
                    can be written as [[fraction(H, 100)]].
  \item Add the two parts together.
                    [[fraction( T, 10 )]] + [[fraction( H, 100 )]]
  \item = [[fraction( T * 10, 100 )]] + [[fraction( H, 100 )]]
                    = [[fraction( T * 10 + H, 100 )]]
\end{itemize}
\end{document}
