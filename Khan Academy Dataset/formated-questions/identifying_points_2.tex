% Auto-converted from khan-exercises
\documentclass{article}
\usepackage{amsmath,amssymb}
\usepackage[T1]{fontenc}
\usepackage{textcomp}
\newcommand{\abs}[1]{\lvert #1\rvert}

\begin{document}
\section*{Ordered pair solutions to linear equations}
\textbf{Question.} Which of the following ordered pairs represents a solution to the equation below?

\textbf{Answer.} [[PAIR( POINTS[CORRECT] )]]

\textbf{Hints.}
\begin{itemize}
  \item We can try plugging in the x-value of each ordered pair into the equation.
  \item If we evaluate and get the y-value of the ordered pair, then that ordered pair is a solution!
  \item Let's consider [[PAIR( point )]].
                            If we plug in [[point[ 0 ]]] for x and evaluate, do we get [[point[ 1 ]]]?
                        
                        y = ([[M]])([[point[ 0 ]]]) + [[B]] = [[M * point[ 0 ]]] + [[B]] = [[M * point[ 0 ] + B]]
  \item Thus the only ordered pair that is a solution to the equation is [[PAIR( POINTS[ CORRECT ] )]].
  \item We come to the same answer by plotting the points and the equation.
  \item Let's try graphing each of the points.
  \item The only point that falls on the line is [[PAIR( POINTS[ CORRECT ] )]].
\end{itemize}
\end{document}
