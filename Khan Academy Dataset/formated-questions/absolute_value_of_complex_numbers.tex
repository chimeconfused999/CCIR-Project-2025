% Auto-converted from khan-exercises
\documentclass{article}
\usepackage{amsmath,amssymb}
\usepackage[T1]{fontenc}
\usepackage{textcomp}
\newcommand{\abs}[1]{\lvert #1\rvert}

\begin{document}
\section*{Absolute value of complex numbers}
\textbf{Question.} Determine the absolute value of the following complex number:

\textbf{Answer.} 34

\textbf{Hints.}
\begin{itemize}
  \item The absolute value of any number is its distance from zero.
                    As complex numbers can be visualized as points on the complex plane, absolute values
                    of complex numbers can be determined using the distance formula.
  \item 3-5i is plotted as a blue circle above.
  \item The absolute value we need is the length of the orange line segment.
  \item The orange line segment is the hypotenuse of a right triangle.
                        Its two legs (shown in blue) have lengths 3 and 5, which corresponds
                        to the absolute values of the real and imaginary parts of the complex number 3-5i.
  \item Substituting into the Pythagorean theorem:
                    \textbackslash\{\}qquad $| 3-5i |$\textasciicircum{}2 = 3\textasciicircum{}2 + 5\textasciicircum{}2, so 
                    \textbackslash\{\}qquad $| 3-5i |$ = $\sqrt{3^2 + 5^2}$.
  \item \textbackslash\{\}qquad $\sqrt{3^2 + 5^2}$ = $\sqrt{9 + 25}$ = $\sqrt{34}$
  \item Simplifying the radical gives $\sqrt{34}$. That is the absolute value of 3-5i.
  \item The radical cannot be simplified. The absolute value of 3-5i is $\sqrt{34}$.
\end{itemize}
\end{document}
