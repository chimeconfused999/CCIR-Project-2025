% Auto-converted from khan-exercises
\documentclass{article}
\usepackage{amsmath,amssymb}
\usepackage[T1]{fontenc}
\usepackage{textcomp}
\newcommand{\abs}[1]{\lvert #1\rvert}

\begin{document}
\section*{Positive and zero exponents}
\textbf{Question.} \textbackslash\{\}Large\{[[negParens( BASE )]]\textasciicircum{}\{[[VALS.exp]]\} = \{?\}\}

\textbf{Answer.} [[round( pow( BASE, EXP ) )]]

\textbf{Hints.}
\begin{itemize}
  \item Anything to the 1st power equals... ?
                        x\textasciicircum{}\{1\} = x, no matter what x is.
                        [[negParens(BASE)]]\textasciicircum{}\{1\} = [[VALS.base]]
  \item Negative one to any even power equals... ?
                            Negative one to any even power equals one.
                            [[negParens(BASE)]]\textasciicircum{}\{[[VALS.exp]]\} = 1
                        
                            Negative one to any odd power equals... ?
                            Negative one to any odd power equals negative one.
                            [[negParens(BASE)]]\textasciicircum{}\{[[VALS.exp]]\} = -1
  \item [[CardinalThrough20( BASE )]] to any power equals... ?
                        [[CardinalThrough20( BASE )]] to any power equals [[cardinalThrough20(round(pow(BASE, EXP)))]].
                        [[negParens(BASE)]]\textasciicircum{}\{[[VALS.exp]]\} = [[round(pow(BASE, EXP))]]
  \item = [[v]]
\end{itemize}
\end{document}
