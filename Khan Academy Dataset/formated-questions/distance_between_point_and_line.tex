% Auto-converted from khan-exercises
\documentclass{article}
\usepackage{amsmath,amssymb}
\usepackage[T1]{fontenc}
\usepackage{textcomp}
\newcommand{\abs}[1]{\lvert #1\rvert}

\begin{document}
\section*{Distance between point and line}
\textbf{Question.} Find the minimum distance between the point \textbackslash\{\}red\{([[X1]], [[Y1]])\}
                        and the line \textbackslash\{\}blue\{y =
                        [[M2\_FRAC]]
                        [[M2\_SIGN]]x
                        + [[B2]]\}.

\textbf{Answer.} [[DISTANCE]]

\textbf{Hints.}
\begin{itemize}
  \item First, find the equation of the perpendicular line that passes through \textbackslash\{\}red\{([[X1]], [[Y1]])\}.
  \item The slope of the blue line is \textbackslash\{\}blue\{[[M2\_FRAC]]\},
                            and its negative reciprocal is \textbackslash\{\}green\{[[M1\_FRAC]]\}.
                        
                        
                            Thus, the equation of our perpendicular line will be of the form \textbackslash\{\}green\{y =
                            [[M1\_FRAC]]
                            [[M1\_SIGN]]x + b\}.
  \item We can plug our point, \textbackslash\{\}red\{([[X1]], [[Y1]])\},
                            into this equation to solve for \textbackslash\{\}green\{b\}, the y-intercept.
                        
                        
                            [[Y1]] = \textbackslash\{\}green\{
                            [[M1\_FRAC]]
                            [[M1\_SIGN]]\}([[X1]]) + \textbackslash\{\}green\{b\}
  \item [[Y1]] = [[decimalFraction( M1 * X1, "true", "true" )]] + \textbackslash\{\}green\{b\}
                        [[Y1]] - [[decimalFraction( M1 * X1, "true", "true" )]] = \textbackslash\{\}green\{b\} = [[decimalFraction( Y1 - M1 * X1, "true", "true" )]]
  \item The equation of the perpendicular line is \textbackslash\{\}green\{y =
                            [[M1\_FRAC]]
                            [[M1\_SIGN]]x
                             + [[B1]]\}.
  \item We can see from the graph (or by setting the equations equal to one another) that the two lines intersect at the point \textbackslash\{\}red\{([[X2]], [[Y2]])\}. Thus, the distance we're looking for is the distance between the two red points.
  \item The distance between two points is: $\sqrt{( x_{1}$ - x\_\{2\} )\textasciicircum{}2 + ( y\_\{1\} - y\_\{2\} )\textasciicircum{}2\}
  \item Plugging in our points \textbackslash\{\}red\{([[X1]], [[Y1]])\} and \textbackslash\{\}red\{([[X2]], [[Y2]])\} gives us: $\sqrt{( \red{[[X1]]}$ - \textbackslash\{\}red\{[[X2]]\} )\textasciicircum{}2 + ( \textbackslash\{\}red\{[[Y1]]\} - \textbackslash\{\}red\{[[Y2]]\} )\textasciicircum{}2\}
  \item = $\sqrt{( [[X1 - X2]] )^2 + ( [[Y1 - Y2]] )^2}$ = $\sqrt{[[DISTANCE]]}$  = [[formattedSquareRootOf( DISTANCE )]]
  \item So the minimum distance between the point \textbackslash\{\}red\{([[X1]], [[Y1]])\}
                        and the line \textbackslash\{\}blue\{y =
                        [[M2\_FRAC]]
                        [[M2\_SIGN]]x
                        + [[B2]]\}
                        is [[formattedSquareRootOf(DISTANCE)]].
  \item First, find the equation of the perpendicular line that passes through \textbackslash\{\}red\{([[X1]], [[Y1]])\}.
  \item Since the slope of the blue line is \textbackslash\{\}blue\{0\}, the perpendicular line will have an infinite slope and therefore will be a vertical line.
  \item The equation of the vertical line that passes through \textbackslash\{\}red\{([[X1]], [[Y1]])\} is \textbackslash\{\}green\{x = [[X1]]\}.
  \item We can see from the graph that the two lines intersect at the point \textbackslash\{\}red\{([[X1]], [[B1]])\}. Thus, the distance we're looking for is the distance between the two red points.
  \item Since their x components are the same, the distance between the two points is simply the change in y:
  \item $| \red{[[Y1]]} - ( \red{[[B1]]} ) |$ = [[abs( Y1 - B1 )]]
  \item So the minimum distance between the point \textbackslash\{\}red\{([[X1]], [[Y1]])\} and the line \textbackslash\{\}blue\{y = [[B1]]\} is [[abs( Y1 - B1 )]].
  \item First, find the equation of the perpendicular line that passes through \textbackslash\{\}red\{([[X1]], [[Y1]])\}.
  \item Since the blue line has an infinite slope, the perpendicular line will have a slope of \textbackslash\{\}green\{0\} and therefore will be a horizontal line.
  \item The equation of the perpendicular line that passes through \textbackslash\{\}red\{([[X1]], [[Y1]])\} is \textbackslash\{\}green\{y = [[Y1]]\}.
  \item We can see from the graph that the two lines intersect at the point \textbackslash\{\}red\{([[B1]], [[Y1]])\}. Thus, the distance we're looking for is the distance between the two red points.
  \item Since their y components are the same, the distance between the two points is simply the change in x:
  \item $| \red{[[X1]]} - ( \red{[[B1]]} ) |$ = [[abs( X1 - B1 )]]
  \item So the minimum distance between the point \textbackslash\{\}red\{([[X1]], [[Y1]])\} and the line \textbackslash\{\}blue\{x = [[B1]]\}is [[abs( X1 - B1 )]].
\end{itemize}
\end{document}
