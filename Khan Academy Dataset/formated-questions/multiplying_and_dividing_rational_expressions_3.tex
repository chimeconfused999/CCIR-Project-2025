% Auto-converted from khan-exercises
\documentclass{article}
\usepackage{amsmath,amssymb}
\usepackage[T1]{fontenc}
\usepackage{textcomp}
\newcommand{\abs}[1]{\lvert #1\rvert}

\begin{document}
\section*{Multiplying and dividing rational expressions 3}
\textbf{Question.} $\frac{[[FRACTION1.numerator.string]]}{[[FRACTION1.denominator.string]]}$ $\times$
                    $\frac{[[FRACTION2.numerator.string]]}{[[FRACTION2.denominator.string]]}$

\textbf{Answer.} ([[NUMERSOL.toString()]])/([[DENOMSOL.toString()]])
                ([[NUMERSOL.toString()]])/([[DENOMSOL.toStringFactored()]])
                ([[NUMERSOL.toStringFactored()]])/([[DENOMSOL.toString()]])
                ([[NUMERSOL.toStringFactored()]])/([[DENOMSOL.toStringFactored()]])
                [[NUMERSOL.toString()]]
                [[NUMERSOL.toStringFactored()]]
            
            [[randVar()]] \textbackslash\{\}neq \textbackslash\{\}space
            -2.5

\textbf{Hints.}
\begin{itemize}
  \item Dividing by an expression is the same as multiplying by its inverse.
                        \textbackslash\{\}qquad
                            $\frac{[[FRACTION1.numerator.string]]}{[[FRACTION1.denominator.string]]}$ $\times$
                            $\frac{[[FRACTION2.numerator.string]]}{[[FRACTION2.denominator.string]]}$
  \item When multiplying fractions, we multiply the numerators and the denominators.
  \item \textbackslash\{\}qquad \textbackslash\{\}dfrac\{
                ([[FRACTION1.numerator.string]])
                [[FRACTION1.numerator.string]] $\times$
                ([[FRACTION2.numerator.string]])
                [[FRACTION2.numerator.string]] \} \{
                ([[FRACTION1.denominator.string]])
                [[FRACTION1.denominator.string]] $\times$
                ([[FRACTION2.denominator.string]])
                [[FRACTION2.denominator.string]] \}
  \item \textbackslash\{\}qquad \textbackslash\{\}dfrac
                    \{[[FRACTION3.numerator[0]]]
                     $\times$ [[FRACTION3.numerator[1]]]([[new RationalExpression([[COEFFICIENT, X], CONSTANT])]])\}
                    \{[[FRACTION3.denominator[0]]]
                     $\times$ [[FRACTION3.denominator[1]]]([[new RationalExpression([[COEFFICIENT, X], CONSTANT])]])\}
  \item \textbackslash\{\}qquad $\frac{[[DENOMTERM.coefficient > 0 ?
                NUMERTERM.getGCD(DENOMTERM) :
                NUMERTERM.getGCD(DENOMTERM).multiply(-1)]]([[new RationalExpression([[COEFFICIENT, X], CONSTANT])]])}{[[NUMERTERM.divide(FACTOR)]]([[new RationalExpression([[COEFFICIENT, X], CONSTANT])]])}$
  \item We can cancel the [[new RationalExpression([[COEFFICIENT, X], CONSTANT])]] so long as [[new RationalExpression([[COEFFICIENT, X], CONSTANT])]] \textbackslash\{\}neq 0.
  \item Therefore [[randVar()]] \textbackslash\{\}neq $\frac{-5}{2}$.
  \item \textbackslash\{\}qquad
                \textbackslash\{\}dfrac\{[[DENOMTERM.coefficient > 0 ?
                NUMERTERM.getGCD(DENOMTERM) :
                NUMERTERM.getGCD(DENOMTERM).multiply(-1)]] \textbackslash\{\}cancel\{([[new RationalExpression([[COEFFICIENT, X], CONSTANT])]]\})\}\{[[NUMERTERM.divide(FACTOR)]] \textbackslash\{\}cancel\{([[new RationalExpression([[COEFFICIENT, X], CONSTANT])]])\}\}
                = [[writeExpressionFraction(NUMERTERM, DENOMTERM)]]
                =
                    [[writeExpressionFraction(NUMERSOL, DENOMSOL)]]
                    [[NUMERSOL]]
\end{itemize}
\end{document}
