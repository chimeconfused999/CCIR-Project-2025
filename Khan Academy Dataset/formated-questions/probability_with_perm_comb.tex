% Auto-converted from khan-exercises
\documentclass{article}
\usepackage{amsmath,amssymb}
\usepackage[T1]{fontenc}
\usepackage{textcomp}
\newcommand{\abs}[1]{\lvert #1\rvert}

\begin{document}
\section*{Probability with permutations and combinations}
\textbf{Question.} If the teacher picks a group of 3 at random, what is the probability that 
                    everyone in the group is a boy?
                 
                    If the teacher picks a group of 3 at random, what is the probability that 
                    everyone in the group is a girl?

\textbf{Answer.} [[(factorial(NUM\_B)*factorial(STUDENTS-GROUP))/(factorial(STUDENTS)*factorial(NUM\_B-GROUP))]]

\textbf{Hints.}
\begin{itemize}
  \item One way to solve this problem is to figure out how many different groups
                        there are of only boys, then divide this
                        by the total number of groups you could choose. Since every group is chosen
                        with equal probability, this will be the probability that a group of all 
                        boys is chosen.
  \item One way to solve this problem is to figure out how many different groups
                        there are of only girls, then divide this
                        by the total number of groups you could choose. Since every group is chosen
                        with equal probability, this will be the probability that a group of all 
                        girls is chosen.
  \item We know two ways to count the number of groups we can choose: we use permutations if order matters,
                        and combinations if it doesn't. Does the order the students are picked matter in this case?
  \item It doesn't matter if we pick John then 
                            Ben or Ben then 
                            John, 
                            so order must not matter. So, the number of ways
                            to pick a group of 3 students out of 7 is
                             $\frac{7!}{(7-3)!3!}$ = 
                            \textbackslash\{\}binom\{7\}\{3\}.
                            [Show me why]
                        
                            It doesn't matter if we pick Julia then 
                            Beatrice or Beatrice then 
                            Julia, 
                            so order must not matter. So, the number of ways
                            to pick a group of 3 students out of 7 is
                             $\frac{7!}{(7-3)!3!}$ = 
                            \textbackslash\{\}binom\{7\}\{3\}.
                            [Show me why]
                        
                        
                            
                                Remember that the
                                7! \textbackslash\{\}; and (7-3)! \textbackslash\{\}; terms come from when we
                                fill up the group, making 7
                                choices for the first slot, then 6 choices for the
                                second, and so on. In this way, we end up making 
                                7$\cdot$6$\cdot$5 
                                = $\frac{7!}{(7-3)!}$ \textbackslash\{\};.
                                The 3! \textbackslash\{\}; term comes from the number of times we've counted
                                a group as different because we chose the students in a different order.
                                There are 3! \textbackslash\{\}; 
                                ways to order a group of 3, so for every group, we've overcounted exactly
                                that many times.
  \item We can use the same logic to count the number of groups that only have boys.
  \item We can use the same logic to count the number of groups that only have girls.
  \item Specifically, the number of ways to pick a group of 3 students out of
                        2 is
                         $\frac{2!}{(2-3)!3!}$ = 
                        \textbackslash\{\}binom\{2\}\{3\}.
  \item So, the probability that the teacher picks a group of all boys is the number of 
                        groups with only boys divided by the number of total groups the teacher could pick.
  \item So, the probability that the teacher picks a group of all girls is the number of 
                        groups with only girls divided by the number of total groups the teacher could pick.
  \item This is  \textbackslash\{\}displaystyle $\frac{\frac{2!}{(2-3)!\cancel{3!}$\}\}
                        \{$\frac{7!}{(7-3)!\cancel{3!}$\}\} = 
                        $\frac{\frac{2!}{-1!}$\}\{$\frac{7!}{4!}$\}
  \item We can re-arrange the terms to make simplification easier
                        \textbackslash\{\}left($\frac{2!}{-1!}$\textbackslash\{\}right)
                        \textbackslash\{\}left($\frac{4!}{7!}$\textbackslash\{\}right) =
                        \textbackslash\{\}left($\frac{2!}{7!}$\textbackslash\{\}right)
                        \textbackslash\{\}left($\frac{4!}{-1!}$\textbackslash\{\}right)
  \item Simplifying, we get  
                        \textbackslash\{\}left(\textbackslash\{\}dfrac\{\textbackslash\{\}cancel\{2!\}\}\{7$\cdot$6$\cdot$5$\cdot$4$\cdot$3 $\cdot$ \textbackslash\{\}cancel\{2!\}\}\textbackslash\{\}right)
                        \textbackslash\{\}left(\textbackslash\{\}dfrac\{4$\cdot$3$\cdot$2$\cdot$1$\cdot$0 $\cdot$ \textbackslash\{\}cancel\{-1!\}\}\{\textbackslash\{\}cancel\{-1!\}\}\textbackslash\{\}right) = 
                        \textbackslash\{\}left($\frac{1}{[[factorial(STUDENTS)/factorial(NUM_B)]]}$\textbackslash\{\}right)
                        \textbackslash\{\}left([[factorial(STUDENTS-GROUP)/factorial(NUM\_B-GROUP)]]\textbackslash\{\}right) =
                        $\frac{[[factorial(NUM_B)*factorial(STUDENTS-GROUP)/GCD]]}{[[factorial(STUDENTS)*factorial(NUM_B-GROUP)/GCD]]}$
  \item One way to solve this problem is to figure out how many ways you can get exactly 5 tails, then
                        divide this by the total number of outcomes you could have gotten. Since every outcome has equal probability, this will be the 
                        probability that you will get exactly 
                        5 tails.
  \item How many outcomes are there where you get exactly 5 tails? Try thinking of each outcome as 
                        a 7-letter word, where the first letter is "H" if the first coin toss was heads and "T" 
                        if it was tails, and so on.
  \item So, the number of outcomes with exactly 5 tails 
                        is the same as the number of these words which have
                        5 T's and 2 H's.
  \item How many of these are there? If we treat all the letters as unique,
                            we'll find that there are 7! different arrangements, overcounting 5!
                            times for every time we only switch the T's around, and 2!
                            times for every time we only switch the H's around.
                            [Show me why]
                        
                        
                            Let's say we toss a coin 5 times, and get tails three times. How many different re-arrangements are there of the
                            letters "HHTTT"? Well, we have five choices for the first slot, four for the second slot, and so on, resulting in
                            5$\cdot$4$\cdot$3$\cdot$2$\cdot$1 = 5! = 120 \textbackslash\{\}; different re-arrangements. But, this treats all the letters
                            as unique, when 
                            HTHTT 
                            is the same as 
                            HTHTT,
                            and 
                            HTHTT,
                            
                            and so on. So really, we need to replace all these different re-arrangements
                            where we only move the tails around with one re-arrangement, HTHTT. There are 3! of these multi-
                            colored arrangements for every normal one, so that means dividing
                            our first guess of 5! by 3!. By the exact same logic, we need to divide by 2! to
                            avoid overcounting every permutation where we just move the heads around. So, the number of re-arrangements is 
                            $\frac{5!}{3!2!}$ = \textbackslash\{\}binom\{5\}\{3\}.
  \item So, there are $\frac{7!}{5!2!}$ = [[factorial(COINS)/(factorial(NUM)*factorial(COINS-NUM))]] different 
                            outcomes where you get exactly 5 tails.
                            [How many total outcomes are there?]
                        
                        
                            Well, if you only flip one coin, there are two outcomes, if you flip two there are four outcomes, if you flip three there
                            are eight. Each time you flip another coin, you double the number of possible outcomes.
  \item Altogether, there are 2\textasciicircum{}\{7\} = 32 total possible outcomes.
  \item So, the probability that you will get exactly 5 tails is
                        $\frac{[[factorial(COINS)/(factorial(NUM)*factorial(COINS-NUM))]]}{32}$ = $\frac{[[NUM_RIGHT/GCD]]}{[[NUM_ALL/GCD]]}$.
  \item So, the probability that you will get exactly 5 tails is
                        $\frac{[[factorial(COINS)/(factorial(NUM)*factorial(COINS-NUM))]]}{32}$.
\end{itemize}
\end{document}
