% Auto-converted from khan-exercises
\documentclass{article}
\usepackage{amsmath,amssymb}
\usepackage[T1]{fontenc}
\usepackage{textcomp}
\newcommand{\abs}[1]{\lvert #1\rvert}

\begin{document}
\section*{Dividing complex numbers}
\textbf{Question.} Divide the following complex numbers.

\textbf{Answer.} 5 + -5i

\textbf{Hints.}
\begin{itemize}
  \item We can divide complex numbers by multiplying both numerator and denominator by the denominator's complex conjugate, which is \textbackslash\{\}green\{4-1i\}.
                
                
                    \textbackslash\{\}qquad $\frac{25-15i}{4+1i}$ =
                        $\frac{25-15i}{4+1i}$ $\cdot$
                        \textbackslash\{\}dfrac\{\textbackslash\{\}green\{4-1i\}\}\{\textbackslash\{\}green\{4-1i\}\}
                    
                
                
                    
                        We can simplify the denominator using the fact (a + b) $\cdot$ (a - b) = a\textasciicircum{}2 - b\textasciicircum{}2.
                    
                    
                        \textbackslash\{\}qquad = \textbackslash\{\}dfrac\{(25-15i) $\cdot$ (4-1i)\}
                        \{[[negParens(B\_REAL)]]\textasciicircum{}2 - ([[coefficient(B\_IMAG)]]i)\textasciicircum{}2\}
                    
                
                
                    
                        Evaluate the squares in the denominator and subtract them.
                    
                    
                        \textbackslash\{\}qquad = \textbackslash\{\}dfrac\{(25-15i) $\cdot$ (4-1i)\}
                        \{(4)\textasciicircum{}2 - ([[coefficient(B\_IMAG)]]i)\textasciicircum{}2\}
                    
                    
                        \textbackslash\{\}qquad = \textbackslash\{\}dfrac\{(25-15i) $\cdot$ (4-1i)\}
                        \{16 + 1\}
                    
                    
                        \textbackslash\{\}qquad = \textbackslash\{\}dfrac\{(25-15i) $\cdot$ (4-1i)\}
                        \{17\}
                    
                    The denominator now doesn't contain any imaginary unit multiples, so it is a real number.
                    
                        Note that when a complex number, a + bi is multiplied by its conjugate,
                        the product is always a\textasciicircum{}2 + b\textasciicircum{}2.
                    
                
                
                    
                        Now, we can multiply out the two factors in the numerator.
                    
                    
                        \textbackslash\{\}qquad \textbackslash\{\}dfrac\{(\textbackslash\{\}blue\{25-15i\}) $\cdot$ (\textbackslash\{\}red\{4-1i\})\}
                        \{17\}
                    
                    
                        \textbackslash\{\}qquad = \textbackslash\{\}dfrac\{\textbackslash\{\}blue\{25\} $\cdot$ \textbackslash\{\}red\{[[negParens(B\_REAL)]]\} + \textbackslash\{\}blue\{-15\} $\cdot$ \textbackslash\{\}red\{[[negParens(B\_REAL)]] i\} + \textbackslash\{\}blue\{25\} $\cdot$ \textbackslash\{\}red\{-1 i\} + \textbackslash\{\}blue\{-15\} $\cdot$ \textbackslash\{\}red\{-1 i\textasciicircum{}2\}\}
                        \{17\}
                    
                    
                        \textbackslash\{\}qquad = \textbackslash\{\}dfrac\{100 + -60i + -25i + 15 i\textasciicircum{}2\}
                        \{17\}
                    
                
                
                    
                        Finally, simplify the fraction.
                    
                    
                        \textbackslash\{\}qquad \textbackslash\{\}dfrac\{100 + -60i + -25i - 15\}
                        \{17\} =
                        \textbackslash\{\}dfrac\{85 + -85i\}
                        \{17\} =
                        5-5i
\end{itemize}
\end{document}
