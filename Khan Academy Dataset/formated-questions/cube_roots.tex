% Auto-converted from khan-exercises
\documentclass{article}
\usepackage{amsmath,amssymb}
\usepackage[T1]{fontenc}
\usepackage{textcomp}
\newcommand{\abs}[1]{\lvert #1\rvert}

\begin{document}
\section*{Cube roots}
\textbf{Question.} \textbackslash\{\}Large\{\textbackslash\{\}sqrt[3]\{729\} = \textbackslash\{\}text\{?\}\}

\textbf{Answer.} 9

\textbf{Hints.}
\begin{itemize}
  \item If you can't think of that number, you can break down 729 into
                its prime factorization and look for equal groups of numbers.
  \item Let's draw a factor tree.
  \item So the prime factorization of 729 is 3$\times$ 3$\times$ 3$\times$ 3$\times$ 3$\times$ 3.
  \item \textbackslash\{\}sqrt[3]\{729\} is the number that, when
                        multiplied by itself three times, equals 729.
  \item We're looking for \textbackslash\{\}sqrt[3]\{729\}, so we want to split the prime factors into three identical groups.
  \item Notice that we can rearrange the factors like so:
                            729 = 3$\times$ 3$\times$ 3$\times$ 3$\times$ 3$\times$ 3 = \textbackslash\{\}left(3$\times$ 3\textbackslash\{\}right)$\times$\textbackslash\{\}left(3$\times$ 3\textbackslash\{\}right)$\times$\textbackslash\{\}left(3$\times$ 3\textbackslash\{\}right)
                        

                        
                            So \textbackslash\{\}left(3$\times$ 3\textbackslash\{\}right)\textasciicircum{}3 = 9\textasciicircum{}3 = 729.
                        
                            So 9\textasciicircum{}3 = 729.
  \item So \textbackslash\{\}sqrt[3]\{729\} is 9.
\end{itemize}
\end{document}
