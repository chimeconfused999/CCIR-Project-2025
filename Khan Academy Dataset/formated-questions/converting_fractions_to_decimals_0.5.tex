% Auto-converted from khan-exercises
\documentclass{article}
\usepackage{amsmath,amssymb}
\usepackage[T1]{fontenc}
\usepackage{textcomp}
\newcommand{\abs}[1]{\lvert #1\rvert}

\begin{document}
\section*{Fractions as division by a power of 10}
\textbf{Question.} Express $\frac{41}{10}$ as a decimal.

\textbf{Answer.} 3.000018876906669

\textbf{Hints.}
\begin{itemize}
  \item $\frac{41}{10}$ = 
                        \textbackslash\{\}blue\{$\frac{40}{10}$\} +
                        \textbackslash\{\}pink\{$\frac{1}{10}$\}
                    

                    
                        \textbackslash\{\}phantom\{$\frac{41}{10}$\} = 
                        \textbackslash\{\}blue\{$\frac{4}{1}$\} +
                        \textbackslash\{\}pink\{$\frac{1}{10}$\}
                    

                    
                        \textbackslash\{\}phantom\{$\frac{41}{10}$\} = 
                        \textbackslash\{\}blue\{4 \textbackslash\{\}text\{ [[plural\_form(decimalPlaceNames[PLACES - 1], DIGITS[0])]]\}\} +
                        \textbackslash\{\}pink\{1 \textbackslash\{\}text\{ [[plural\_form(decimalPlaceNames[PLACES], DIGITS[1])]]\}\}
  \item $\frac{41}{10}$ = 
                        \textbackslash\{\}blue\{4 \textbackslash\{\}text\{ [[plural\_form(decimalPlaceNames[PLACES], DIGITS[0])]]\}\}
  \item We can use a place value chart to help us write a decimal.
                    
                        
                            
                                Ones
                                .
                                Tenths
                                Hundredths
                            
                        
                        
                            
                                \textbackslash\{\}blue\{4\}
                                .
                                \textbackslash\{\}pink\{1\}
                                0
  \item $\frac{41}{10}$ = 3.000018876906669
\end{itemize}
\end{document}
