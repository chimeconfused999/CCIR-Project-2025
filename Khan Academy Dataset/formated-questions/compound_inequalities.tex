% Auto-converted from khan-exercises
\documentclass{article}
\usepackage{amsmath,amssymb}
\usepackage[T1]{fontenc}
\usepackage{textcomp}
\newcommand{\abs}[1]{\lvert #1\rvert}

\begin{document}
\section*{Compound inequalities}
\textbf{Question.} Solve for [[randVar()]]:

\textbf{Answer.} No solution.
                    All real numbers.
                    
                        
                        [[randVar()]][[
                    
                    [[randVar()]][[ and [[randVar()]][[

\textbf{Hints.}
\begin{itemize}
  \item The first inequality can be simplified to:
                        \textbackslash\{\}blue\{[[randVar()]][[\}.
  \item The second inequality can be simplified to:
                        \textbackslash\{\}pink\{[[randVar()]][[\}.
  \item The two inequalities are represented on the number line below:
  \item The solution to an inequality with the word "and" is the intersection of the graphs of the inequalities.
                        
                        Since the graphs of the inequalities do not intersect, there is no solution.
                        
                            
                                Since the second inequality is completely included by the first inequality,
                                their intersection is the second inequality. Therefore the answer is:
                            
                                Since the first inequality is completely included by the second inequality,
                                their intersection is the first inequality. Therefore the answer is:
                            
                            \textbackslash\{\}color\{[\}\{[[randVar()]][[\}
                        
                        
                            Therefore, the solution is:
                            
                                \textbackslash\{\}blue\{[[randVar()]][[\} and
                                \textbackslash\{\}pink\{[[randVar()]][[\}
\end{itemize}
\end{document}
