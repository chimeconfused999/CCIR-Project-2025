% Auto-converted from khan-exercises
\documentclass{article}
\usepackage{amsmath,amssymb}
\usepackage[T1]{fontenc}
\usepackage{textcomp}
\newcommand{\abs}[1]{\lvert #1\rvert}

\begin{document}
\section*{Angles 1}
\textbf{Question.} Given the following angles:
                        
                            
                                \textbackslash\{\}overline\{AB\} \textbackslash\{\}perp \textbackslash\{\}overline\{CD\},
                                line segments AB and CD are perpendicular.
                            
                            
                                
                                
                                    \textbackslash\{\}green\{\textbackslash\{\}angle\{AGF\}\} = 62\textasciicircum{}\textbackslash\{\}circ
                                
                                
                                    \textbackslash\{\}green\{\textbackslash\{\}angle\{DGF\}\} = 28\textasciicircum{}\textbackslash\{\}circ
                                
                            
                        
                        
                            What is
                            
                            \textbackslash\{\}blue\{\textbackslash\{\}angle\{CGE\}\} = \{?\}
                            \textbackslash\{\}blue\{\textbackslash\{\}angle\{BGE\}\} = \{?\}

\textbf{Answer.} 

\textbf{Hints.}
\begin{itemize}

\end{itemize}
\end{document}
