% Auto-converted from khan-exercises
\documentclass{article}
\usepackage{amsmath,amssymb}
\usepackage[T1]{fontenc}
\usepackage{textcomp}
\newcommand{\abs}[1]{\lvert #1\rvert}

\begin{document}
\section*{Reading bar charts 3}
\textbf{Question.} What was the average student score for the midterm?

\textbf{Answer.} [[COLUMN === 0 ? mean( MIDTERM ) : mean( FINAL )]]

\textbf{Hints.}
\begin{itemize}
  \item Use the key to figure out which of the two bars shows the scores for the midterm.
  \item Find each of the midterm scores represented by the blueorange bars.
  \item To find the average student score, add up the individual scores and divide by the number of students.
  \item $\frac{[[plus.apply( null, COLUMN === 0 ? MIDTERM : FINAL )]]}{5}$ =
                            $\frac{[[sum( COLUMN === 0 ? MIDTERM :  FINAL )]]}{5}$ =
                            [[COLUMN === 0 ? mean( MIDTERM ) : mean( FINAL )]]
  \item The average student score on the midterm was [[COLUMN === 0 ? mean( MIDTERM ) : mean( FINAL )]].
  \item Find each of the midterm scores represented by the blue bars.
  \item To find the average student score on the midterm, add up the individual scores and divide by the number of students.
  \item $\frac{[[plus.apply( null, MIDTERM )]]}{5}$ =
                            $\frac{[[sum( MIDTERM )]]}{5}$ =
                            [[mean( MIDTERM )]]
                        
                        The average student score on the midterm was [[mean( MIDTERM )]].
  \item Find each of the final exam scores represented by the orange bars.
  \item To find the average student score on the final exam, add up the individual scores and divide by the number of students.
  \item $\frac{[[plus.apply( null, FINAL )]]}{5}$ =
                            $\frac{[[sum( FINAL )]]}{5}$ =
                            [[mean( FINAL )]]
                        
                        The average student score on the final exam was [[mean( FINAL )]].
  \item The average student score was higher on the midterm
                        than on the final exam.
  \item The average student score was higher on the final exam
                        than the midterm.
  \item The average student score was the same on both the midterm
                        and the final exam.
  \item Use the key to figure out which of the two bars shows the scores for the midterm.
  \item Find each of the midterm scores represented by the blueorange bars.
                        \textbackslash\{\}qquad\textbackslash\{\}large\{[[( COLUMN === 0 ? MIDTERM : FINAL ).join(", ")]]\}
  \item Put the midterm scores in order from least to greatest.
                        \textbackslash\{\}qquad\textbackslash\{\}large\{[, [, $, ., m, a, p, (,  , n, e, w,  , A, r, r, a, y, (,  , N, U, M, _, S, T, U, D, E, N, T, S,  , ), ,,  , f, u, n, c, t, i, o, n, (, ),  , {,  , r, e, t, u, r, n,  , r, a, n, d, R, a, n, g, e, (,  , 0,  , 1, 2, ,,  , ),  , *,  , 5, ;,  , },  , 2, ), ], ]}$
  \item Since there are an odd number of scores, the median score is just the middle score.
  \item Since there are an even number of scores, the median score is the average of the two middle scores.
                        
                            
                                $\frac{
                                [[COLUMN === 0 ? sortNumbers( MIDTERM )[ NUM_STUDENTS / 2 - 1 ] : sortNumbers( FINAL )[ NUM_STUDENTS / 2 - 1 ]]] +
                                [[COLUMN === 0 ? sortNumbers( MIDTERM )[ NUM_STUDENTS / 2 ] : sortNumbers( FINAL )[ NUM_STUDENTS / 2 ]]]
                                }{2}$ = [[COLUMN === 0 ? median( MIDTERM ) : median( FINAL )]]
  \item The median score on the midterm was [[COLUMN === 0 ? median( MIDTERM ) : median( FINAL )]].
  \item Use the key to figure out which of the two bars shows the scores for the midterm.
  \item Find each of the midterm scores represented by the blueorange bars.
  \item The mode is the most frequently occurring score.
  \item [[COLUMN === 0 ? mode( MIDTERM ) : mode( FINAL )]] is the most frequently occurring score.
  \item The range is the difference between the largest value and the smallest value.
  \item Use the key to figure out which of the two bars shows the scores for the midterm.
  \item Find the largest and smallest midterm scores represented by the blueorange bars.
  \item The smallest midterm score is [[COLUMN === 0 ? Math.min.apply( null, MIDTERM ) : Math.min.apply( null, FINAL )]]. The largest midterm score is [[COLUMN === 0 ? Math.max.apply( null, MIDTERM ) : Math.max.apply( null, FINAL )]].
  \item Find the range by subtracting the smallest score from the largest score.
                        \textbackslash\{\}qquad [[COLUMN === 0 ? Math.max.apply( null, MIDTERM ) : Math.max.apply( null, FINAL )]] - [[COLUMN === 0 ? Math.min.apply( null, MIDTERM ) : Math.min.apply( null, FINAL )]] = [[MAX - MIN]]
  \item The range of midterm scores is [[MAX - MIN]].
  \item The midrange is halfway between the largest value and the smallest value.
  \item Use the key to figure out which of the two bars shows the scores for the midterm.
  \item Find the largest and smallest midterm scores represented by the blueorange bars.
  \item The smallest midterm score is [[COLUMN === 0 ? Math.min.apply( null, MIDTERM ) : Math.min.apply( null, FINAL )]]. The largest midterm score is [[COLUMN === 0 ? Math.max.apply( null, MIDTERM ) : Math.max.apply( null, FINAL )]].
  \item Find the midrange by averaging the smallest and largest scores.
                        \textbackslash\{\}qquad $\frac{[[COLUMN === 0 ? Math.min.apply( null, MIDTERM ) : Math.min.apply( null, FINAL )]] + [[COLUMN === 0 ? Math.max.apply( null, MIDTERM ) : Math.max.apply( null, FINAL )]]}{2}$ = [[(MAX + MIN) / 2]]
  \item The midrange of midterm scores is [[(MAX + MIN) / 2]].
\end{itemize}
\end{document}
