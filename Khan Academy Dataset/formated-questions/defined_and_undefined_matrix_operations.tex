% Auto-converted from khan-exercises
\documentclass{article}
\usepackage{amsmath,amssymb}
\usepackage[T1]{fontenc}
\usepackage{textcomp}
\newcommand{\abs}[1]{\lvert #1\rvert}

\begin{document}
\section*{Defined and undefined matrix operations}
\textbf{Question.} Is [[PRETTY\_MAT\_1\_ID + OPERATION + PRETTY\_MAT\_2\_ID]] defined?

\textbf{Answer.} [[ANSWER]]

\textbf{Hints.}
\begin{itemize}
  \item In order for addition of two matrices to be
                            defined, the matrices must have the same
                            dimensions.
                        
                        
                            If [[PRETTY\_MAT\_1\_ID]] is of
                            dimension (\textbackslash\{\}blue m $\times$ \textbackslash\{\}red n) and
                            [[PRETTY\_MAT\_2\_ID]] is of
                            dimension (\textbackslash\{\}blue p $\times$ \textbackslash\{\}red q),
                            then for their sum to be defined:
                        
                    
                        
                            In order for subtraction of two matrices to be
                            defined, the matrices must have the same
                            dimensions.
                        
                        
                            If [[PRETTY\_MAT\_1\_ID]] is of
                            dimension (\textbackslash\{\}blue m $\times$ \textbackslash\{\}red n) and
                            [[PRETTY\_MAT\_2\_ID]] is of
                            dimension (\textbackslash\{\}blue p $\times$ \textbackslash\{\}red q),
                            then for their difference to be defined:
                        
                    
                    
                        1. \textbackslash\{\}blue m (number of rows in [[PRETTY\_MAT\_1\_ID]]) must equal \textbackslash\{\}blue p (number of rows in [[PRETTY\_MAT\_2\_ID]]) and
                    
                    
                        2. \textbackslash\{\}red n (number of columns in [[PRETTY\_MAT\_1\_ID]]) must equal \textbackslash\{\}red q (number of columns in [[PRETTY\_MAT\_2\_ID]])
  \item Do [[PRETTY\_MAT\_1\_ID]] and [[PRETTY\_MAT\_2\_ID]] have the same number of rows?
                    
                    [[DIM\_1 === DIM\_3 ? YES : NO]]
                    
                        [[YES]]
                        [[NO]]
  \item Do [[PRETTY\_MAT\_1\_ID]] and [[PRETTY\_MAT\_2\_ID]] have the same number of columns?
                    
                    [[DIM\_2 === DIM\_4 ? YES : NO]]
                    
                        [[YES]]
                        [[NO]]
  \item Since [[PRETTY\_MAT\_1\_ID]] has
                        the same dimensions
                        ([[DIM\_1 + "\textbackslash\{\}$\times$" + DIM\_2]]) as
                        [[PRETTY\_MAT\_2\_ID]] ([[DIM\_3 + "\textbackslash\{\}$\times$" + DIM\_4]]),
                        [[PRETTY\_MAT\_1\_ID + OPERATION + PRETTY\_MAT\_2\_ID]]
                        is defined.
                    
                        Since [[PRETTY\_MAT\_1\_ID]] has
                        different dimensions
                        ([[DIM\_1 + "\textbackslash\{\}$\times$" + DIM\_2]]) from
                        [[PRETTY\_MAT\_2\_ID]] ([[DIM\_3 + "\textbackslash\{\}$\times$" + DIM\_4]]),
                        [[PRETTY\_MAT\_1\_ID + OPERATION + PRETTY\_MAT\_2\_ID]]
                        is not defined.
  \item In order for multiplication of two matrices to be defined, the two inner dimensions must be equal.
  \item If the two matrices have dimensions (\textbackslash\{\}blue m $\times$ \textbackslash\{\}red n) and (\textbackslash\{\}red p $\times$ \textbackslash\{\}green q), then \textbackslash\{\}red n (number of columns in the first matrix) must equal \textbackslash\{\}red p (number of rows in the second matrix) for their product to be defined.
  \item How many columns does the first matrix, [[PRETTY\_MAT\_1\_ID]], have?
                    
                    [[DIM\_2]]
  \item How many rows does the second matrix, [[PRETTY\_MAT\_2\_ID]], have?
                    
                    [[DIM\_3]]
  \item Since [[PRETTY\_MAT\_1\_ID]] has the same
                    number of columns ([[DIM\_2]]) as
                    [[PRETTY\_MAT\_2\_ID]] has rows
                    ([[DIM\_3]]),
                    [[PRETTY\_MAT\_1\_ID + PRETTY\_MAT\_2\_ID]]
                    is defined.
  \item Since [[PRETTY\_MAT\_1\_ID]] has a
                    different number of columns
                    ([[DIM\_2]]) than
                    [[PRETTY\_MAT\_2\_ID]] has rows
                    ([[DIM\_3]]),
                    [[PRETTY\_MAT\_1\_ID + PRETTY\_MAT\_2\_ID]]
                    is not defined.
\end{itemize}
\end{document}
