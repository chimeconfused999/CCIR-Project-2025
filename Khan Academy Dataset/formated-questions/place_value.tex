% Auto-converted from khan-exercises
\documentclass{article}
\usepackage{amsmath,amssymb}
\usepackage[T1]{fontenc}
\usepackage{textcomp}
\newcommand{\abs}[1]{\lvert #1\rvert}

\begin{document}
\section*{Place value}
\textbf{Question.} What is the place value of 7 in [[digitsToInteger( DIGITS )]]?

\textbf{Answer.} [[capitalize(SOLUTION)]]

\textbf{Hints.}
\begin{itemize}
  \item [[digitsToInteger( DIGITS )]] can be represented as follows.
                        = [[(function() \{
                            var maxPower = DIGITS.length - 1;
                            var products = $.map( DIGITS, function( digit, index ){
                                return "(" + digit + "\$\textbackslash\{\}times$" + pow(10, maxPower - index) + ")";
                            });
                            return products.join( "+" );
                        })()]]$
  \item =  [[(function() \{
                            var maxPower = DIGITS.length - 1;
                            var words = $.map( DIGITS, function( digit, index ) {
                                var value = powerToPlace( maxPower - index );
                                return "<code>" + digit + "</code> " + 
                                       KhanUtil.plural(value, digit);
                            });
                            return words.join(" <code>+</code> ");
                        })()]]$
  \item Thus, 7 is in the [ place.
  \item [[hint]]
  \item Add all these parts up:
  \item = [[digitsToInteger( DIGITS )]]
\end{itemize}
\end{document}
