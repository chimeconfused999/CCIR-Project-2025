% Auto-converted from khan-exercises
\documentclass{article}
\usepackage{amsmath,amssymb}
\usepackage[T1]{fontenc}
\usepackage{textcomp}
\newcommand{\abs}[1]{\lvert #1\rvert}

\begin{document}
\section*{Geometric sequences 2}
\textbf{Question.} The geometric sequence (a\_i) is defined by the formula:
                a\_i = 8, 3 \textbackslash\{\}left(-3, 4\textbackslash\{\}right)\textasciicircum{}\{i - 1\}
                What is a\_\{5\}, the [[ordinalThrough20(N)]] term in the sequence?

\textbf{Answer.} 0.84375

\textbf{Hints.}
\begin{itemize}
  \item From the given formula, we can see that the first term of the sequence is 8, 3 and the common ratio is -3, 4.
  \item The second term is simply the first term times the common ratio.
                    Therefore, the second term is equal to a\_2 = 8, 3 $\cdot$ -3, 4 = -2, 1.
  \item To find a\_\{5\}, we can simply substitute i = 5 into the given formula.
                    Therefore, the [[ordinalThrough20(N)]] term is equal to a\_\{5\} = 8, 3 \textbackslash\{\}left(-3, 4\textbackslash\{\}right)\textasciicircum{}\{5 - 1\} = 27, 32.
  \item From the given formula, we can see that the first term of the sequence is 8, 3 and the common ratio is -3, 4.
  \item The second term is simply the first term times the common ratio.
                    Therefore, the second term is equal to a\_2 = 8, 3 $\cdot$ -3, 4 = -2, 1.
  \item To find the [[ordinalThrough20(N)]] term, we can rewrite the given recurrence as an explicit formula.
                    The general form for a geometric sequence is a\_i = a\_1 r\textasciicircum{}\{i - 1\}. In this case, we have a\_i = 8, 3 \textbackslash\{\}left(-3, 4\textbackslash\{\}right)\textasciicircum{}\{i - 1\}.
                    To find a\_\{5\}, we can simply substitute i = 5 into the formula.
                    Therefore, the [[ordinalThrough20(N)]] term is equal to a\_\{5\} = 8, 3 \textbackslash\{\}left(-3, 4\textbackslash\{\}right)\textasciicircum{}\{5 - 1\} = 27, 32.
\end{itemize}
\end{document}
