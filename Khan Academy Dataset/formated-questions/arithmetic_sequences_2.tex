% Auto-converted from khan-exercises
\documentclass{article}
\usepackage{amsmath,amssymb}
\usepackage[T1]{fontenc}
\usepackage{textcomp}
\newcommand{\abs}[1]{\lvert #1\rvert}

\begin{document}
\section*{Arithmetic sequences 2}
\textbf{Question.} The arithmetic sequence (a\_i) is defined by the formula:
                a\_i = [[A]] + [[D]](i - 1)
                What is a\_\{[[N]]\}, the [[ordinalThrough20(N)]] term in the sequence?

\textbf{Answer.} [[A + D * (N - 1)]]

\textbf{Hints.}
\begin{itemize}
  \item From the given formula, we can see that the first term of the sequence is [[A]] and the common difference is [[D]].
  \item The second term is simply the first term plus the common difference.
                    Therefore, the second term is equal to a\_2 = [[A]] + [[D]] = [[A + D]].
  \item To find a\_\{[[N]]\}, we can simply substitute i = [[N]] into the given formula.
                    Therefore, the [[ordinalThrough20(N)]] term is equal to a\_\{[[N]]\} = [[A]] + [[D]] ([[N]] - 1) = [[A + D * (N - 1)]].
  \item From the given formula, we can see that the first term of the sequence is [[A]] and the common difference is [[D]].
  \item The second term is simply the first term plus the common difference.
                    Therefore, the second term is equal to a\_2 = a\_1 + [[D]] = [[A]] + [[D]] = [[A + D]].
  \item To find the [[ordinalThrough20(N)]] term, we can rewrite the given recurrence as an explicit formula.
                    The general form for an arithmetic sequence is a\_i = a\_1 + d(i - 1). In this case, we have a\_i = [[A]] + [[D]](i - 1).
                    To find a\_\{[[N]]\}, we can simply substitute i = [[N]] into the our formula.
                    Therefore, the [[ordinalThrough20(N)]] term is equal to a\_\{[[N]]\} = [[A]] + [[D]] ([[N]] - 1) = [[A + D * (N - 1)]].
\end{itemize}
\end{document}
