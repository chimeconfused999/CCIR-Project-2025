% Auto-converted from khan-exercises
\documentclass{article}
\usepackage{amsmath,amssymb}
\usepackage[T1]{fontenc}
\usepackage{textcomp}
\newcommand{\abs}[1]{\lvert #1\rvert}

\begin{document}
\section*{Multiplying and dividing complex numbers in polar form}
\textbf{Question.} Multiply the following complex numbers:

\textbf{Answer.} Radius: 1
                            
                                
                                
                            
                        
                        
                            Angle: 0
                            
                                
                                
                            
                        
                        1
                    
                    [
                        graph.currComplexPolar.getAngleNumerator(),
                        graph.currComplexPolar.getRadius()
                    ]
                    
                        var angle = guess[0];
                        var radius = guess[1];

                        if (angle === 0 \&\& radius === 1) \{
                            return "";
                        \}

                        return angle === ANSWER\_ANGLE\_NUMERATOR \&\&
                               radius === ANSWER\_RADIUS;
                    
                    
                        redrawComplexPolarForm(guess[0], guess[1]);
                    
                    
                        redrawComplexPolarForm(guess[0], guess[1]);

\textbf{Hints.}
\begin{itemize}
  \item Multiplying complex numbers in polar forms can be done by multiplying the lengths
                    and adding the angles.
  \item The first number, [["\textbackslash\{\}\textbackslash\{\}blue\{" + polarForm(A\_RADIUS, A\_ANGLE, USE\_EULER\_FORM) + "\}"]],
                    has angle [["\textbackslash\{\}\textbackslash\{\}blue\{" + piFraction(A\_ANGLE, true) + "\}"]]
                    and radius \textbackslash\{\}blue\{14\}.
  \item The second number, [["\textbackslash\{\}\textbackslash\{\}green\{" + polarForm(B\_RADIUS, B\_ANGLE, USE\_EULER\_FORM) + "\}"]],
                    has angle [["\textbackslash\{\}\textbackslash\{\}green\{" + piFraction(B\_ANGLE) + "\}"]]
                    and radius \textbackslash\{\}green\{2\}.
  \item The radius of the result will be
                    \textbackslash\{\}blue\{14\} $\cdot$ \textbackslash\{\}green\{2\} = \textbackslash\{\}orange\{7\}.
  \item The sum of the angles is [["\textbackslash\{\}\textbackslash\{\}blue\{" + piFraction(A\_ANGLE, true) + "\}"]] + [["\textbackslash\{\}\textbackslash\{\}green\{" + piFraction(B\_ANGLE) + "\}"]] = [["\textbackslash\{\}\textbackslash\{\}orange\{" + piFraction((A\_ANGLE\_NUMERATOR - B\_ANGLE\_NUMERATOR ) * PI * 2 / DENOMINATOR, true) + "\}"]].
                    
                    
                        The angle [["\textbackslash\{\}\textbackslash\{\}orange\{" + piFraction((A\_ANGLE\_NUMERATOR - B\_ANGLE\_NUMERATOR ) * PI * 2 / DENOMINATOR, true) + "\}"]] is more than 2 \textbackslash\{\}pi.
                        A complex number goes a full circle if its angle is increased by 2 \textbackslash\{\}pi, so it goes back to itself.
                        Because of that, angles of complex numbers are convenient to keep between 0 and 2 \textbackslash\{\}pi.
                    
                    
                        [["\textbackslash\{\}\textbackslash\{\}orange\{" + piFraction((A\_ANGLE\_NUMERATOR - B\_ANGLE\_NUMERATOR ) * PI * 2 / DENOMINATOR, true) + "\}"]] - 2 \textbackslash\{\}pi = [["\textbackslash\{\}\textbackslash\{\}orange\{" + piFraction(ANSWER\_ANGLE, true) + "\}"]]
  \item The angle of the result is [["\textbackslash\{\}\textbackslash\{\}blue\{" + piFraction(A\_ANGLE, true) + "\}"]] + [["\textbackslash\{\}\textbackslash\{\}green\{" + piFraction(B\_ANGLE) + "\}"]] = [["\textbackslash\{\}\textbackslash\{\}orange\{" + piFraction(ANSWER\_ANGLE, true) + "\}"]].
  \item The radius of the result is \textbackslash\{\}orange\{7\}
                    and the angle of the result is [["\textbackslash\{\}\textbackslash\{\}orange\{" + piFraction(ANSWER\_ANGLE, true) + "\}"]].
  \item Dividing complex numbers in polar forms can be done by dividing the radii
                    and subtracting the angles.
  \item The dividend, [["\textbackslash\{\}\textbackslash\{\}blue\{" + polarForm(A\_RADIUS, A\_ANGLE, USE\_EULER\_FORM) + "\}"]],
                    has angle [["\textbackslash\{\}\textbackslash\{\}blue\{" + piFraction(A\_ANGLE, true) + "\}"]]
                    and radius \textbackslash\{\}blue\{14\}.
  \item The divisor, [["\textbackslash\{\}\textbackslash\{\}green\{" + polarForm(B\_RADIUS, B\_ANGLE, USE\_EULER\_FORM) + "\}"]],
                    has angle [["\textbackslash\{\}\textbackslash\{\}green\{" + piFraction(B\_ANGLE) + "\}"]]
                    and radius \textbackslash\{\}green\{2\}.
  \item The radius of the result will be
                    \textbackslash\{\}dfrac\{\textbackslash\{\}blue\{14\}\}\{\textbackslash\{\}green\{2\}\} = \textbackslash\{\}orange\{7\}.
  \item The difference of the angles is
                        [["\textbackslash\{\}\textbackslash\{\}blue\{" + piFraction(A\_ANGLE, true) + "\}"]] - [["\textbackslash\{\}\textbackslash\{\}green\{" + piFraction(B\_ANGLE) + "\}"]] = [["\textbackslash\{\}\textbackslash\{\}orange\{" + piFraction((A\_ANGLE\_NUMERATOR - B\_ANGLE\_NUMERATOR ) * PI * 2 / DENOMINATOR, true) + "\}"]].
                    
                    
                        The angle [["\textbackslash\{\}\textbackslash\{\}orange\{" + piFraction((A\_ANGLE\_NUMERATOR - B\_ANGLE\_NUMERATOR ) * PI * 2 / DENOMINATOR, true) + "\}"]] is negative.
                        A complex number goes a full circle if its angle is increased by 2 \textbackslash\{\}pi, so it goes back to itself.
                        Because of that, angles of complex numbers are convenient to keep between 0 and 2 \textbackslash\{\}pi.
                    
                    
                        [["\textbackslash\{\}\textbackslash\{\}orange\{" + piFraction((A\_ANGLE\_NUMERATOR - B\_ANGLE\_NUMERATOR ) * PI * 2 / DENOMINATOR, true) + "\}"]] + 2 \textbackslash\{\}pi = [["\textbackslash\{\}\textbackslash\{\}orange\{" + piFraction(ANSWER\_ANGLE, true) + "\}"]]
  \item The angle of the result is [["\textbackslash\{\}\textbackslash\{\}blue\{" + piFraction(A\_ANGLE, true) + "\}"]] - [["\textbackslash\{\}\textbackslash\{\}green\{" + piFraction(B\_ANGLE) + "\}"]] = [["\textbackslash\{\}\textbackslash\{\}orange\{" + piFraction(ANSWER\_ANGLE, true) + "\}"]].
  \item The radius of the result is \textbackslash\{\}orange\{7\}
                    and the angle of the result is [["\textbackslash\{\}\textbackslash\{\}orange\{" + piFraction(ANSWER\_ANGLE, true) + "\}"]].
\end{itemize}
\end{document}
