% Auto-converted from khan-exercises
\documentclass{article}
\usepackage{amsmath,amssymb}
\usepackage[T1]{fontenc}
\usepackage{textcomp}
\newcommand{\abs}[1]{\lvert #1\rvert}

\begin{document}
\section*{Line graph intuition}
\textbf{Question.} Graph a line that has [[SLOPE\_NAME]] slope:

\textbf{Answer.} [POINTS.pointA.coord, POINTS.pointB.coord]
                
                
                    if (SLOPE !== 2\&\&
                            guess[0][0] === -5 \&\&
                            guess[0][1] === 5 \&\&
                            guess[1][0] === 5 \&\&
                            guess[1][1] === 5) \{
                        return "";
                    \}
                    var slope = (guess[1][1] - guess[0][1]) /
                        (guess[1][0] - guess[0][0]);
                    if (SLOPE === 0) \{
                        return slope > 0;
                    \} else if (SLOPE === 1) \{
                        return slope < 0;
                    \} else if (SLOPE === 2) \{
                        return abs(slope) < 0.001;
                    \} else if (SLOPE === 3) \{
                        return guess[1][0] === guess[0][0];
                    \}
                
                
                    POINTS.pointA.setCoord(guess[0]);
                    POINTS.pointB.setCoord(guess[1]);
                    graph.line1.transform(true);

\textbf{Hints.}
\begin{itemize}
  \item The slope of a line is the ratio of the change in
                    y over the change in x. Also
                    known as rise over run.
  \item For a positive slope, if the change in y from
                    one point to another is positive then the change in
                    x should also be positive.
  \item For a negative slope, if the change in y from
                    one point to another is positive then the change in
                    x should be negative.
  \item For a zero slope, the change in y should be
                    zero from any point to another point.
  \item For an undefined slope, the change in x should
                    be zero from any point to another point.
  \item A line with [[SLOPE\_NAME]] slope looks like:
  \item The slope of the original line is negative.
                    
                        The slope of the original line is positive.
  \item If the slope of the new line is negative and
                            greater than the slope of the original line with
                            negative slope then the new line should be less
                            steep.
                        
                    
                    
                        
                            If the slope of the new line is negative and less
                            than the slope of the original line with negative
                            slope, then the new line should be steeper.
  \item If the slope of the new line is positive and greater
                            than the slope of the original line with positive
                            slope, then the new line should be steeper.
                        
                    
                    
                        
                            If the slope of the new line is positive and less
                            than the slope of the original line with positive
                            slope, then the new line should be less steep.
  \item Since the original slope is 
                        [[fractionReduce(SLOPE\_N, SLOPE\_D)]],
                        the slope of the new line could be 
                        [[decimalFraction(NEW\_SLOPE, true, true)]]
                         and could look like this:
\end{itemize}
\end{document}
