% Auto-converted from khan-exercises
\documentclass{article}
\usepackage{amsmath,amssymb}
\usepackage[T1]{fontenc}
\usepackage{textcomp}
\newcommand{\abs}[1]{\lvert #1\rvert}

\begin{document}
\section*{Telling time 3}
\textbf{Question.} What time is it?

\textbf{Answer.} The time is:
                        6 :
                        14 [[icu.getDateFormatSymbols().am\_pm[HOUR >= 7 ? 0 : 1]]]

\textbf{Hints.}
\begin{itemize}
  \item The small hand is for the hour, and the big hand is for the minutes.
  \item The hour hand is between 6 and 7,
                            so the hour is 6.
                        
                        
                            The hour hand is close to but hasn't passed 7,
                            so the hour is still 6.
  \item Each large tick mark represents $\frac{60}{12}$ = 5 minutes.
                        Each small tick mark represents 1 minute.
  \item The minute hand is 4 small ticks past the
                                2,
                                which represents 2 $\times$ 5 + 4 = 14 minutes.
  \item The time is 6:14.
\end{itemize}
\end{document}
