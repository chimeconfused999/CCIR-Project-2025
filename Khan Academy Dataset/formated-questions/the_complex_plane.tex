% Auto-converted from khan-exercises
\documentclass{article}
\usepackage{amsmath,amssymb}
\usepackage[T1]{fontenc}
\usepackage{textcomp}
\newcommand{\abs}[1]{\lvert #1\rvert}

\begin{document}
\section*{The complex plane}
\textbf{Question.} Move the orange dot to -1-1i.

\textbf{Answer.} graph.movablePoint.coord
                
                
                    return graph.movablePoint.coord.join() === [REAL, IMAG].join();
                
                
                    graph.movablePoint.setCoord(guess);

\textbf{Hints.}
\begin{itemize}
  \item Complex numbers can be visualized as points on a plane. The coordinates on the
                    real and imaginary axes correspond to the real and imaginary parts of the complex number.
  \item -1-1i has real part -1 and imaginary part -1.
  \item The vertical orange line represents all complex numbers with real part -1 (including -1-1i).
  \item The horizontal blue line represents all complex numbers with imaginary part -1, also including -1-1i.
  \item The only complex number with real part -1 and imaginary part -1 is -1-1i,
                        so it lies on the intersection of the vertical orange line and the horizontal blue line.
\end{itemize}
\end{document}
