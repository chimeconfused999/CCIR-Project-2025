% Auto-converted from khan-exercises
\documentclass{article}
\usepackage{amsmath,amssymb}
\usepackage[T1]{fontenc}
\usepackage{textcomp}
\newcommand{\abs}[1]{\lvert #1\rvert}

\begin{document}
\section*{Conditional statements}
\textbf{Question.} [[makeDisplay(SOL)]]
            
        
        [[SORTER.init("sortable")]]

\textbf{Answer.} Drag the phrases left and right so they form the correct "if-then" sentence
        
        SORTER.getContent()
        
            return guess.join(" ") === \_.map(SOLUTION, makeSoln).join(" ");
        
        
            SORTER.setContent(guess);

\textbf{Hints.}
\begin{itemize}
  \item In an "if-then" sentence there is a given part, and a conclusion part. The sentence goes "If given part, then conclusion part."
  \item The conclusion is always true when the given is true, but not necessarily the other way around.
  \item The given part stands by itself. Sometimes it describes an object, and sometimes it is an object itself.
  \item In this sentence, the given part is "If,I,1,,1,,1,then,I,2,,2,,2 boys,young,can't drive,true", which describes what kind of boys, young, can't drive, true we are dealing with.
  \item The conclusion part is something that depends on the given part, and is always true when the given part is true.
  \item In this sentence, the conclusion part is "When , I ,1, ,1,  , ,1, , I will ,2, ,2,  , ,2, .", which says that boys, young, can't drive, true When , I ,1, ,1,  , ,1, , I will ,2, ,2,  , ,2, . when the given part is true.
  \item To find the answer, we combine the given and conclusion part, to find "[[\_.map(SOLUTION, makeStatement).join(" ")]]".
  \item In an "if-then" sentence there is a given part, and a conclusion part. The sentence goes "If given part, then conclusion part."
  \item The conclusion is always true when the given is true, but not necessarily the other way around.
  \item The given part stands by itself. Sometimes it describes an object, and sometimes it is an object itself.
  \item In this sentence, the given part is "[[TYPE]]", which describes what kind of [[CATEGORY]] we are dealing with.
  \item The conclusion part is something that depends on the given part, and is always true when the given part is true.
  \item In this sentence, the conclusion part is "undefined undefined", which says that [[AMBIGUOUS\_PLURAL(TYPE)]] undefined undefined when the given part is true.
  \item To find the answer, we combine the given and conclusion part, to find "[[\_.map(SOLUTION, makeStatement).join(" ")]]".
  \item In an "if-then" sentence there is a given part, and a conclusion part. The sentence goes "If given part, then conclusion part."
  \item The conclusion is always true when the given is true, but not necessarily the other way around.
  \item The given part stands by itself. Sometimes it describes an object, and sometimes it is an event that might happen.
  \item In this sentence, the given part is "undefined undefined", which is an event that might happen.
  \item The conclusion part is something that will happen when the given part happens.
  \item In this sentence, the conclusion part is "undefined undefined", which is another event that might happen when I undefined undefined.
  \item To find the answer, we combine the given and conclusion part, to find "[[\_.map(SOLUTION, makeStatement).join(" ")]]".
\end{itemize}
\end{document}
