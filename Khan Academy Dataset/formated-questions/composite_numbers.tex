% Auto-converted from khan-exercises
\documentclass{article}
\usepackage{amsmath,amssymb}
\usepackage[T1]{fontenc}
\usepackage{textcomp}
\newcommand{\abs}[1]{\lvert #1\rvert}

\begin{document}
\section*{Composite numbers}
\textbf{Question.} Which of these numbers is composite?

\textbf{Answer.} [[rand( 5 ) === 0 ? getEvenComposite() : getOddComposite()]]

\textbf{Hints.}
\begin{itemize}
  \item A composite number is a number that has more than two factors (including 1 and itself).
  \item [[toSentence( \_.filter( CHOICES, function( p ) \{ return p !== COMPOSITE; \} ) )]] each have only two factors.
  \item The factors of [[rand( 5 ) === 0 ? getEvenComposite() : getOddComposite()]] are [[toSentence( getFactors( COMPOSITE ) )]].
  \item Thus, [[rand( 5 ) === 0 ? getEvenComposite() : getOddComposite()]] is the composite number.
\end{itemize}
\end{document}
