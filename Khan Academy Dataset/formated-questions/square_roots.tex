% Auto-converted from khan-exercises
\documentclass{article}
\usepackage{amsmath,amssymb}
\usepackage[T1]{fontenc}
\usepackage{textcomp}
\newcommand{\abs}[1]{\lvert #1\rvert}

\begin{document}
\section*{Square roots of perfect squares}
\textbf{Question.} \textbackslash\{\}Large\{$\sqrt{121}$ = \textbackslash\{\}text\{?\}\}

\textbf{Answer.} 11

\textbf{Hints.}
\begin{itemize}
  \item If you can't think of that number, you can break down 121 into
                its prime factorization and look for equal groups of numbers.
  \item Let's draw a factor tree.
  \item So the prime factorization of 121 is 11$\times$ 11.
  \item $\sqrt{121}$ is the number that, when
                        multiplied by itself, equals 121.
  \item We're looking for $\sqrt{121}$, so we want to split the prime factors into two identical groups.
  \item We only have two prime factors, and we want to split them into two groups, so this is easy.
                        121 = 11$\times$ 11, so 11\textasciicircum{}2 = 121.
  \item Notice that we can rearrange the factors like so:
                            121 = 11 $\times$ 11 = \textbackslash\{\}left(11\textbackslash\{\}right) $\times$ \textbackslash\{\}left(11\textbackslash\{\}right)
                        

                        
                            So 11\textasciicircum{}2 = 121.
  \item So $\sqrt{121}$ is 11.
\end{itemize}
\end{document}
