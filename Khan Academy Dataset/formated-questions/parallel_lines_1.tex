% Auto-converted from khan-exercises
\documentclass{article}
\usepackage{amsmath,amssymb}
\usepackage[T1]{fontenc}
\usepackage{textcomp}
\newcommand{\abs}[1]{\lvert #1\rvert}

\begin{document}
\section*{Parallel lines 1}
\textbf{Question.} If we know that the blue angle is [[MEASURE]]\textasciicircum{}\textbackslash\{\}circ,
        what is the measure of angle x?

\textbf{Answer.} [[SOLUTION]] \textasciicircum{}\textbackslash\{\}circ

\textbf{Hints.}
\begin{itemize}
  \item Notice that the two angles are in the same position but at different intersections.
  \item One way to describe the angles is to say that they are corresponding angles.
  \item Corresponding angles are always equal.
  \item Vertical angles are equal, so the pink angle measures [[MEASURE]] degrees.
  \item The pink and green angles are corresponding angles, so they are also equal.
  \item Vertical angles are equal, so the pink angle measures [[MEASURE]] degrees.
  \item The pink and green angles are corresponding angles, so they are also equal.
  \item The pink angles are adjacent to the blue angle and form a straight line,
                        so they measure 180\textasciicircum{}\textbackslash\{\}circ - [[MEASURE]]\textasciicircum{}\textbackslash\{\}circ = [[180 - MEASURE]]\textasciicircum{}\textbackslash\{\}circ.
  \item The pink angles are equal because they are opposite each other.
  \item One of the pink angles corresponds with the green angle, and the other pink angle forms an alternate interior angle.
  \item Angle x equals the pink angles and measures [[SOLUTION]]\textasciicircum{}\textbackslash\{\}circ.
                    
                    Note that the blue and green angles are supplementary.
  \item The pink angles are adjacent to the blue angle and form a straight line,
                        so they measure 180\textasciicircum{}\textbackslash\{\}circ - [[MEASURE]]\textasciicircum{}\textbackslash\{\}circ = [[180 - MEASURE]]\textasciicircum{}\textbackslash\{\}circ.
  \item The pink angles are equal because they are vertical angles.
  \item One of the pink angles corresponds with the green angle, and the other pink angle forms an alternate exterior angle.
  \item Angle x equals the pink angles and measures [[SOLUTION]]\textasciicircum{}\textbackslash\{\}circ.
                    
                    Note that the blue and green angles are supplementary.
  \item Angle x is [[SOLUTION]]\textasciicircum{}\textbackslash\{\}circ
\end{itemize}
\end{document}
