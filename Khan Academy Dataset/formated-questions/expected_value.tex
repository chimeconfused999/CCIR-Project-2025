% Auto-converted from khan-exercises
\documentclass{article}
\usepackage{amsmath,amssymb}
\usepackage[T1]{fontenc}
\usepackage{textcomp}
\newcommand{\abs}[1]{\lvert #1\rvert}

\begin{document}
\section*{Expected value}
\textbf{Question.} How much money do you expect to make (or lose) per game?

\textbf{Answer.} $
                12$

\textbf{Hints.}
\begin{itemize}
  \item The expected value is a weighted average of the possible values with
                    the weights determined by the probability of observing that value.
  \item There are two events that can happen in this game: either
                    you roll an even number, 5, or you don't. So, the
                    expected value will look like this:
                    
                    E = 
                    (money gained when you roll an even number, 5)
                    $\cdot$
                    (probability of rolling an even number, 5)
                    +
                    (money gained when you don't roll an even number, 5)
                    $\cdot$
                    (probability of not rolling an even number, 5).
  \item The money you gain when you win is $10.
                    The probability of winning is the probability
                    that you roll an even number, 5.$
  \item This probability is the number of winning outcomes
                    divided by the total number of
                    outcomes, NaN.
  \item The money you gain when you lose is -$[[MAKE_COUNT * MAKE - LOSE_COUNT * LOSE]]
                    (since you actually lose money). The probability that
                    you lose is the probability that you don't roll
                    an even number, 5.$
  \item This probability must be
                    1 - NaN = 8.
  \item So, if we take the average of the amount of money you make
                    on each outcome, weighted by how probable each outcome is,
                    we get the expected amount of money you will make:
                    (10\textbackslash\{\}cdotNaN) +
                        (-[[MAKE\_COUNT * MAKE - LOSE\_COUNT * LOSE]]$\cdot$8) =
                        Yes, the expected value is positive. = 
                        $12.00.$
  \item The expected value is a weighted average of the possible values with
                    the weights determined by the probability of observing that value.
  \item In this case, there are [[(function()\{
                        if(SIDES < 7) \{
                            return \_.map(\_.range(SIDES), function(i)\{
                                    return "\textbackslash\{\}$\frac{"+(i+1)+"}{"+SIDES+"}$"; \})
                                    .join("+");
                        \}

                        first = \_.map(\_.range(3), function(i)\{
                                return "\textbackslash\{\}$\frac{"+(i+1)+"}{"+SIDES+"}$"; \})
                                .join("+");
                        last = \_.map(\_.range(3), function(i)\{
                                return "\textbackslash\{\}$\frac{"+(SIDES-2+i)+"}{"+SIDES+"}$"; \}).join("+");
                        return [first,"\textbackslash\{\}\textbackslash\{\}cdots",last].join("+");
                    \})()]] outcomes:
                    the first outcome is rolling a 1, the second outcome is
                    rolling a 2, and so on. The value of each of these outcomes
                    is just the number you roll.
  \item So, the value of the first outcome is 1, and its
                    probability is $\frac{1}{[[(function(){
                        if(SIDES < 7) {
                            return _.map(_.range(SIDES), function(i){
                                    return "\\dfrac{"+(i+1)+"}$\{"+SIDES+"\}"; \})
                                    .join("+");
                        \}

                        first = \_.map(\_.range(3), function(i)\{
                                return "\textbackslash\{\}$\frac{"+(i+1)+"}{"+SIDES+"}$"; \})
                                .join("+");
                        last = \_.map(\_.range(3), function(i)\{
                                return "\textbackslash\{\}$\frac{"+(SIDES-2+i)+"}{"+SIDES+"}$"; \}).join("+");
                        return [first,"\textbackslash\{\}\textbackslash\{\}cdots",last].join("+");
                    \})()]]\}.
  \item The value of the second outcome is 2, the value of
                    the third outcome is 3, and so on. There are
                    [[(function()\{
                        if(SIDES < 7) \{
                            return \_.map(\_.range(SIDES), function(i)\{
                                    return "\textbackslash\{\}$\frac{"+(i+1)+"}{"+SIDES+"}$"; \})
                                    .join("+");
                        \}

                        first = \_.map(\_.range(3), function(i)\{
                                return "\textbackslash\{\}$\frac{"+(i+1)+"}{"+SIDES+"}$"; \})
                                .join("+");
                        last = \_.map(\_.range(3), function(i)\{
                                return "\textbackslash\{\}$\frac{"+(SIDES-2+i)+"}{"+SIDES+"}$"; \}).join("+");
                        return [first,"\textbackslash\{\}\textbackslash\{\}cdots",last].join("+");
                    \})()]] outcomes altogether, and each of them
                    occurs with probability
                    $\frac{1}{[[(function(){
                        if(SIDES < 7) {
                            return _.map(_.range(SIDES), function(i){
                                    return "\\dfrac{"+(i+1)+"}$\{"+SIDES+"\}"; \})
                                    .join("+");
                        \}

                        first = \_.map(\_.range(3), function(i)\{
                                return "\textbackslash\{\}$\frac{"+(i+1)+"}{"+SIDES+"}$"; \})
                                .join("+");
                        last = \_.map(\_.range(3), function(i)\{
                                return "\textbackslash\{\}$\frac{"+(SIDES-2+i)+"}{"+SIDES+"}$"; \}).join("+");
                        return [first,"\textbackslash\{\}\textbackslash\{\}cdots",last].join("+");
                    \})()]]\}.
  \item So, if we average the values of each of these outcomes,
                    we get the expected value we will roll, which is
                    0 =
                        [[mixedFractionFromImproper(ANS\_N,SIDES,true,true)]].
  \item The expected value is a weighted average of the possible values with
                    the weights determined by the probability of observing that value.
  \item This means the expected value, considering both the price
                    of the ticket and the possible winnings is
                    E =  (money gained when you win)
                    $\cdot$ (probability of winning) +
                    (money gained when you lose)
                    $\cdot$ (probability of losing).
  \item Let's figure out each of these terms one at a time. The
                    money you gain when you win is
                    $\frac{1}{NaN} and from the question, we
                    know the probability of winning is
                    [[ODD_F]].$
  \item When you lose, you gain no money, or
                    $0, and the probability of losing is 1
                    - [[ODD_F]].$
  \item Putting it all together, the expected value is
                    E = ($\frac{1}{NaN})
                    ([[ODD_F]]) + ($0) (1 - [[ODD\_F]]) =
                    $ $\textbackslash\{\}frac\{\textbackslash\{\}frac\{1\}\{NaN\}$}{[[BUY ?
                    COST*ODDS + randRange(1,3)*100 :
                    COST*ODDS - randRange(1,3)*100]]} =
                    $\textbackslash\{\}frac\{NaN\}\{NaN\}.
  \item $\frac{NaN}{NaN} -
                    $300 is negative.
                    
                    So, we expect to lose money by buying a lottery ticket, because
                    the expected value is negative.
\end{itemize}
\end{document}
