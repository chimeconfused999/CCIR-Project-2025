% Auto-converted from khan-exercises
\documentclass{article}
\usepackage{amsmath,amssymb}
\usepackage[T1]{fontenc}
\usepackage{textcomp}
\newcommand{\abs}[1]{\lvert #1\rvert}

\begin{document}
\section*{Reading tables 2}
\textbf{Question.} What number should go in the empty cell?

\textbf{Answer.} [[ANSWER]] [[UNIT]]

\textbf{Hints.}
\begin{itemize}
  \item The table is missing [[HINT1]].
  \item The table accounts for [[ENROLLMENTS[ ROW\_INDEX ].slice( 0, COL\_INDEX ).join( "+" )]] [[UNIT]].
                        
                    
                    
                        [[HINT2]] is simply the sum, or [[ANSWER]].
  \item [[HINT2]] is [[ENROLLMENTS[ ROW\_INDEX ][ COLUMNS.length - 1 ]]].
                    
                
                
                    The table already accounts for [[ENROLLMENTS[ ROW\_INDEX ].slice( 0, COL\_INDEX ).concat( ENROLLMENTS[ ROW\_INDEX ].slice( COL\_INDEX + 1, COLUMNS.length - 1 ) ).join( "+" )]] =
                    [[ENROLLMENTS[ ROW\_INDEX ][ COLUMNS.length - 1 ] - ANSWER]] [[UNIT]].
                    
                
                The missing number must be the difference between [[HINT2.slice( 0, 1 ).toLowerCase() + HINT2.slice( 1 )]], [[ENROLLMENTS[ ROW\_INDEX ][ COLUMNS.length - 1 ]]], and the values already accounted for, [[ENROLLMENTS[ ROW\_INDEX ][ COLUMNS.length - 1 ] - ANSWER]].
                
                    [[ENROLLMENTS[ ROW\_INDEX ][ COLUMNS.length - 1 ]]] - [[ENROLLMENTS[ ROW\_INDEX ][ COLUMNS.length - 1 ] - ANSWER]] = [[ANSWER]]
\end{itemize}
\end{document}
