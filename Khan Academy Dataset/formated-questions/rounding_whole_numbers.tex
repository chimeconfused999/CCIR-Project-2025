% Auto-converted from khan-exercises
\documentclass{article}
\usepackage{amsmath,amssymb}
\usepackage[T1]{fontenc}
\usepackage{textcomp}
\newcommand{\abs}[1]{\lvert #1\rvert}

\begin{document}
\section*{Rounding whole numbers}
\textbf{Question.} Round [[commafy(NUM)]] to the nearest [[numberPlaceNames[PLACE].toString()]].

\textbf{Answer.} [[Math.round(roundTo(-PLACE, NUM))]]

\textbf{Hints.}
\begin{itemize}
  \item Because we want to round to the [[plural\_form(numberPlaceNames[PLACE], 2).toString()]] place, we need to look at the digit in the [[plural\_form(numberPlaceNames[PLACE - 1], 2)]] place.
  \item The digit in the [[plural\_form(numberPlaceNames[PLACE - 1], 2)]] place is 5.
  \item Because the [[plural\_form(numberPlaceNames[PLACE - 1], 2)]] place digit is 5,
            we round up to [[commafy(ROUNDED)]].
  \item Because 5 is less than 5, we round down to [[commafy(ROUNDED)]].
\end{itemize}
\end{document}
