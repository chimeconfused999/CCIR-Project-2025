% Auto-converted from khan-exercises
\documentclass{article}
\usepackage{amsmath,amssymb}
\usepackage[T1]{fontenc}
\usepackage{textcomp}
\newcommand{\abs}[1]{\lvert #1\rvert}

\begin{document}
\section*{Solving for the x-intercept}
\textbf{Question.} What is the x-intercept?

\textbf{Answer.} \textbackslash\{\}Large\{(\}-8,\textbackslash\{\} 0\textbackslash\{\}Large\{)\}

\textbf{Hints.}
\begin{itemize}
  \item The x-intercept is the point where the line crosses the x-axis.  This happens when y is zero.
  \item Set y to zero and solve for x.
            *,2,x + *,8,0 = -16
  \item *,2,x = -16
  \item (1, 2) $\cdot$ (2x) = (1, 2) $\cdot$ (-16)
            x = -8
  \item This line intersects the x-axis at (-8, 0).
\end{itemize}
\end{document}
