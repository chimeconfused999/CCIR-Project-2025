% Auto-converted from khan-exercises
\documentclass{article}
\usepackage{amsmath,amssymb}
\usepackage[T1]{fontenc}
\usepackage{textcomp}
\newcommand{\abs}[1]{\lvert #1\rvert}

\begin{document}
\section*{Ideal stoichiometry}
\textbf{Question.} How many moles of \textbackslash\{\}text\{KCl\} will be produced from
        58 \textbackslash\{\}text\{g\} of \textbackslash\{\}text\{KClO\}\_3?

\textbf{Answer.} [[roundTo(3, GIVEN\_MOL * P1\_RATIO / GIVEN\_RATIO)]] moles (you can round to the nearest tenth of a mole)

\textbf{Hints.}
\begin{itemize}
  \item \textbackslash\{\}dfrac\{58 \textbackslash\{\}cancel\{\textbackslash\{\}text\{g\}\}\}\{[[roundTo(3, molarMass("K") + molarMass("Cl") + molarMass("O") * 3)]] \textbackslash\{\}cancel\{\textbackslash\{\}text\{g\}\} / \textbackslash\{\}text\{mol\}\} =
                \textbackslash\{\}text\{ [[plural(GIVEN\_MOL, "mole")]]\} \textbackslash\{\}text\{ of \}\textbackslash\{\}text\{KClO\}\_3
            
            [Explain]
            
                
                    First we want to convert the given amount of \textbackslash\{\}text\{KClO\}\_3 from grams to moles. To do this, we divide
                    the given amount of \textbackslash\{\}text\{KClO\}\_3 by the molar mass of \textbackslash\{\}text\{KClO\}\_3.
                
                     \textbackslash\{\}dfrac\{\textbackslash\{\}text\{grams of \}\textbackslash\{\}text\{KClO\}\_3\}\{\textbackslash\{\}text\{molar mass of \}\textbackslash\{\}text\{KClO\}\_3\} = \textbackslash\{\}text\{moles of \}\textbackslash\{\}text\{KClO\}\_3
                
                    To find the molar mass of \textbackslash\{\}text\{KClO\}\_3, we look up the atomic weight of each atom in a molecule of
                    \textbackslash\{\}text\{KClO\}\_3 in the periodic table and add them together.
                    In this case, it's [[roundTo(3, molarMass("K") + molarMass("Cl") + molarMass("O") * 3)]] \textbackslash\{\}text\{g/mol\}.
                
                    Dividing the given 58 \textbackslash\{\}text\{g\} of \textbackslash\{\}text\{KClO\}\_3 by the molar mass of
                    [[roundTo(3, molarMass("K") + molarMass("Cl") + molarMass("O") * 3)]] \textbackslash\{\}text\{g/mol\} tells us we're starting with
                    \textbackslash\{\}text\{[[roundTo(3, GIVEN\_MASS / GIVEN\_MOLAR\_MASS)]] [[plural\_form(MOLE, GIVEN\_MOL)]]\} of \textbackslash\{\}text\{KClO\}\_3.
  \item The mole ratio of \textbackslash\{\}dfrac\{\textbackslash\{\}text\{KClO\}\_3\}\{\textbackslash\{\}text\{KCl\}\} in the reaction is
                $\frac{2}{2}$.
                [Explain]
            
            
                
                    The reaction is \textbackslash\{\}blue\{2\}\textbackslash\{\}text\{KClO\}\_3
                     \textbackslash\{\}rightarrow
                    \textbackslash\{\}red\{2\}\textbackslash\{\}text\{KCl\}
                     + 3\textbackslash\{\}text\{O\}\_2.
                    The coefficients in front of each molecule tell us in what ratios the molecules react. In this case
                    [[cardinalThrough20(GIVEN\_RATIO)]] \textbackslash\{\}text\{KClO\}\_3 for every
                    [[cardinalThrough20(P1\_RATIO)]] \textbackslash\{\}text\{KCl\} molecules.
                
            
            \textbackslash\{\}qquad
                \textbackslash\{\}dfrac\{\textbackslash\{\}text\{KClO\}\_3\}\{\textbackslash\{\}text\{KCl\}\} = $\frac{2}{2}$ =
                \textbackslash\{\}dfrac\{\textbackslash\{\}text\{ [[plural(GIVEN\_MOL, "mole")]]\}\}\{x\}
  \item x = \textbackslash\{\}text\{ [[roundTo(3, GIVEN\_MOL * P1\_RATIO / GIVEN\_RATIO)]] [[plural\_form(MOLE, P1\_MOL)]]\} of \textbackslash\{\}text\{KCl\} produced.
\end{itemize}
\end{document}
