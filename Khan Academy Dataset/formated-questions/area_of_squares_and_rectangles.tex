% Auto-converted from khan-exercises
\documentclass{article}
\usepackage{amsmath,amssymb}
\usepackage[T1]{fontenc}
\usepackage{textcomp}
\newcommand{\abs}[1]{\lvert #1\rvert}

\begin{document}
\section*{Area problems}
\textbf{Question.} One side of a square is area
                [[plural\_form(UNIT\_TEXT, S)]] long. What is its area?

\textbf{Answer.} [[S * S]] square
                [[plural\_form(UNIT\_TEXT)]]

\textbf{Hints.}
\begin{itemize}
  \item The area is the length times the width.
  \item The length is area [[plural\_form(UNIT\_TEXT, S)]] and
                        the width is area [[plural\_form(UNIT\_TEXT, S)]], so the
                        area is area\textbackslash\{\}timesarea
                        square [[plural\_form(UNIT\_TEXT, S * S)]].
  \item \textbackslash\{\}qquad\textbackslash\{\}text\{2\} = area $\times$ area
                        = [[S * S]]
                    
                    
                        We can also count [[S * S]]
                        square [[plural\_form(UNIT\_TEXT, S * S)]].
  \item The area is the length times the width.
                    
                    \textbackslash\{\}qquad \textbackslash\{\}pink\{\textbackslash\{\}text\{?\}\} $\times$ \textbackslash\{\}pink\{\textbackslash\{\}text\{?\}\} =
                        [[S * S]]\textbackslash\{\}text\{ [[randFromArray(metricUnits.concat([genericUnit]))]]\}
  \item \textbackslash\{\}qquad \textbackslash\{\}pink\{area\} $\times$
                        \textbackslash\{\}pink\{area\} =
                        [[S * S]]\textbackslash\{\}text\{ [[randFromArray(metricUnits.concat([genericUnit]))]]\}
                    
                    
                        The sides of a square are all the same length, so each
                        side must be area
                        [[plural\_form(UNIT\_TEXT, S)]] long.
  \item The area is the length times the width.
  \item The length is [[L]] [[plural\_form(UNIT\_TEXT, L)]].
                        The width is NaNNaNNaNNaNNaNNaNNaNNaNNaNNaNNaNNaNNaNNaNNaNNaNNaNNaNNaNNaNNaNNaNNaNNaNNaNNaNNaNNaNNaNNaNNaNNaNNaNNaNNaNNaNNaNNaNNaN[[randRange(W + 1, 9)]]111111111111111111111111111111111111111 [[plural\_form(UNIT\_TEXT, W)]].
                        So the area is [[L]]\textbackslash\{\}timesNaNNaNNaNNaNNaNNaNNaNNaNNaNNaNNaNNaNNaNNaNNaNNaNNaNNaNNaNNaNNaNNaNNaNNaNNaNNaNNaNNaNNaNNaNNaNNaNNaNNaNNaNNaNNaNNaNNaN[[randRange(W + 1, 9)]]111111111111111111111111111111111111111
                        square [[plural\_form(UNIT\_TEXT, L * W)]].
  \item \textbackslash\{\}qquad\textbackslash\{\}text\{2\} = [[L]] $\times$ NaNNaNNaNNaNNaNNaNNaNNaNNaNNaNNaNNaNNaNNaNNaNNaNNaNNaNNaNNaNNaNNaNNaNNaNNaNNaNNaNNaNNaNNaNNaNNaNNaNNaNNaNNaNNaNNaNNaN[[randRange(W + 1, 9)]]111111111111111111111111111111111111111
                        = [[L * W]]
                    
                    
                        We can also count [[L * W]]
                        square [[plural\_form(UNIT\_TEXT, L * W)]].
\end{itemize}
\end{document}
