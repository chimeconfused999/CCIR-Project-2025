% Auto-converted from khan-exercises
\documentclass{article}
\usepackage{amsmath,amssymb}
\usepackage[T1]{fontenc}
\usepackage{textcomp}
\newcommand{\abs}[1]{\lvert #1\rvert}

\begin{document}
\section*{Complementary and supplementary angles}
\textbf{Question.} If \textbackslash\{\}angle [[A + O + C]] is a right angle and m [[ANGLE\_ONE]] = [[ANGLE]]\textasciicircum{}\textbackslash\{\}circ, what is m [[ANGLE\_TWO]]?
                
                
                NOTE: Angles not necessarily drawn to scale.

\textbf{Answer.} [[90 - ANGLE]] \textbackslash\{\}Large\{\textasciicircum{}\textbackslash\{\}circ\}

\textbf{Hints.}
\begin{itemize}
  \item From the diagram, we see that [[ANGLE\_BOT]] and [[ANGLE\_TOP]] are complementary angles.
  \item Therefore, m [[ANGLE\_BOT]] + m [[ANGLE\_TOP]] = 90\textasciicircum{}\textbackslash\{\}circ.
  \item Thus, m [[ANGLE\_TWO]] = 90\textasciicircum{}\textbackslash\{\}circ - m [[ANGLE\_ONE]] = 90\textasciicircum{}\textbackslash\{\}circ - [[ANGLE]]\textasciicircum{}\textbackslash\{\}circ = [[90 - ANGLE]]\textasciicircum{}\textbackslash\{\}circ.
  \item From the diagram, we see that [[ANGLE\_BOT]] and [[ANGLE\_TOP]] are supplementary angles.
  \item Therefore, m [[ANGLE\_BOT]] + m [[ANGLE\_TOP]] = 180\textasciicircum{}\textbackslash\{\}circ.
  \item Thus, m [[ANGLE\_TWO]] = 180\textasciicircum{}\textbackslash\{\}circ - m [[ANGLE\_ONE]] = 180\textasciicircum{}\textbackslash\{\}circ - [[ANGLE]]\textasciicircum{}\textbackslash\{\}circ = [[180 - ANGLE]]\textasciicircum{}\textbackslash\{\}circ.
\end{itemize}
\end{document}
