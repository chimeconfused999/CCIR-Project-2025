% Auto-converted from khan-exercises
\documentclass{article}
\usepackage{amsmath,amssymb}
\usepackage[T1]{fontenc}
\usepackage{textcomp}
\newcommand{\abs}[1]{\lvert #1\rvert}

\begin{document}
\section*{Angles of a polygon}
\textbf{Question.} What is the sum of this polygon's interior angles?

\textbf{Answer.} 720 \textbackslash\{\}Large\{\textasciicircum{}\textbackslash\{\}circ\}

\textbf{Hints.}
\begin{itemize}
  \item There are a couple of ways to approach this problem.
  \item Since this polygon has 6 sides,
                        we can draw 6 triangles that all meet in the center.
  \item We can combine all the triangles' angles, and then we must subtract 360\textasciicircum{}\{\textbackslash\{\}circ\} because the circle in the middle is extra.
  \item There are 180\textasciicircum{}\{\textbackslash\{\}circ\} in a triangle.
  \item \textbackslash\{\}begin\{align*\}\&6 $\times$ 180\textasciicircum{}\{\textbackslash\{\}circ\} - 360\textasciicircum{}\{\textbackslash\{\}circ\} \textbackslash\{\}\textbackslash\{\}
                    \&= 1080\textasciicircum{}\{\textbackslash\{\}circ\} - 360\textasciicircum{}\{\textbackslash\{\}circ\} \textbackslash\{\}\textbackslash\{\}
                    \&= 720\textasciicircum{}\{\textbackslash\{\}circ\}\textbackslash\{\}end\{align*\}
  \item An alternative approach is shown below.
                    We can use four of the [[cardinalThrough20( SIDES )]] sides to make two triangles, as shown in orange.
  \item There are 2 sides between the orange triangles,
                    to make 2 additional triangles.
  \item We chopped this polygon into 4 triangles,
                    and each triangle's angles sum to 180\textasciicircum{}\{\textbackslash\{\}circ\}.
  \item 4 $\times$ 180\textasciicircum{}\{\textbackslash\{\}circ\} = 720\textasciicircum{}\{\textbackslash\{\}circ\}
                    The sum of the polygon's interior angles is 720\textasciicircum{}\{\textbackslash\{\}circ\}.
  \item The exterior angles are shown above.
  \item The exterior angles fit together to form a circle.
  \item Therefore, the sum of the exterior angles is 360\textasciicircum{}\{\textbackslash\{\}circ\}.
\end{itemize}
\end{document}
