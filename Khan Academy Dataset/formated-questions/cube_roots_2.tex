% Auto-converted from khan-exercises
\documentclass{article}
\usepackage{amsmath,amssymb}
\usepackage[T1]{fontenc}
\usepackage{textcomp}
\newcommand{\abs}[1]{\lvert #1\rvert}

\begin{document}
\section*{Cube roots 2}
\textbf{Question.} \textbackslash\{\}Large\{\textbackslash\{\}sqrt[3]\{270\} = \textbackslash\{\}text\{?\}\}

\textbf{Answer.} 270

\textbf{Hints.}
\begin{itemize}
  \item \textbackslash\{\}sqrt[3]\{270\} is the number that, when multiplied by itself three times, equals 270.
  \item First break down 270 into its prime factorization and look for factors that appear three times.
  \item Let's draw a factor tree.
  \item So the prime factorization of 270 is 2$\times$ 3$\times$ 3$\times$ 3$\times$ 5.
  \item Notice that we can rearrange the factors like so:
                        270 = 2 $\times$ 3 $\times$ 3 $\times$ 3 $\times$ 5 = 
                            [[MULTIPLES.join(" \textbackslash\{\}$\times$ ")]] $\times$ 2$\times$ 5
  \item So \textbackslash\{\}sqrt[3]\{270\} = 
                        [[ROOTS.join(" \textbackslash\{\}$\times$ ")]] $\times$ \textbackslash\{\}sqrt[3]\{2$\times$ 5\}
  \item \textbackslash\{\}sqrt[3]\{270\} =
                        3 $\times$ \textbackslash\{\}sqrt[3]\{2$\times$ 5\}
  \item \textbackslash\{\}sqrt[3]\{270\} = 3 \textbackslash\{\}sqrt[3]\{10\}
\end{itemize}
\end{document}
