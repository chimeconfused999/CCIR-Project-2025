% Auto-converted from khan-exercises
\documentclass{article}
\usepackage{amsmath,amssymb}
\usepackage[T1]{fontenc}
\usepackage{textcomp}
\newcommand{\abs}[1]{\lvert #1\rvert}

\begin{document}
\section*{Partial fraction expansion}
\textbf{Question.} Expand
                $\frac{[[NUMER]]}{[[DENOM]]}$
                using partial fractions.

\textbf{Answer.} [[A]]
                    [[B]]
                    [[C]]
                    [[D]]
                
                
                    [[B]]
                    [[A]]
                    [[D]]
                    [[C]]
                
                
                    
                        
                            
                                
                                    a
                                
                                
                                    +
                                
                                
                                    a
                                
                            
                            
                                
                                    x-a
                                
                                
                                    x-a

\textbf{Hints.}
\begin{itemize}
  \item First, factor the denominator to find the denominators of
                    the two fractions we will split our fraction into.
  \item [[DENOM]] =
                        ([[ADENOM]])([[BDENOM]])
  \item Because the original denominator can be factored into these
                    two parts, we can write out our original fraction as the
                    sum of two fractions whose denominators are the two factors
                    we just found.
  \item $\frac{[[NUMER]]}{
                            ([[ADENOM]])([[BDENOM]])
                        }$ =
                        $\frac{?}{[[ADENOM]]}$ +
                        $\frac{?}{[[BDENOM]]}$
  \item Now, we replace the numerators with polynomials of a
                        degree one less than the degree of the polynomial in
                        the denominator.
                    
                    
                        In our case, both of the denominators have a degree of
                        1, so we replace our numerators with
                        polynomials of degree 0, or constants. We
                        will use the constants A and
                        B.
  \item $\frac{[[NUMER]]}{
                            ([[ADENOM]])([[BDENOM]])
                        }$ =
                        $\frac{A}{[[ADENOM]]}$ +
                        $\frac{B}{[[BDENOM]]}$
  \item Now, to get rid of the fractions, we multiply by the common
                    denominator,
                    ([[ADENOM]])([[BDENOM]]).
  \item [[NUMER]] =
                        A([[BDENOM]]) + B([[ADENOM]])
  \item Now we can solve for A and B. An
                    easy way to do this is to try to choose values for
                    x that will get one of A or
                    B to cancel out, and then solve for the other
                    one.
  \item Let's try to cancel out B. We see that if
                        we plug in [[C]] for
                        x, the term with B cancels
                        out, and we are left with:
                    
                    
                        
                            [[expr(["+", E * C, F])]] =
                            A([[expr(["+", C, -D])]])
  \item [[E * C + F]] =
                        [[expr(["*", C - D, "A"])]]
  \item A=[[A]]
  \item We can do the same thing to solve for B,
                        but instead plugging in [[D]] for
                        x:
                    
                    
                        
                            [[expr(["+", E * D, F])]] =
                            B([[expr(["+", D, -C])]])
  \item [[E * D + F]] =
                        [[expr(["*", D - C, "B"])]]
  \item B=[[B]]
  \item Now, we plug back in to our fractions, and get:
                    
                    
                        
                            $\frac{[[NUMER]]}{[[DENOM]]}$ =
                            $\frac{[[A]]}{[[ADENOM]]}$ +
                            $\frac{[[B]]}{[[BDENOM]]}$
\end{itemize}
\end{document}
