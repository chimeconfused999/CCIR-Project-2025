% Auto-converted from khan-exercises
\documentclass{article}
\usepackage{amsmath,amssymb}
\usepackage[T1]{fontenc}
\usepackage{textcomp}
\newcommand{\abs}[1]{\lvert #1\rvert}

\begin{document}
\section*{Pythagorean identities}
\textbf{Question.} 

\textbf{Answer.} [[FUNC]]

\textbf{Hints.}
\begin{itemize}
  \item We can use the identity 
                    \textbackslash\{\}blue\{\textbackslash\{\}sin\textasciicircum{}2 \textbackslash\{\}theta\} + \textbackslash\{\}orange\{\textbackslash\{\}cos\textasciicircum{}2 \textbackslash\{\}theta\}
                        = 1
                    to simplify this expression.
                    
                    We can see why this is true by using the 
                    Pythagorean Theorem.
  \item So, (\textbackslash\{\}sin\textasciicircum{}2 \textbackslash\{\}theta + \textbackslash\{\}cos\textasciicircum{}2 \textbackslash\{\}theta)([[FUNC]]) 
                        = 1 $\cdot$ [[FUNC]] = [[FUNC]]
  \item So, \textbackslash\{\}dfrac\{[[FUNC]]\}
                        \{\textbackslash\{\}sin\textasciicircum{}2 \textbackslash\{\}theta + \textbackslash\{\}cos\textasciicircum{}2 \textbackslash\{\}theta\} = 
                        $\frac{[[FUNC]]}{1}$ = [[FUNC]]
  \item We can use the identity 
                    \textbackslash\{\}blue\{\textbackslash\{\}sin\textasciicircum{}2 \textbackslash\{\}theta\} + \textbackslash\{\}orange\{\textbackslash\{\}cos\textasciicircum{}2 \textbackslash\{\}theta\}
                        = 1
                    to simplify this expression.
                    
                    We can see why this is true by using the 
                    Pythagorean Theorem.
  \item So, [[IDENT]] = [[EQUIV]]
  \item Plugging into our expression, we get
                    
                        \textbackslash\{\}qquad
                            ([[IDENT]])([[FUNC]])
                            =
                            ([[EQUIV]])([[FUNC]]) 
                        
                    
                    
                        \textbackslash\{\}qquad
                            $\frac{[[IDENT]]}{[[FUNC]]}$
                            =
                            $\frac{[[EQUIV]]}{[[FUNC]]}$
  \item To make simplifying easier, let's put everything
                    in terms of \textbackslash\{\}sin and \textbackslash\{\}cos. 
                    [[FUNC]] = [[FUNC\_SIMP]],
                    so we can plug that in to get
                    
                        \textbackslash\{\}qquad
                            ([[EQUIV]])([[FUNC]])
                            =
                            \textbackslash\{\}left([[EQUIV]]\textbackslash\{\}right)
                            \textbackslash\{\}left([[FUNC\_SIMP]]\textbackslash\{\}right) 
                        
                    
                    
                        \textbackslash\{\}qquad
                            $\frac{[[EQUIV]]}{[[FUNC]]}$
                            =
                            $\frac{[[EQUIV]]}{[[FUNC_SIMP]]}$
  \item This is [[ANS]].
  \item We can derive a useful identity from 
                    \textbackslash\{\}blue\{\textbackslash\{\}sin\textasciicircum{}2 \textbackslash\{\}theta\} + \textbackslash\{\}orange\{\textbackslash\{\}cos\textasciicircum{}2 \textbackslash\{\}theta\}
                        = 1
                    to simplify this expression.
                    
                    We can see why this identity is true by using the 
                    Pythagorean Theorem.
  \item Dividing both sides by \textbackslash\{\}cos\textasciicircum{}2\textbackslash\{\}theta, we get
                    \textbackslash\{\}qquad $\frac{\sin^2\theta}{\cos^2\theta}$ 
                    + $\frac{\cos^2\theta}{\cos^2\theta}$ 
                    = $\frac{1}{\cos^2\theta}$
                    
                        \textbackslash\{\}qquad \textbackslash\{\}tan\textasciicircum{}2\textbackslash\{\}theta + 1 = \textbackslash\{\}sec\textasciicircum{}2\textbackslash\{\}theta
                    
                    \textbackslash\{\}qquad [[IDENT]] 
                        = [[EQUIV]]
  \item Plugging into our expression, we get
                    
                        \textbackslash\{\}qquad
                            ([[IDENT]])([[FUNC]])
                            =
                            \textbackslash\{\}left([[EQUIV]]\textbackslash\{\}right)
                            \textbackslash\{\}left([[FUNC]]\textbackslash\{\}right) 
                        
                    
                    
                        \textbackslash\{\}qquad
                            $\frac{[[IDENT]]}{[[FUNC]]}$
                            = 
                            $\frac{[[EQUIV]]}{[[FUNC]]}$
  \item To make simplifying easier, let's put everything
                    in terms of \textbackslash\{\}sin and \textbackslash\{\}cos.
                    We know [[EQUIV]] 
                        = [[EQUIV\_SIMP]]
                    and [[FUNC]] = [[FUNC\_SIMP]],
                    so we can substitute to get
                    
                        \textbackslash\{\}qquad 
                            \textbackslash\{\}left([[EQUIV]]\textbackslash\{\}right)
                            \textbackslash\{\}left([[FUNC]]\textbackslash\{\}right) 
                            =
                            \textbackslash\{\}left([[EQUIV\_SIMP]]\textbackslash\{\}right)
                            \textbackslash\{\}left([[FUNC\_SIMP]]\textbackslash\{\}right) 
                        
                    
                    
                        \textbackslash\{\}qquad
                            $\frac{[[EQUIV]]}{[[FUNC]]}$
                            =
                            $\frac{[[EQUIV_SIMP]]}{[[FUNC_SIMP]]}$
  \item To make simplifying easier, let's put everything
                    in terms of \textbackslash\{\}sin and \textbackslash\{\}cos.
                    We know [[EQUIV]] 
                        = [[EQUIV\_SIMP]], so we can substitute
                    to get
                    
                        \textbackslash\{\}qquad
                            \textbackslash\{\}left([[EQUIV]]\textbackslash\{\}right)
                            \textbackslash\{\}left([[FUNC]]\textbackslash\{\}right) 
                            =
                            \textbackslash\{\}left([[EQUIV\_SIMP]]\textbackslash\{\}right)
                            \textbackslash\{\}left([[FUNC\_SIMP]]\textbackslash\{\}right) 
                        
                    
                    
                        \textbackslash\{\}qquad
                            $\frac{[[EQUIV]]}{[[FUNC]]}$
                            =
                            $\frac{[[EQUIV_SIMP]]}{[[FUNC_SIMP]]}$
  \item This is [[ANS\_SIMP]] = [[ANS]].
  \item This is [[ANS]].
  \item We can derive a useful identity from 
                    \textbackslash\{\}blue\{\textbackslash\{\}sin\textasciicircum{}2 \textbackslash\{\}theta\} + \textbackslash\{\}orange\{\textbackslash\{\}cos\textasciicircum{}2 \textbackslash\{\}theta\}
                        = 1
                    to simplify this expression.
                    
                    We can see why this identity is true by using the 
                    Pythagorean Theorem.
  \item Dividing both sides by \textbackslash\{\}sin\textasciicircum{}2\textbackslash\{\}theta, we get
                    \textbackslash\{\}qquad $\frac{\sin^2\theta}{\sin^2\theta}$ 
                    + $\frac{\cos^2\theta}{\sin^2\theta}$ 
                    = $\frac{1}{\sin^2\theta}$
                    
                        \textbackslash\{\}qquad 1 + \textbackslash\{\}cot\textasciicircum{}2\textbackslash\{\}theta = \textbackslash\{\}csc\textasciicircum{}2\textbackslash\{\}theta
                    
                    \textbackslash\{\}qquad [[IDENT]] 
                        = [[EQUIV]]
  \item Plugging into our expression, we get
                    
                        \textbackslash\{\}qquad
                            ([[IDENT]])([[FUNC]])
                            =
                            \textbackslash\{\}left([[EQUIV]]\textbackslash\{\}right)
                            \textbackslash\{\}left([[FUNC]]\textbackslash\{\}right) 
                        
                    
                    
                        \textbackslash\{\}qquad
                            $\frac{[[IDENT]]}{[[FUNC]]}$
                            = 
                            $\frac{[[EQUIV]]}{[[FUNC]]}$
  \item To make simplifying easier, let's put everything
                    in terms of \textbackslash\{\}sin and \textbackslash\{\}cos.
                    We know [[EQUIV]] 
                        = [[EQUIV\_SIMP]]
                    and [[FUNC]] = [[FUNC\_SIMP]],
                    so we can substitute to get
                    
                        \textbackslash\{\}qquad 
                            \textbackslash\{\}left([[EQUIV]]\textbackslash\{\}right)
                            \textbackslash\{\}left([[FUNC]]\textbackslash\{\}right) 
                            =
                            \textbackslash\{\}left([[EQUIV\_SIMP]]\textbackslash\{\}right)
                            \textbackslash\{\}left([[FUNC\_SIMP]]\textbackslash\{\}right) 
                        
                    
                    
                        \textbackslash\{\}qquad
                            $\frac{[[EQUIV]]}{[[FUNC]]}$
                            =
                            $\frac{[[EQUIV_SIMP]]}{[[FUNC_SIMP]]}$
  \item To make simplifying easier, let's put everything
                    in terms of \textbackslash\{\}sin and \textbackslash\{\}cos.
                    We know [[EQUIV]]
                        = [[EQUIV\_SIMP]], so we can substitute
                    to get
                    
                        \textbackslash\{\}qquad
                            \textbackslash\{\}left([[EQUIV]]\textbackslash\{\}right)
                            \textbackslash\{\}left([[FUNC]]\textbackslash\{\}right) 
                            =
                            \textbackslash\{\}left([[EQUIV\_SIMP]]\textbackslash\{\}right)
                            \textbackslash\{\}left([[FUNC\_SIMP]]\textbackslash\{\}right) 
                        
                    
                    
                        \textbackslash\{\}qquad
                            $\frac{[[EQUIV]]}{[[FUNC]]}$
                            =
                            $\frac{[[EQUIV_SIMP]]}{[[FUNC_SIMP]]}$
  \item This is [[ANS\_SIMP]] = [[ANS]].
  \item This is [[ANS]].
\end{itemize}
\end{document}
