% Auto-converted from khan-exercises
\documentclass{article}
\usepackage{amsmath,amssymb}
\usepackage[T1]{fontenc}
\usepackage{textcomp}
\newcommand{\abs}[1]{\lvert #1\rvert}

\begin{document}
\section*{Markup and commission word problems}
\textbf{Question.} Alex's Furniture Store buys a [[furniture(1)]]
                at a wholesale price of $137.00.
                If the markup rate at Alex's Furniture Store is 60\%,
                what is the markup for the [[furniture(1)]]?$

\textbf{Answer.} \textbackslash\{\}$\ 2.00$

\textbf{Hints.}
\begin{itemize}
  \item Remember that a markup rate is a percentage of the wholesale price that a store adds to get a selling or retail price. The amount of markup can be found with the following equation: 
                    markup rate $\times$ wholesale price = amount of markup
  \item Since the markup rate is a percentage, we have to convert it into a decimal first. Percent means "out of one hundred," so 60\textbackslash\{\}\% is equivalent to $\frac{60}{100}$ which is also equal to 60 \textbackslash\{\}div 100.
                    60 \textbackslash\{\}div 100 = 0.60
  \item Now you have all the information you need to use the formula above!
  \item 0.60 $\times$ $137.00 = $2.00
                    The amount of markup on the [[furniture(1)]] is $2.00.$
  \item In order to find the retail price, we must first find the amount of markup. Remember that a markup rate is a percentage of the wholesale price that a store adds to get a selling or retail price. With this knowledge, we can figure out the following equation:
                    markup rate $\times$ wholesale price = amount of markup
  \item Since the markup rate is a percentage, we have to convert it into a decimal first. Percent means "out of one hundred," so 60\textbackslash\{\}\% is equivalent to $\frac{60}{100}$ which is also equal to 60 \textbackslash\{\}div 100.
                    60 \textbackslash\{\}div 100 = 0.60
  \item Now you have all the information you need to find the amount of markup!
                    0.60 $\times$ $137.00 = $2.00
  \item Since the markup rate is a percentage of the wholesale price that is added to get the retail price, we can find the retail price with the following equation:
                    amount of markup + wholesale price = retail price
  \item $2.00 + $137.00 = $2.00
                    The retail price of the [[furniture(1)]] should be $2.00.
  \item A commission rate is a percentage of the retail price of an item that the seller makes if he or she can sell the item. To find the amount of commission made, use the following formula:
                    commission rate $\times$ retail price = amount of commission made
  \item Since the commission rate is a percentage, we have to convert it into a decimal first. Percent means "out of one hundred," so 9\textbackslash\{\}\% is equivalent to $\frac{9}{100}$ which is also equal to 9 \textbackslash\{\}div 100.
                    9 \textbackslash\{\}div 100 = 0.09
  \item Now you have all the information you need to use the formula above!
  \item 0.09 $\times$ $542.00 = $2.00
                    The amount of commission Alex makes by selling a [[electronic(1)]] is $2.00.$
  \item First, find the amount of commission made by using the following formula:
                    commission rate $\times$ total sales = amount of commission made
  \item Since the commission rate is a percentage, we have to convert it into a decimal first. Percent means "out of one hundred," so 9\textbackslash\{\}\% is equivalent to $\frac{9}{100}$ which is also equal to 9 \textbackslash\{\}div 100.
                    9 \textbackslash\{\}div 100 = 0.09
  \item Now you have all the information you need to find the amount of commission made:
                    0.09 $\times$ $550.00 = $2.00
  \item We can find the total salary for Alex by adding the amount of commission made to his base salary.We can find the total salary for Alex by adding the amount of commission made to her base salary.
                    amount of commission made + base salary = total salary
  \item $2.00 + $350.00 = $NaN
                    The total salary Alex made last week was $NaN.
\end{itemize}
\end{document}
