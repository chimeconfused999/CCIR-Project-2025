% Auto-converted from khan-exercises
\documentclass{article}
\usepackage{amsmath,amssymb}
\usepackage[T1]{fontenc}
\usepackage{textcomp}
\newcommand{\abs}[1]{\lvert #1\rvert}

\begin{document}
\section*{Ellipse intuition}
\textbf{Question.} The sum of the distances between each point on an ellipse and the ellipse's two foci is always the same.
                    Find the foci of the ellipse below by moving the orange points to their correct positions.
                    You can check your answer by moving your cursor across the ellipse's perimeter and seeing how the sum of distances from the foci changes.

\textbf{Answer.} [DUMMY\_GRAPH.focus1.coord, DUMMY\_GRAPH.focus2.coord]
                        
                            if (\_.isEqual(guess, FOCUS\_START)) \{
                                return "You need to move the foci to the correct positions.";
                            \}

                            return (guess[0][0] === F1[0] \&\& guess[0][1] === F1[1] \&\&
                                    guess[1][0] === F2[0] \&\& guess[1][1] === F2[1]) ||
                                   (guess[0][0] === F2[0] \&\& guess[0][1] === F2[1] \&\&
                                    guess[1][0] === F1[0] \&\& guess[1][1] === F1[1]);
                        
                        
                            graph.focus1.setCoord(guess[0]);
                            graph.focus2.setCoord(guess[1]);

\textbf{Hints.}
\begin{itemize}
  \item Both foci must lie along the major axis of the ellipse.
  \item The foci must be equally distant from the center of the ellipse.
  \item We can adjust the positions of the foci along the major axis
                        until we've found the unique state where the sum of the distances is constant.
  \item One focus is ([[F1[0]]], [[F1[1]]])
                            and the other is ([[F2[0]]], [[F2[1]]]).
\end{itemize}
\end{document}
