% Auto-converted from khan-exercises
\documentclass{article}
\usepackage{amsmath,amssymb}
\usepackage[T1]{fontenc}
\usepackage{textcomp}
\newcommand{\abs}[1]{\lvert #1\rvert}

\begin{document}
\section*{Vertical angles}
\textbf{Question.} If we know that the blue angle is pink angle\textasciicircum{}\textbackslash\{\}circ,
                what is the measure of angle x?

\textbf{Answer.} pink angle \textbackslash\{\}Large\{\textasciicircum{}\textbackslash\{\}circ\}

\textbf{Hints.}
\begin{itemize}
  \item What is the measure of the pink angle?
  \item The pink and blue angles add up to 180\textasciicircum{}\textbackslash\{\}circ because they are adjacent and form a straight line.
            
                \textbackslash\{\}pink\{\textbackslash\{\}text\{green angle\}\} = 180\textasciicircum{}\{\textbackslash\{\}circ\} - \textbackslash\{\}blue\{pink angle\textasciicircum{}\{\textbackslash\{\}circ\}\} = \textbackslash\{\}green\{[[180 - MEASURE]]\textasciicircum{}\{\textbackslash\{\}circ\}\}
  \item The pink and green angles also add up to 180\textasciicircum{}\textbackslash\{\}circ because they are adjacent and form a straight line too.
            
                \textbackslash\{\}green\{\textbackslash\{\}text\{[[GREEN\_ANGLE]]\}\}= 180\textasciicircum{}\{\textbackslash\{\}circ\} - \textbackslash\{\}pink\{[[180 - MEASURE]]\textasciicircum{}\{\textbackslash\{\}circ\}\} = \textbackslash\{\}green\{pink angle\textasciicircum{}\{\textbackslash\{\}circ\}\}
\end{itemize}
\end{document}
