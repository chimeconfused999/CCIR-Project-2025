% Auto-converted from khan-exercises
\documentclass{article}
\usepackage{amsmath,amssymb}
\usepackage[T1]{fontenc}
\usepackage{textcomp}
\newcommand{\abs}[1]{\lvert #1\rvert}

\begin{document}
\section*{Telling time 0.5}
\textbf{Question.} What time is it?

\textbf{Answer.} The time is: 5 : 00 [[icu.getDateFormatSymbols().am\_pm[HOUR >= 7 ? 0 : 1]]]

\textbf{Hints.}
\begin{itemize}
  \item The small hand is for the hour, and the big hand is for the minutes.
  \item The hour hand is pointing at 5, so the hour is 5.
                    The hour hand is between 5 and 6, so the hour is 5.
                    The hour hand is close to but hasn't passed 6, so the hour is still 5.
  \item The minute hand starts pointing straight up for 0 minutes, and makes a complete circle in 1 hour.
                    For each quarter of the circle that the minute hand passes, add 15 minutes.
  \item The minute hand has passed 0 fourths of a circle, which represents 0 minutes.
  \item The time is 5:00.
\end{itemize}
\end{document}
