% Auto-converted from khan-exercises
\documentclass{article}
\usepackage{amsmath,amssymb}
\usepackage[T1]{fontenc}
\usepackage{textcomp}
\newcommand{\abs}[1]{\lvert #1\rvert}

\begin{document}
\section*{Geometric sequences 1}
\textbf{Question.} The first [[cardinalThrough20(N)]] terms of a geometric sequence are given:
                7,3,-7,1,21,1, \textbackslash\{\}ldots
                What is the [[ordinalThrough20(N + 1)]] term in the sequence?

\textbf{Answer.} -63

\textbf{Hints.}
\begin{itemize}
  \item In any geometric sequence, each term is equal to the previous term times the common ratio.
  \item Thus, the second term is equal to the first term times the common ratio. In this sequence, the second term, -7, 1, is -3, 1 times the first term, 7, 3.
  \item Therefore, the common ratio is -3, 1.
  \item The [[ordinalThrough20(N + 1)]] term in the sequence is equal to the [[ordinalThrough20(N)]] term times the common ratio, or 21, 1 $\cdot$ -3, 1 = -63, 1.
\end{itemize}
\end{document}
