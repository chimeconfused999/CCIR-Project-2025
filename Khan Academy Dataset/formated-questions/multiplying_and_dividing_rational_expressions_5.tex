% Auto-converted from khan-exercises
\documentclass{article}
\usepackage{amsmath,amssymb}
\usepackage[T1]{fontenc}
\usepackage{textcomp}
\newcommand{\abs}[1]{\lvert #1\rvert}

\begin{document}
\section*{Multiplying and dividing rational expressions 5}
\textbf{Question.} 

\textbf{Answer.} ([[NUMERSOL.toString()]])/([[DENOMSOL.toString()]])
                ([[NUMERSOL.toStringFactored()]])/([[DENOMSOL.toString()]])
                ([[NUMERSOL.toString()]])/([[DENOMSOL.toStringFactored()]])
                ([[NUMERSOL.toStringFactored()]])/([[DENOMSOL.toStringFactored()]])
            
            
                [[-B]]
                [[-A]]
                
                    [[X]] \textbackslash\{\}neq \textbackslash\{\}space
                    [[X]] \textbackslash\{\}neq \textbackslash\{\}space

\textbf{Hints.}
\begin{itemize}
  \item Dividing by an expression is the same as multiplying by its inverse.
                \textbackslash\{\}qquad
                    $\frac{[[NUMERATORS[ORDER[0]]]]}{[[DENOMINATORS[ORDER[1]]]]}$ $\times$
                    $\frac{[[NUMERATORS[1 - ORDER[0]]]]}{[[DENOMINATORS[1 - ORDER[1]]]]}$
  \item First factor out any common factors.
                \textbackslash\{\}qquad
                    $\frac{[[NUMERATORS[ORDER[0]].toStringFactored()]]}{[[DENOMINATORS[ORDER[1]].toStringFactored()]]}$ $\times$
                    $\frac{[[NUMERATORS[1 - ORDER[0]].toStringFactored()]]}{[[DENOMINATORS[1 - ORDER[1]].toStringFactored()]]}$
  \item Then factor the quadratic expressions.
                \textbackslash\{\}qquad \textbackslash\{\}dfrac
                    \{[[NUMERATORS[1].toStringFactored()]][[NUMER\_QUADRATIC]]\}
                    \{[[DENOMINATORS[1].toStringFactored()]][[DENOM\_QUADRATIC]]\}
                    $\times$ \textbackslash\{\}dfrac
                    \{[[NUMER\_QUADRATIC]][[NUMERATORS[1].toStringFactored()]]\}
                    \{[[DENOM\_QUADRATIC]][[DENOMINATORS[1].toStringFactored()]]\}
  \item Then multiply the two numerators and multiply the two denominators.
                \textbackslash\{\}qquad \textbackslash\{\}dfrac
                    \{[[NUMERATORS[1].toStringFactored(true)]] $\times$ [[NUMER\_QUADRATIC]]
                     [[NUMER\_QUADRATIC]] $\times$ [[NUMERATORS[1].toStringFactored(true)]]\}
                    \{[[DENOMINATORS[1].toStringFactored(true)]] $\times$ [[DENOM\_QUADRATIC]]
                     [[DENOM\_QUADRATIC]] $\times$ [[DENOMINATORS[1].toStringFactored(true)]]\}
                

                \textbackslash\{\}qquad = \textbackslash\{\}dfrac
                    \{[[getProduct(NUMER\_PRODUCT[0], NUMER\_PRODUCT[1])]]\}
                    \{[[getProduct(DENOM\_PRODUCT[0], DENOM\_PRODUCT[1])]]\}
  \item Notice that ([[TERM\_A]]) and ([[TERM\_B]]) appear in both the numerator and denominator so we can cancel them.
  \item \textbackslash\{\}qquad = \textbackslash\{\}dfrac
                    \{[[getProduct(NUMER\_PRODUCT[0], NUMER\_PRODUCT[1], CANCEL\_ORDER[0].slice(0, 1))]]\}
                    \{[[getProduct(DENOM\_PRODUCT[0], DENOM\_PRODUCT[1], CANCEL\_ORDER[1].slice(0, 1))]]\}
                

                We are dividing by [[TERM\_A]], so [[TERM\_A]] \textbackslash\{\}neq 0.
                Therefore, [[X]] \textbackslash\{\}neq [[-A]].
  \item \textbackslash\{\}qquad \textbackslash\{\}dfrac
                    \{[[getProduct(NUMER\_PRODUCT[0], NUMER\_PRODUCT[1], CANCEL\_ORDER[0])]]\}
                    \{[[getProduct(DENOM\_PRODUCT[0], DENOM\_PRODUCT[1], CANCEL\_ORDER[1])]]\}
                

                We are dividing by [[TERM\_B]], so [[TERM\_B]] \textbackslash\{\}neq 0.
                Therefore, [[X]] \textbackslash\{\}neq [[-B]].
  \item \textbackslash\{\}qquad \textbackslash\{\}dfrac
                    \{[[NUMERSOL.multiply(COMMON\_FACTOR).toStringFactored()]]\}
                    \{[[DENOMSOL.multiply(COMMON\_FACTOR).toStringFactored()]]\}
  \item $\frac{[[NUMERSOL.toStringFactored()]]}{[[DENOMSOL.toStringFactored()]]}$;
                [[X]] \textbackslash\{\}neq [[-A]]; [[X]] \textbackslash\{\}neq [[-B]]
\end{itemize}
\end{document}
