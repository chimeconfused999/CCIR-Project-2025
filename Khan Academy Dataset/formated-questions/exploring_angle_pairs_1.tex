% Auto-converted from khan-exercises
\documentclass{article}
\usepackage{amsmath,amssymb}
\usepackage[T1]{fontenc}
\usepackage{textcomp}
\newcommand{\abs}[1]{\lvert #1\rvert}

\begin{document}
\section*{Exploring angle pairs}
\textbf{Question.} Name an angle vertical to \textbackslash\{\}angle EGF.

\textbf{Answer.} BGC
                CGB

                
                    \textbackslash\{\}angle

\textbf{Hints.}
\begin{itemize}
  \item Vertical angles are formed at the intersection of two straight lines.
  \item Name the angle opposite \textbackslash\{\}red\{\textbackslash\{\}angle EGF\}.
  \item The angle vertical to \textbackslash\{\}red\{\textbackslash\{\}angle EGF\} is
                        \textbackslash\{\}green\{\textbackslash\{\}angle BGC\}.
  \item Adjacent angles share a ray and have a common vertex, but do not overlap.
  \item Name an angle that is next to \textbackslash\{\}red\{\textbackslash\{\}angle EGF\}.
  \item One angle adjacent to \textbackslash\{\}red\{\textbackslash\{\}angle EGF\} is
                        \textbackslash\{\}green\{\textbackslash\{\}angle EGD\}.
  \item A linear pair is two adjacent angles that form a straight angle.
  \item Name an angle that shares a ray with \textbackslash\{\}angle EGF
                    and will add to 180\textasciicircum{}\{\textbackslash\{\}circ\}
  \item One angle that forms a linear pair with \textbackslash\{\}red\{\textbackslash\{\}angle EGF\} is
                        \textbackslash\{\}green\{\textbackslash\{\}angle FGB\}.
  \item First, identify \textbackslash\{\}red\{\textbackslash\{\}angle EGF\}.
\end{itemize}
\end{document}
