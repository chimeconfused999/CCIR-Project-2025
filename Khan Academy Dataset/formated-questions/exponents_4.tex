% Auto-converted from khan-exercises
\documentclass{article}
\usepackage{amsmath,amssymb}
\usepackage[T1]{fontenc}
\usepackage{textcomp}
\newcommand{\abs}[1]{\lvert #1\rvert}

\begin{document}
\section*{Fractional exponents 2}
\textbf{Question.} \textbackslash\{\}Large\{[[fracParens( BASE\_N, BASE\_D )]]\textasciicircum{}\{[[fracSmall( ( EXP\_NEG ? -1 : 1 ) * EXP\_N, EXP\_D )]]\} = \{?\}\}

\textbf{Answer.} [[SOL\_N / SOL\_D]]

\textbf{Hints.}
\begin{itemize}
  \item = [[fracParens( BASEF\_N, BASEF\_D )]]\textasciicircum{}\{[[fracSmall( EXP\_N, EXP\_D )]]\}
  \item To simplify [[fracParens( BASEF\_N, BASEF\_D )]]\textasciicircum{}\{[[fracSmall( 1, EXP\_D )]]\}, figure out what goes in the blank: \textbackslash\{\}left(? \textbackslash\{\}right)\textasciicircum{}\{[[abs( EXP\_D )]]\}=[[frac( BASEF\_N, BASEF\_D )]]
  \item To simplify [[fracParens( BASEF\_N, BASEF\_D )]]\textasciicircum{}\{[[fracSmall( 1, EXP\_D )]]\}, figure out what goes in the blank: \textbackslash\{\}left(\textbackslash\{\}color\{blue\}\{[[frac( ROOT\_N, ROOT\_D )]]\}\textbackslash\{\}right)\textasciicircum{}\{[[abs( EXP\_D )]]\}=[[frac( BASEF\_N, BASEF\_D )]]
  \item so \textbackslash\{\}quad[[fracParens( BASEF\_N, BASEF\_D )]]\textasciicircum{}\{[[fracSmall( 1, EXP\_D )]]\}=[[frac( ROOT\_N, ROOT\_D )]]
\end{itemize}
\end{document}
