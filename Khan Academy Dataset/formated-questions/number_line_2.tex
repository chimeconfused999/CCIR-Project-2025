% Auto-converted from khan-exercises
\documentclass{article}
\usepackage{amsmath,amssymb}
\usepackage[T1]{fontenc}
\usepackage{textcomp}
\newcommand{\abs}[1]{\lvert #1\rvert}

\begin{document}
\section*{Negative numbers on the number line}
\textbf{Question.} Move the \textbackslash\{\}orange\{\textbackslash\{\}text\{orange dot\}\} to
                    \textbackslash\{\}orange\{16\} on the number line.

\textbf{Answer.} graph.movablePoint.coord[0]
                
                    if ( guess === 0 ) \{
                        return "";
                    \}
                    return abs( NUMBER - guess ) < 0.001;
                
                
                    graph.movablePoint.setCoord([guess, 0]);

\textbf{Hints.}
\begin{itemize}
  \item We know where 0 is on the number line because it is labeled.
  \item We know each tick mark represents 4 because -4 and 4 are labeled.
  \item Because 16 is positive, it will be to the right of 0.
  \item Each tick mark represents 4,
                        so 16 will be
                        4 tick marks
                        to the right of 0.
  \item The orange number shows where 16 is on the number line.
  \item We know where 0 is on the number line because it is labeled.
  \item We know each tick mark represents 4 because -4 and 4 are labeled.
  \item Because 16 is positive, it will be above 0.
  \item Each tick mark represents 4,
                        so 16 will be
                        4 tick marks
                        above 0.
  \item The orange number shows where 16 is on the number line.
  \item We know where 0 is on the number line because it is labeled.
  \item We know each tick mark represents 4 because -4 and 4 are labeled.
  \item Because the blue dot is to the right of 0, the number will be positive.
  \item The blue dot represents the number 16.
  \item We know where 0 is on the number line because it is labeled.
  \item We know each tick mark represents 4 because -4 and 4 are labeled.
  \item Because the blue dot is above 0, the number will be positive.
  \item The blue dot represents the number 16.
\end{itemize}
\end{document}
