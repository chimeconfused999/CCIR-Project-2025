% Auto-converted from khan-exercises
\documentclass{article}
\usepackage{amsmath,amssymb}
\usepackage[T1]{fontenc}
\usepackage{textcomp}
\newcommand{\abs}[1]{\lvert #1\rvert}

\begin{document}
\section*{Parallel lines 2}
\textbf{Question.} Solve for x:

\textbf{Answer.} [[SOLUTION]] \textasciicircum{}\textbackslash\{\}circ

\textbf{Hints.}
\begin{itemize}
  \item Corresponding angles are equal to one another. Watch this video to understand why.
  \item \textbackslash\{\}blue\{\textbackslash\{\}angle A\} and \textbackslash\{\}green\{\textbackslash\{\}angle B\} are corresponding angles.
                        Therefore, we can set them equal to one another.
                    
                    \textbackslash\{\}blue\{[[A]]x + [[B]]\textasciicircum{}\textbackslash\{\}circ\} = \textbackslash\{\}green\{[[C]]x + [[D]]\textasciicircum{}\textbackslash\{\}circ\}
  \item Subtract \textbackslash\{\}pink\{[[C]]x\} from both sides.
                    
                        ([[A]]x + [[B]]\textasciicircum{}\textbackslash\{\}circ) \textbackslash\{\}pink\{- [[C]]x\} =
                        ([[C]]x + [[D]]\textasciicircum{}\textbackslash\{\}circ) \textbackslash\{\}pink\{- [[C]]x\}
                    
                    [[A - C]]x + [[B]]\textasciicircum{}\textbackslash\{\}circ = [[D]]\textasciicircum{}\textbackslash\{\}circ
  \item Subtract \textbackslash\{\}pink\{[[abs(B)]]\textasciicircum{}\textbackslash\{\}circ\} from both sides.
                    Add \textbackslash\{\}pink\{[[abs(B)]]\textasciicircum{}\textbackslash\{\}circ\} to both sides.
                    
                        ([[A - C]]x + [[B]]\textasciicircum{}\textbackslash\{\}circ) \textbackslash\{\}pink\{+ [[-B]]\textasciicircum{}\textbackslash\{\}circ\} = [[D]]\textasciicircum{}\textbackslash\{\}circ \textbackslash\{\}pink\{+ [[-B]]\textasciicircum{}\textbackslash\{\}circ\}
                    
                    [[A - C]]x = [[D - B]]\textasciicircum{}\textbackslash\{\}circ
  \item Divide both sides by \textbackslash\{\}pink\{[[A - C]]\}.
                    $\frac{[[A - C]]x}{\pink{[[A - C]]}$\} = $\frac{[[D - B]]^\circ}{\pink{[[A - C]]}$\}
  \item Simplify.
                    x = [[SOLUTION]]\textasciicircum{}\textbackslash\{\}circ
  \item Alternate interior angles are equal to one another. Watch this video to understand why.
  \item \textbackslash\{\}blue\{\textbackslash\{\}angle A\} and \textbackslash\{\}green\{\textbackslash\{\}angle B\} are alternate interior angles.
                        Therefore, we can set them equal to one another.
                    
                    \textbackslash\{\}blue\{[[A]]x + [[B]]\textasciicircum{}\textbackslash\{\}circ\} = \textbackslash\{\}green\{[[C]]x + [[D]]\textasciicircum{}\textbackslash\{\}circ\}
  \item Subtract \textbackslash\{\}pink\{[[C]]x\} from both sides.
                    
                        ([[A]]x + [[B]]\textasciicircum{}\textbackslash\{\}circ) \textbackslash\{\}pink\{- [[C]]x\} =
                        ([[C]]x + [[D]]\textasciicircum{}\textbackslash\{\}circ) \textbackslash\{\}pink\{- [[C]]x\}
                    
                    [[A - C]]x + [[B]]\textasciicircum{}\textbackslash\{\}circ = [[D]]\textasciicircum{}\textbackslash\{\}circ
  \item Subtract \textbackslash\{\}pink\{[[abs(B)]]\textasciicircum{}\textbackslash\{\}circ\} from both sides.
                    Add \textbackslash\{\}pink\{[[abs(B)]]\textasciicircum{}\textbackslash\{\}circ\} to both sides.
                    
                        ([[A - C]]x + [[B]]\textasciicircum{}\textbackslash\{\}circ) \textbackslash\{\}pink\{+ [[-B]]\textasciicircum{}\textbackslash\{\}circ\} =
                        [[D]]\textasciicircum{}\textbackslash\{\}circ \textbackslash\{\}pink\{+ [[-B]]\textasciicircum{}\textbackslash\{\}circ\}
                    [[A - C]]x = [[D - B]]\textasciicircum{}\textbackslash\{\}circ
  \item Divide both sides by \textbackslash\{\}pink\{[[A - C]]\}.
                    $\frac{[[A - C]]x}{\pink{[[A - C]]}$\} = $\frac{[[D - B]]^\circ}{\pink{[[A - C]]}$\}
  \item Simplify.
                    x = [[SOLUTION]]\textasciicircum{}\textbackslash\{\}circ
  \item Alternate exterior angles are equal to one another. Watch this video to understand why.
  \item \textbackslash\{\}blue\{\textbackslash\{\}angle A\} and \textbackslash\{\}green\{\textbackslash\{\}angle B\} are alternate exterior angles.
                        Therefore, we can set them equal to one another.
                    
                    \textbackslash\{\}blue\{[[A]]x + [[B]]\textasciicircum{}\textbackslash\{\}circ\} = \textbackslash\{\}green\{[[C]]x + [[D]]\textasciicircum{}\textbackslash\{\}circ\}
  \item Subtract \textbackslash\{\}pink\{[[C]]x\} from both sides.
                        
                            ([[A]]x + [[B]]\textasciicircum{}\textbackslash\{\}circ) \textbackslash\{\}pink\{- [[C]]x\} =
                            ([[C]]x + [[D]]\textasciicircum{}\textbackslash\{\}circ) \textbackslash\{\}pink\{- [[C]]x\}
                        [[A - C]]x + [[B]]\textasciicircum{}\textbackslash\{\}circ = [[D]]\textasciicircum{}\textbackslash\{\}circ
                    
                    
                        Subtract \textbackslash\{\}pink\{[[abs(B)]]\textasciicircum{}\textbackslash\{\}circ\} from both sides.
                        Add \textbackslash\{\}pink\{[[abs(B)]]\textasciicircum{}\textbackslash\{\}circ\} to both sides.
                        
                            ([[A - C]]x + [[B]]\textasciicircum{}\textbackslash\{\}circ) \textbackslash\{\}pink\{+ [[-B]]\textasciicircum{}\textbackslash\{\}circ\} =
                            [[D]]\textasciicircum{}\textbackslash\{\}circ \textbackslash\{\}pink\{+ [[-B]]\textasciicircum{}\textbackslash\{\}circ\}
                        
                        [[A - C]]x = [[D - B]]\textasciicircum{}\textbackslash\{\}circ
                    
                    
                        Divide both sides by \textbackslash\{\}pink\{[[A - C]]\}.
                        $\frac{[[A - C]]x}{\pink{[[A - C]]}$\} = $\frac{[[D - B]]^\circ}{\pink{[[A - C]]}$\}
                    
                    
                        Simplify.
                        x = [[SOLUTION]]\textasciicircum{}\textbackslash\{\}circ
  \item Subtract \textbackslash\{\}pink\{[[A]]x\} from both sides.
                        ([[A]]x + [[B]]) \textbackslash\{\}pink\{- [[A]]x\} = ([[C]]x + [[D]]) \textbackslash\{\}pink\{- [[A]]x\}
                        [[B]] = [[C - A]]x + [[D]]
                    
                    
                        Subtract \textbackslash\{\}pink\{[[abs(D)]]\} from both sides.
                        Add \textbackslash\{\}pink\{[[abs(D)]]\} to both sides.
                        [[B]] \textbackslash\{\}pink\{+ [[-D]]\} = ([[C - A]]x + [[D]]) \textbackslash\{\}pink\{+ [[-D]]\}
                        [[B - D]] = [[C - A]]x
                    
                    
                        Divide both sides by \textbackslash\{\}pink\{[[C - A]]\}.
                        $\frac{[[B - D]]}{\pink{[[C - A]]}$\} = $\frac{[[C - A]]x}{\pink{[[C - A]]}$\}
                    
                    
                        Simplify.
                        [[SOLUTION]] = x
  \item The pink angles are adjacent to \textbackslash\{\}blue\{\textbackslash\{\}angle A\} and form a straight line, so we know that:
                    \textbackslash\{\}blue\{[[A]]x + [[B]]\textasciicircum{}\textbackslash\{\}circ\} + \textbackslash\{\}pink\{C\} = 180\textasciicircum{}\textbackslash\{\}circ
                    The pink angles equal each other because they are vertical angles.
  \item One of the pink angles corresponds with \textbackslash\{\}green\{\textbackslash\{\}angle B\},
                        and the other pink angle forms an alternative interior angle.
                        Therefore, \textbackslash\{\}green\{\textbackslash\{\}angle B\} equals the pink angle measure.
                    \textbackslash\{\}pink\{C\} = \textbackslash\{\}green\{[[C]]x + [[D]]\textasciicircum{}\textbackslash\{\}circ\}
  \item Substitute \textbackslash\{\}green\{[[C]]x + [[D]]\textasciicircum{}\textbackslash\{\}circ\} for \textbackslash\{\}pink\{C\} in our first equation.
                    \textbackslash\{\}blue\{[[A]]x + [[B]]\textasciicircum{}\textbackslash\{\}circ\} + \textbackslash\{\}green\{[[C]]x + [[D]]\textasciicircum{}\textbackslash\{\}circ\} = 180\textasciicircum{}\textbackslash\{\}circ
  \item Combine like terms.
                    [[A + C]]x + [[B + D]]\textasciicircum{}\textbackslash\{\}circ = 180\textasciicircum{}\textbackslash\{\}circ
  \item Subtract \textbackslash\{\}pink\{[[abs(B + D)]]\textasciicircum{}\textbackslash\{\}circ\} from both sides.
                    Add \textbackslash\{\}pink\{[[abs(B + D)]]\textasciicircum{}\textbackslash\{\}circ\} to both sides.
                    
                        ([[A + C]]x + [[B + D]]\textasciicircum{}\textbackslash\{\}circ) \textbackslash\{\}pink\{+ [[-(B + D)]]\textasciicircum{}\textbackslash\{\}circ\} =
                        180\textasciicircum{}\textbackslash\{\}circ \textbackslash\{\}pink\{+ [[-(B + D)]]\textasciicircum{}\textbackslash\{\}circ\}
                    
                    [[A + C]]x = [[180 - B - D]]\textasciicircum{}\textbackslash\{\}circ
  \item Divide by \textbackslash\{\}pink\{[[A + C]]\}.
                    $\frac{[[A + C]]x}{\pink{[[A + C]]}$\} = $\frac{[[180 - B - D]]^\circ}{\pink{[[A + C]]}$\}
  \item Simplify.
                    x = [[(180 - B - D) / (A + C)]]\textasciicircum{}\textbackslash\{\}circ
                    Note that the pinks angles are supplementary to \textbackslash\{\}blue\{\textbackslash\{\}angle A\}.
  \item The pink angles are adjacent to \textbackslash\{\}blue\{\textbackslash\{\}angle A\} and form a straight line, so we know that:
                    \textbackslash\{\}blue\{[[A]]x + [[B]]\textasciicircum{}\textbackslash\{\}circ\} + \textbackslash\{\}pink\{C\} = 180\textasciicircum{}\textbackslash\{\}circ
                    The pink angles equal each other because they are vertical angles.
  \item One of the pink angles corresponds with \textbackslash\{\}green\{\textbackslash\{\}angle B\},
                        and the other pink angle forms an alternative interior angle.
                        Therefore, \textbackslash\{\}green\{\textbackslash\{\}angle B\} equals the pink angle measure.
                    
                    \textbackslash\{\}pink\{C\} = \textbackslash\{\}green\{[[C]]x + [[D]]\textasciicircum{}\textbackslash\{\}circ\}
  \item Substitute \textbackslash\{\}green\{[[C]]x + [[D]]\textasciicircum{}\textbackslash\{\}circ\} for \textbackslash\{\}pink\{C\} in our first equation.
                    \textbackslash\{\}blue\{[[A]]x + [[B]]\textasciicircum{}\textbackslash\{\}circ\} + \textbackslash\{\}green\{[[C]]x + [[D]]\textasciicircum{}\textbackslash\{\}circ\} = 180\textasciicircum{}\textbackslash\{\}circ
  \item Combine like terms.
                    [[A + C]]x + [[B + D]]\textasciicircum{}\textbackslash\{\}circ = 180\textasciicircum{}\textbackslash\{\}circ
  \item Subtract \textbackslash\{\}pink\{[[abs(B + D)]]\textasciicircum{}\textbackslash\{\}circ\} from both sides.
                    Add \textbackslash\{\}pink\{[[abs(B + D)]]\textasciicircum{}\textbackslash\{\}circ\} to both sides.
                    
                        ([[A + C]]x + [[B + D]]\textasciicircum{}\textbackslash\{\}circ) \textbackslash\{\}pink\{+ [[-(B + D)]]\textasciicircum{}\textbackslash\{\}circ\} =
                        180\textasciicircum{}\textbackslash\{\}circ \textbackslash\{\}pink\{+ [[-(B + D)]]\textasciicircum{}\textbackslash\{\}circ\}
                    
                    [[A + C]]x = [[180 - B - D]]\textasciicircum{}\textbackslash\{\}circ
  \item Divide by \textbackslash\{\}pink\{[[A + C]]\}.
                    $\frac{[[A + C]]x}{\pink{[[A + C]]}$\} = $\frac{[[180 - B - D]]^\circ}{\pink{[[A + C]]}$\}
  \item Simplify.
                    x = [[(180 - B - D) / (A + C)]]\textasciicircum{}\textbackslash\{\}circ
                    Note that the pinks angles are supplementary to \textbackslash\{\}blue\{\textbackslash\{\}angle A\}.
\end{itemize}
\end{document}
