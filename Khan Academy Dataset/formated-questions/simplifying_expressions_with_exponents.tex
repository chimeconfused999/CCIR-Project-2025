% Auto-converted from khan-exercises
\documentclass{article}
\usepackage{amsmath,amssymb}
\usepackage[T1]{fontenc}
\usepackage{textcomp}
\newcommand{\abs}[1]{\lvert #1\rvert}

\begin{document}
\section*{Simplifying expressions with exponents}
\textbf{Question.} Rewrite \textbackslash\{\}large\{$\frac{^,^,[[randVar()]],3,4}{^,*,^,[[randVar()]],4,^,[[randVar()]],4,-4}$\} in the form
                    \textbackslash\{\}large\{[[randVar()]]\textasciicircum{}n[[randVar()]]\textasciicircum{}m\}.

\textbf{Answer.} [[randVar()]]\textasciicircum{}28 * [[randVar()]]\textasciicircum{}16
                    [[randVar()]]\textasciicircum{}28 * [[randVar()]]\textasciicircum{}16

\textbf{Hints.}
\begin{itemize}
  \item To start, simplify the numerator and the denominator independently.
  \item We can use the distributive property of exponents on the numerator.
                            
                                (\textbackslash\{\}blue\{\textasciicircum{},[[randVar()]],3\}\textbackslash\{\}green\{\textasciicircum{},[[randVar()]],-3\})\textasciicircum{}\{5\} = 
                                \textbackslash\{\}blue\{\textasciicircum{},\textasciicircum{},[[randVar()]],3,5\}\textbackslash\{\}green\{\textasciicircum{},\textasciicircum{},[[randVar()]],-3,5\}
                            
                        
                        
                            
                                \textbackslash\{\}blue\{\textasciicircum{},\textasciicircum{},[[randVar()]],3,5 = \textasciicircum{},[[randVar()]],12\}
                                \textbackslash\{\}green\{\textasciicircum{},\textasciicircum{},[[randVar()]],-3,5 = \textasciicircum{}, [[randVar()]], 12\}
                            
                            So, 
                                (\textbackslash\{\}blue\{\textasciicircum{},[[randVar()]],3\}\textbackslash\{\}green\{\textasciicircum{},[[randVar()]],-3\})\textasciicircum{}\{5\} = 
                                \textbackslash\{\}blue\{\textasciicircum{},[[randVar()]],12\}\textbackslash\{\}green\{\textasciicircum{}, [[randVar()]], 12\}.
  \item We can use the distributive property of exponents on the denominator.
                        
                            (\textbackslash\{\}blue\{\textasciicircum{},[[randVar()]],4\}\textbackslash\{\}green\{\textasciicircum{},[[randVar()]],4\})\textasciicircum{}\{-4\} = 
                            \textbackslash\{\}blue\{\textasciicircum{},\textasciicircum{},[[randVar()]],4,-4\}\textbackslash\{\}green\{\textasciicircum{},\textasciicircum{},[[randVar()]],4,-4\}
                        
                        
                            
                                \textbackslash\{\}blue\{\textasciicircum{},\textasciicircum{},[[randVar()]],4,-4 = \textasciicircum{},[[randVar()]],-16\}
                                \textbackslash\{\}green\{\textasciicircum{},\textasciicircum{},[[randVar()]],4,-4 = \textasciicircum{},[[randVar()]],-16\}
                            
                            So, 
                                (\textbackslash\{\}blue\{\textasciicircum{},[[randVar()]],4\}\textbackslash\{\}green\{\textasciicircum{},[[randVar()]],4\})\textasciicircum{}\{-4\} = 
                                \textbackslash\{\}blue\{\textasciicircum{},[[randVar()]],-16\}\textbackslash\{\}green\{\textasciicircum{},[[randVar()]],-16\}.
  \item Therefore, 
                            \textbackslash\{\}dfrac\{
                            
                                (\textbackslash\{\}blue\{\textasciicircum{},[[randVar()]],3\}\textbackslash\{\}green\{\textasciicircum{},[[randVar()]],-3\})\textasciicircum{}\{5\}
                            
                            
                                \textbackslash\{\}blue\{\textasciicircum{},[[randVar()]],3\}\textbackslash\{\}green\{\textasciicircum{},[[randVar()]],-3\}
                            
                            \}\{
                            
                                (\textbackslash\{\}blue\{\textasciicircum{},[[randVar()]],4\}\textbackslash\{\}green\{\textasciicircum{},[[randVar()]],4\})\textasciicircum{}\{-4\}
                            
                            
                                \textbackslash\{\}blue\{\textasciicircum{},[[randVar()]],4\}\textbackslash\{\}green\{\textasciicircum{},[[randVar()]],4\}
                            
                            \} = \textbackslash\{\}dfrac\{\textbackslash\{\}blue\{\textasciicircum{},[[randVar()]],12\}\textbackslash\{\}green\{\textasciicircum{}, [[randVar()]], 12\}\}\{
                            \textbackslash\{\}blue\{\textasciicircum{},[[randVar()]],-16\}\textbackslash\{\}green\{\textasciicircum{},[[randVar()]],-16\}\}.
  \item Break up the equation by variable and simplify.
                        
                            \textbackslash\{\}dfrac\{\textbackslash\{\}blue\{\textasciicircum{},[[randVar()]],12\}\textbackslash\{\}green\{\textasciicircum{}, [[randVar()]], 12\}\}\{\textbackslash\{\}blue\{\textasciicircum{},[[randVar()]],-16\}\textbackslash\{\}green\{\textasciicircum{},[[randVar()]],-16\}\} =
                            \textbackslash\{\}blue\{$\frac{^,[[randVar()]],12}{^,[[randVar()]],-16}$\} $\cdot$ \textbackslash\{\}green\{$\frac{^, [[randVar()]], 12}{^,[[randVar()]],-16}$\} =
                            \textbackslash\{\}blue\{[[randVar()]]\textasciicircum{}\{-15 - [[negParens(EXPDEN1 * EXPDEN3)]]\}\} $\cdot$
                            \textbackslash\{\}green\{[[randVar()]]\textasciicircum{}\{-10 - [[negParens(EXPDEN2 * EXPDEN3)]]\}\} = 
                            *,\textasciicircum{},[[randVar()]],28,\textasciicircum{},[[randVar()]],16
  \item To start, simplify the numerator and the denominator independently.
  \item We can use the distributive property of exponents on the numerator.
                        \textbackslash\{\}blue\{\textasciicircum{},\textasciicircum{},[[randVar()]],3,5\} = \textbackslash\{\}blue\{\textasciicircum{},[[randVar()]],12\}
  \item We can use the distributive property of exponents on the denominator.
                        
                            (\textbackslash\{\}blue\{\textasciicircum{},[[randVar()]],4\}\textbackslash\{\}green\{\textasciicircum{},[[randVar()]],4\})\textasciicircum{}\{-4\} = 
                            \textbackslash\{\}blue\{\textasciicircum{},\textasciicircum{},[[randVar()]],4,-4\}\textbackslash\{\}green\{\textasciicircum{},\textasciicircum{},[[randVar()]],4,-4\}
                        
                        
                            
                                \textbackslash\{\}blue\{\textasciicircum{},\textasciicircum{},[[randVar()]],4,-4 = \textasciicircum{},[[randVar()]],-16\}
                                \textbackslash\{\}green\{\textasciicircum{},\textasciicircum{},[[randVar()]],4,-4 = \textasciicircum{},[[randVar()]],-16\}
                            
                            So, 
                                (\textbackslash\{\}blue\{\textasciicircum{},[[randVar()]],4\}\textbackslash\{\}green\{\textasciicircum{},[[randVar()]],4\})\textasciicircum{}\{-4\} = 
                                \textbackslash\{\}blue\{\textasciicircum{},[[randVar()]],-16\}\textbackslash\{\}green\{\textasciicircum{},[[randVar()]],-16\}.
  \item Therefore, 
                            \textbackslash\{\}dfrac\{
                            (\textbackslash\{\}blue\{\textasciicircum{},[[randVar()]],3\})\textasciicircum{}\{5\}
                            \textbackslash\{\}blue\{\textasciicircum{},[[randVar()]],3\}
                            \}\{
                            
                                (\textbackslash\{\}blue\{\textasciicircum{},[[randVar()]],4\}\textbackslash\{\}green\{\textasciicircum{},[[randVar()]],4\})\textasciicircum{}\{-4\}
                            
                            
                                \textbackslash\{\}blue\{\textasciicircum{},[[randVar()]],4\}\textbackslash\{\}green\{\textasciicircum{},[[randVar()]],4\}
                            
                            \} = \textbackslash\{\}dfrac\{\textbackslash\{\}blue\{\textasciicircum{},[[randVar()]],12\}\}\{
                            \textbackslash\{\}blue\{\textasciicircum{},[[randVar()]],-16\}\textbackslash\{\}green\{\textasciicircum{},[[randVar()]],-16\}\}.
  \item Break up the equation by variable and simplify.
                        
                            \textbackslash\{\}dfrac\{\textbackslash\{\}blue\{\textasciicircum{},[[randVar()]],12\}\}\{\textbackslash\{\}blue\{\textasciicircum{},[[randVar()]],-16\}\textbackslash\{\}green\{\textasciicircum{},[[randVar()]],-16\}\} =
                            \textbackslash\{\}blue\{$\frac{^,[[randVar()]],12}{^,[[randVar()]],-16}$\} $\cdot$ \textbackslash\{\}green\{$\frac{1}{^,[[randVar()]],-16}$\} =
                            \textbackslash\{\}blue\{[[randVar()]]\textasciicircum{}\{-15 - [[negParens(EXPDEN1 * EXPDEN3)]]\}\} $\cdot$
                            \textbackslash\{\}green\{[[randVar()]]\textasciicircum{}\{ - [[negParens(EXPDEN2 * EXPDEN3)]]\}\} = 
                            *,\textasciicircum{},[[randVar()]],28,\textasciicircum{},[[randVar()]],16
\end{itemize}
\end{document}
