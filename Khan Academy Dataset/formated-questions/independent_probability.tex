% Auto-converted from khan-exercises
\documentclass{article}
\usepackage{amsmath,amssymb}
\usepackage[T1]{fontenc}
\usepackage{textcomp}
\newcommand{\abs}[1]{\lvert #1\rvert}

\begin{document}
\section*{Independent probability}
\textbf{Question.} If you flip a coin and roll a 6-sided die, what is the probability that you will flip a [[HT]] and roll [[RESULT\_DESC]]?

\textbf{Answer.} [[0.5 * RESULT\_POSSIBLE.length / 6]]

\textbf{Hints.}
\begin{itemize}
  \item Flipping a [[HT]] and rolling [[RESULT\_DESC]] are independent events: they don't affect each other. So, to get the
                    probability of both happening, we just need to multiply the probability of one by the probability of the other.
  \item The probability of flipping a [[HT]] is $\frac{1}{2}$.
  \item The probability of rolling [[RESULT\_DESC]] is $\frac{[[RESULT_POSSIBLE.length]]}{6}$, since there
                    is [[RESULT\_POSSIBLE.length]] outcome
                    which satisfy our condition
                    (namely, [[toSentence(RESULT\_POSSIBLE)]]), and 6 total possible outcomes.
                
                    The probability of rolling [[RESULT\_DESC]] is $\frac{[[RESULT_POSSIBLE.length]]}{6}$, since there
                    are [[RESULT\_POSSIBLE.length]] outcomes
                    which satisfy our condition
                    (namely, [[toSentence(RESULT\_POSSIBLE)]]), and 6 total possible outcomes.
  \item The probability of rolling [[RESULT\_DESC]] is $\frac{[[RESULT_POSSIBLE.length]]}{6}$, since there
                    is [[RESULT\_POSSIBLE.length]] outcome
                    which satisfies our condition
                    (namely, [[toSentence(RESULT\_POSSIBLE)]]), and 6 total possible outcomes.
                
                    The probability of rolling [[RESULT\_DESC]] is $\frac{[[RESULT_POSSIBLE.length]]}{6}$, since there
                    are [[RESULT\_POSSIBLE.length]] outcomes
                    which satisfies our condition
                    (namely, [[toSentence(RESULT\_POSSIBLE)]]), and 6 total possible outcomes.
  \item So, the probability of both these events happening is $\frac{1}{2}$ $\cdot$ $\frac{[[RESULT_POSSIBLE.length]]}{6}$
                    = $\frac{[[PRETTY_N]]}{[[PRETTY_D]]}$.
  \item We know that \textbackslash\{\}blue\{[[SINGLE\_PCT]] \textbackslash\{\}\%\} of the time, he'll make
                    his first shot.
                    
                    Then [[SINGLE\_PCT]] \textbackslash\{\}\% of
                    the time he makes his first shot, he will also make his second shot, and
                    [[localeToFixed((1-PROB)*100, 0)]] \textbackslash\{\}\% of the time he makes his
                    first shot, he will miss his second shot.
                    
                
                
                    Notice how we can completely ignore the rightmost section of the line now, because those were the times that
                    he missed the first free throw, and we only care about if he makes the first and the second.
                    So the chance of making two free throws in a row is [[SINGLE\_PCT]]\textbackslash\{\}\% of the times
                    that he made the first shot, which happens [[SINGLE\_PCT]]\textbackslash\{\}\% of the time in general.
                
                
                    This is [[SINGLE\_PCT]]\textbackslash\{\}\% $\cdot$ [[SINGLE\_PCT]]\textbackslash\{\}\%, or
                    [[localeToFixed(PROB, 2)]] $\cdot$ [[localeToFixed(PROB, 2)]] \textbackslash\{\}approx [[localeToFixed(PROB*PROB, 3)]].
                
                
                    
                        We can repeat this process again to get the probability of making three free throws in a row. We simply take
                        [[SINGLE\_PCT]]\textbackslash\{\}\% of probability that he makes two in a row, which we know from the previous step is
                        [[localeToFixed(PROB*PROB, 3)]] \textbackslash\{\}approx \textbackslash\{\}purple\{[[TWO\_PCT]]\textbackslash\{\}\%\}.
                    
                    
                    
                        [[SINGLE\_PCT]]\textbackslash\{\}\% of \textbackslash\{\}purple\{[[TWO\_PCT]]\textbackslash\{\}\%\} is
                        [[localeToFixed(PROB, 2)]] $\cdot$ [[localeToFixed(PROB * PROB, 3)]] \textbackslash\{\}approx 
                        [[localeToFixed(PROB\_CUBED, 3)]], or
                        about \textbackslash\{\}green\{[[THREE\_PCT]]\textbackslash\{\}\%\}:
                    
                    
                
                
                    There is a pattern here: the chance of making two free throws in a row was
                    [[localeToFixed(PROB, 2)]] $\cdot$ [[localeToFixed(PROB, 2)]], and the probability of making
                    three in a row was [[localeToFixed(PROB, 2)]] $\cdot$ [[localeToFixed(PROB * PROB, 3)]] =
                    [[localeToFixed(PROB, 2)]] $\cdot$ ([[localeToFixed(PROB, 2)]] $\cdot$ [[localeToFixed(PROB, 2)]])
                    = [[localeToFixed(PROB, 2)]]\textasciicircum{}3.
                
                
                    In general, you can continue in this way to find the probability of making any number of shots.
                
                
                    The probability of making [[STREAK]] free throws in a row is [[localeToFixed(PROB, 2)]] \textasciicircum{} [[STREAK]].
  \item We know that \textbackslash\{\}blue\{[[SINGLE\_PCT]] \textbackslash\{\}\%\} of the time, he'll miss
                    his first shot
                    (100 \textbackslash\{\}\% - [[localeToFixed(PR*100, 0)]] \textbackslash\{\}\% = [[SINGLE\_PCT]] \textbackslash\{\}\%).
                    
                    
                        Then [[SINGLE\_PCT]] \textbackslash\{\}\% of
                        the time he misses his first shot, he will also miss his second shot, and
                        [[localeToFixed((1-PROB)*100, 0)]] \textbackslash\{\}\% of the time he misses his
                        first shot, he will make his second shot.
                    
                    
                
                
                    Notice how we can completely ignore the rightmost section of the line now, because those were the times that
                    he made the first free throw, and we only care about if he misses the first and the second.
                    So the chance of missing two free throws in a row is [[SINGLE\_PCT]]\textbackslash\{\}\% of the times
                    that he missed the first shot, which happens [[SINGLE\_PCT]]\textbackslash\{\}\% of the time in general.
                
                
                    This is [[SINGLE\_PCT]]\textbackslash\{\}\% $\cdot$ [[SINGLE\_PCT]]\textbackslash\{\}\%, or
                    [[localeToFixed(PROB, 2)]] $\cdot$ [[localeToFixed(PROB, 2)]] \textbackslash\{\}approx
                    [[localeToFixed(PROB * PROB, 3)]].
                
                
                    
                        We can repeat this process again to get the probability of missing three free throws in a row. We simply take
                        [[SINGLE\_PCT]]\textbackslash\{\}\% of probability that he misses two in a row, which we know from the previous step is
                        [[localeToFixed(PROB*PROB, 3)]] \textbackslash\{\}approx \textbackslash\{\}purple\{[[TWO\_PCT]]\textbackslash\{\}\%\}.
                    
                    
                    [[SINGLE\_PCT]]\textbackslash\{\}\% of \textbackslash\{\}purple\{[[TWO\_PCT]]\textbackslash\{\}\%\} is
                    [[localeToFixed(PROB, 2)]] $\cdot$ [[localeToFixed(PROB*PROB, 3)]] \textbackslash\{\}approx [[localeToFixed(PROB\_CUBED, 3)]], or
                    about \textbackslash\{\}green\{[[THREE\_PCT]]\textbackslash\{\}\%\}:
                    
                
                
                    There is a pattern here: the chance of missing two free throws in a row was
                    [[localeToFixed(PROB, 2)]] $\cdot$ [[localeToFixed(PROB, 2)]], and the probability of missing
                    three in a row was [[localeToFixed(PROB, 2)]] $\cdot$ [[localeToFixed(PROB*PROB, 3)]] =
                    [[localeToFixed(PROB, 2)]] $\cdot$ ([[localeToFixed(PROB, 2)]] $\cdot$ [[localeToFixed(PROB, 2)]])
                    = [[localeToFixed(PROB, 2)]]\textasciicircum{}3.
                
                
                    In general, you can continue in this way to find the probability of missing any number of shots.
                
                
                    The probability of missing [[STREAK]] free throws in a row is
                    [[localeToFixed(PROB, 2)]] \textasciicircum{} [[STREAK]] = [[ANS]].
  \item If the Captain hits the pirate ship, it won't affect whether he's
                    also hit by the pirate's cannons (and vice-versa), because they both fired at the same time.
                    So, these events are independent.
  \item If the Captain hits the pirate ship, it won't affect whether she's
                    also hit by the pirate's cannons (and vice-versa), because they both fired at the same time.
                    So, these events are independent.
  \item Since they are independent, in order to get the probability that [[QUESTION]], we just need to multiply together
                    the probability that the captain hits and the probability that
                    the pirate hits.
                    
                        Since they are independent, in order to get the probability that [[QUESTION]], we just need to multiply together
                    the probability that the captain hits and the probability that
                    the pirate misses.
                    
                    
                        Since they are independent, in order to get the probability that [[QUESTION]], we just need to multiply together
                    the probability that the captain misses and the probability that
                    the pirate hits.
                    
                        Since they are independent, in order to get the probability that [[QUESTION]], we just need to multiply together
                    the probability that the captain misses and the probability that
                    the pirate misses.
  \item The probability that the Captain hits is [[C\_HIT\_PRETTY]].
  \item The probability that the Captain misses is 1 -  (the probability the Captain
                    hits), which is 1 - [[C\_HIT\_PRETTY]] = [[C\_MISS\_PRETTY]].
  \item The probability that the pirate hits is [[P\_HIT\_PRETTY]].
  \item The probability that the pirate misses is 1 -  (the probability the pirate
                    hits), which is 1 - [[P\_HIT\_PRETTY]] = [[P\_MISS\_PRETTY]].
  \item So, the probability that [[QUESTION]] is
                    [[C ? C\_HIT\_PRETTY : C\_MISS\_PRETTY]] $\cdot$ [[P ? P\_HIT\_PRETTY : P\_MISS\_PRETTY]] =
                    $\frac{[[ANS_N/getGCD(ANS_N,ANS_D)]]}{[[ANS_D/getGCD(ANS_N,ANS_D)]]}$.
\end{itemize}
\end{document}
