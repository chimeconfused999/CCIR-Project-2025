% Auto-converted from khan-exercises
\documentclass{article}
\usepackage{amsmath,amssymb}
\usepackage[T1]{fontenc}
\usepackage{textcomp}
\newcommand{\abs}[1]{\lvert #1\rvert}

\begin{document}
\section*{Solving quadratics by completing the square 2}
\textbf{Question.} Complete the square to solve for x.

\textbf{Answer.} [[B / 2]]
                        [[C * -1 + pow( B / 2, 2 )]]
                        [[X1]]
                        [[X2]]
                    
                    
                        [[B / 2]]
                        [[C * -1 + pow( B / 2, 2 )]]
                        [[X2]]
                        [[X1]]
                    
                    
                        
                            Completed Square: 
                            (x + \{\} )\textasciicircum{}2 = \{\}  
                            Solution: 
                            x = \{\}\textbackslash\{\}quad\textbackslash\{\}text\{[[OR]]\}\textbackslash\{\}quad x = \{\}

\textbf{Hints.}
\begin{itemize}
  \item First, divide the polynomial by [[MULT]], the coefficient of the x\textasciicircum{}2 term.
                x\textasciicircum{}2  + [[decimalFraction( B, 1, 1 )]][[B\_SIGN]]x + [[decimalFraction( C, 1, 1 )]] = 0
  \item Move the constant term to the right side of the equation.
                    x\textasciicircum{}2  + [[decimalFraction( B, 1, 1 )]][[B\_SIGN]]x = [[decimalFraction( C * -1, 1, 1 )]]
                
                
                    We complete the square by taking half of the coefficient of our x term, squaring it, and adding it to both sides of the equation. The coefficient of our x term is [[decimalFraction( B, 1, 1 )]], so half of it would be [[decimalFraction( B / 2, 1, 1 )]], and squaring it gives us \textbackslash\{\}color\{blue\}\{[[decimalFraction( pow( B / 2, 2 ), 1, 1 )]]\}.
                    x\textasciicircum{}2  + [[decimalFraction( B, 1, 1 )]][[B\_SIGN]]x \textbackslash\{\}color\{blue\}\{ + [[decimalFraction( pow( B / 2, 2 ), 1, 1 )]]\} = [[decimalFraction( C * -1, 1, 1 )]] \textbackslash\{\}color\{blue\}\{ + [[decimalFraction( pow( B / 2, 2 ), 1, 1 )]]\}
                
                
                    We can now rewrite the left side of the equation as a squared term.
                    ( x + [[decimalFraction( B / 2, 1, 1 )]] )\textasciicircum{}2 = [[decimalFraction( C * -1 + pow( B / 2, 2 ), 1, 1 )]]
  \item Note that the left side of the equation is already a perfect square trinomial. The coefficient of our x term is [[decimalFraction( B, 1, 1 )]], half of it is [[decimalFraction( B / 2, 1, 1 )]], and squaring it gives us \textbackslash\{\}color\{blue\}\{[[decimalFraction( pow( B / 2, 2 ), 1, 1 )]]\}, our constant term.
                
                    Thus, we can rewrite the left side of the equation as a squared term.
                    ( x + [[decimalFraction( B / 2, 1, 1 )]] )\textasciicircum{}2 = [[decimalFraction( C * -1 + pow( B / 2, 2 ), 1, 1 )]]
  \item Take the square root of both sides.
                x  + [[decimalFraction( B / 2, 1, 1 )]] = $\pm$[[decimalFraction( sqrt( C * -1 + pow( B / 2, 2 ) ), 1, 1 )]]
  \item Isolate x to find the solution(s).
                x = [[decimalFraction( -B / 2, 1, 1 )]]$\pm$[[decimalFraction( sqrt( C * -1 + pow( B / 2, 2 ) ), 1, 1 )]]
  \item The solutions are: x = [[decimalFraction( -B / 2 + sqrt( C * -1 + pow( B / 2, 2 ) ), 1, 1 )]] \textbackslash\{\}text\{ [[OR]] \} x = [[decimalFraction( -B / 2 - sqrt( C * -1 + pow( B / 2, 2 ) ), 1, 1 )]]
                
                
                    The solution is: x = [[decimalFraction( -B / 2 + sqrt( C * -1 + pow( B / 2, 2 ) ), 1, 1 )]]
                
                We already found the completed square: ( x + [[decimalFraction( B / 2, 1, 1 )]] )\textasciicircum{}2 = [[decimalFraction( C * -1 + pow( B / 2, 2 ), 1, 1 )]]
\end{itemize}
\end{document}
