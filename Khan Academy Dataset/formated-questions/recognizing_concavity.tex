% Auto-converted from khan-exercises
\documentclass{article}
\usepackage{amsmath,amssymb}
\usepackage[T1]{fontenc}
\usepackage{textcomp}
\newcommand{\abs}[1]{\lvert #1\rvert}

\begin{document}
\section*{Recognizing concavity}
\textbf{Question.} 

\textbf{Answer.} graph.slidingWindow.getX()
        
        
            var correct = \_.reduce(\_.range(guess, guess + 1, 0.02), function(correct, x) \{
                return correct \&\& PREDICATE(x);
            \}, true);
            if (!graph.moved \&\& !correct) \{
                return ""
            \}
            return correct;
        
        
            graph.slidingWindow.moveTo(guess, 0);

\textbf{Hints.}
\begin{itemize}
  \item The first derivative, f\textasciicircum{}\textbackslash\{\}prime(x), is greater
                    than 0 wherever the function is increasing.
  \item The intervals
                        where f(x) is increasing
                        are
                        highlighted above.
  \item The second derivative, f\textasciicircum{}\{\textbackslash\{\}prime\textbackslash\{\}prime\}(x), is greater
                    than 0 wherever the function is concave up.
  \item The intervals
                        where f(x) is concave up
                        are
                        highlighted above.
  \item Select any part of the function that is highlighted for
                        both conditions.
  \item The first derivative, f\textasciicircum{}\textbackslash\{\}prime(x), is greater
                    than 0 wherever the function is increasing.
  \item The intervals
                        where f(x) is increasing
                        are
                        highlighted above.
  \item The second derivative, f\textasciicircum{}\{\textbackslash\{\}prime\textbackslash\{\}prime\}(x), is less
                    than 0 wherever the function is concave down.
  \item The intervals
                        where f(x) is concave down
                        are
                        highlighted above.
  \item Select any part of the function that is highlighted for
                        both conditions.
  \item The first derivative, f\textasciicircum{}\textbackslash\{\}prime(x), is less
                    than 0 wherever the function is decreasing.
  \item The intervals
                        where f(x) is decreasing
                        are
                        highlighted above.
  \item The second derivative, f\textasciicircum{}\{\textbackslash\{\}prime\textbackslash\{\}prime\}(x), is greater
                    than 0 wherever the function is concave up.
  \item The intervals
                        where f(x) is concave up
                        are
                        highlighted above.
  \item Select any part of the function that is highlighted for
                        both conditions.
  \item The first derivative, f\textasciicircum{}\textbackslash\{\}prime(x), is less
                    than 0 wherever the function is decreasing.
  \item The intervals
                        where f(x) is decreasing
                        are
                        highlighted above.
  \item The second derivative, f\textasciicircum{}\{\textbackslash\{\}prime\textbackslash\{\}prime\}(x), is less
                    than 0 wherever the function is concave down.
  \item The intervals
                        where f(x) is concave down
                        are
                        highlighted above.
  \item Select any part of the function that is highlighted for
                        both conditions.
\end{itemize}
\end{document}
