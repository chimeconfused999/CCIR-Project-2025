% Auto-converted from khan-exercises
\documentclass{article}
\usepackage{amsmath,amssymb}
\usepackage[T1]{fontenc}
\usepackage{textcomp}
\newcommand{\abs}[1]{\lvert #1\rvert}

\begin{document}
\section*{Area of a circle}
\textbf{Question.} Suppose the radius of a circle is \textbackslash\{\}color\{[[R\_COLOR]]\}\{4\}. What is its area?

\textbf{Answer.} 50.26548245743669

\textbf{Hints.}
\begin{itemize}
  \item First, find the radius:
                    
                        \textbackslash\{\}begin\{align\}
                        r \&= \textbackslash\{\}dfrac d2 \textbackslash\{\}\textbackslash\{\}
                        r \&= $\frac{\color{[[D_COLOR]]}{8}$\}\{2\} \textbackslash\{\}\textbackslash\{\}
                        r \&= \textbackslash\{\}color\{[[R\_COLOR]]\}\{4\}
                        \textbackslash\{\}end\{align\}
  \item First, find the radius:
                    
                        \textbackslash\{\}begin\{align\}
                        r \&= $\frac{c}{2\pi}$ \textbackslash\{\}\textbackslash\{\}
                        r \&= $\frac{\color{[[C_COLOR]]}{8\pi}$\}\{2\} \textbackslash\{\}\textbackslash\{\}
                        r \&= \textbackslash\{\}color\{[[R\_COLOR]]\}\{4\}
                        \textbackslash\{\}end\{align\}
  \item The equation for the area of a circle is:
            K = \textbackslash\{\}pi r\textasciicircum{}2
  \item K = \textbackslash\{\}pi $\cdot$ \textbackslash\{\}color\{[[R\_COLOR]]\}\{4\}\textasciicircum{}2
  \item K  = \textbackslash\{\}color\{[[K\_COLOR]]\}\{16\textbackslash\{\}pi\}
\end{itemize}
\end{document}
