% Auto-converted from khan-exercises
\documentclass{article}
\usepackage{amsmath,amssymb}
\usepackage[T1]{fontenc}
\usepackage{textcomp}
\newcommand{\abs}[1]{\lvert #1\rvert}

\begin{document}
\section*{Solving quadratics by completing the square 1}
\textbf{Question.} Complete the square to solve for x.

\textbf{Answer.} [[B/2]]
                        [[C * -1 + pow( B / 2, 2 )]]
                        [[X1]]
                        [[X2]]
                    
                    
                        [[B/2]]
                        [[C * -1 + pow( B / 2, 2 )]]
                        [[X2]]
                        [[X1]]
                    
                    
                        
                            Completed Square: 
                            (x + \{\} )\textasciicircum{}2 = \{\}  
                            Solution: 
                            x = \{\}\textbackslash\{\}quad\textbackslash\{\}text\{[[OR]]\}\textbackslash\{\}quad x = \{\}

\textbf{Hints.}
\begin{itemize}
  \item Begin by moving the constant term to the right side of the equation.
                    x\textasciicircum{}2  + [[B]]x = [[C * -1]]
                
                
                    We complete the square by taking half of the coefficient of our x term, squaring it, and adding it to both sides of the equation. Since the coefficient of our x term is [[B]], half of it would be [[B / 2]], and squaring it gives us \textbackslash\{\}color\{blue\}\{[[pow( B / 2, 2 )]]\}.
                    x\textasciicircum{}2 + [[B]]x \textbackslash\{\}color\{blue\}\{ + [[pow( B / 2, 2 )]]\} = [[C * -1]] \textbackslash\{\}color\{blue\}\{ + [[pow( B / 2, 2 )]]\}
                
                
                    We can now rewrite the left side of the equation as a squared term.
                    ( x + [[B / 2]] )\textasciicircum{}2 = [[C * -1 + pow( B / 2, 2 )]]
  \item The left side of the equation is already a perfect square trinomial. The coefficient of our x term is [[B]], half of it is [[B / 2]], and squaring it gives us \textbackslash\{\}color\{blue\}\{[[pow( B / 2, 2 )]]\}, our constant term.
                
                    Thus, we can rewrite the left side of the equation as a squared term.
                    ( x + [[B / 2]] )\textasciicircum{}2 = [[C * -1 + pow( B / 2, 2 )]]
  \item Take the square root of both sides.
                x  + [[B / 2]] = $\pm$[[sqrt( C * -1 + pow( B / 2, 2 ) )]]
  \item Isolate x to find the solution(s).
                x = [[-B / 2]]$\pm$[[sqrt( C * -1 + pow( B / 2, 2 ) )]]
  \item So the solutions are: x = [[-B / 2 + sqrt( C * -1 + pow( B / 2, 2 ) )]] \textbackslash\{\}text\{ [[OR]] \} x = [[-B / 2 - sqrt( C * -1 + pow( B / 2, 2 ) )]]
                
                
                    The solution is: x = [[-B / 2 + sqrt( C * -1 + pow( B / 2, 2 ) )]]
                
                We already found the completed square: ( x + [[B / 2]] )\textasciicircum{}2 = [[C * -1 + pow( B / 2, 2 )]]
\end{itemize}
\end{document}
