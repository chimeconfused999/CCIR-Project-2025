% Auto-converted from khan-exercises
\documentclass{article}
\usepackage{amsmath,amssymb}
\usepackage[T1]{fontenc}
\usepackage{textcomp}
\newcommand{\abs}[1]{\lvert #1\rvert}

\begin{document}
\section*{Arithmetic series}
\textbf{Question.} The arithmetic sequence (a\_i) is defined by the formula:
                a\_i = [[A]] + [[D]](i - 1)
                
                    What is the sum of the first [[N]] terms in the series, starting with a\_1?

\textbf{Answer.} [[SUM]]

\textbf{Hints.}
\begin{itemize}
  \item The sum of an arithmetic series is the number of terms in the series times the average of the first and last terms.
  \item To find the sum of the first [[N]] terms,
                    we'll need the first and [[ordinalThrough20(N)]] terms of the series.
  \item a\_1 = [[A]] + [[D]] (1 - 1) = [[A]]
                    a\_\{[[N]]\} = [[A]] + [[D]] ([[N]] - 1) = [[A + D * (N - 1)]]
  \item Therefore, the sum of the first [[N]] terms is
                    \textbackslash\{\}qquad n\textbackslash\{\}left(\textbackslash\{\}dfrac\{a\_1 + a\_\{[[N]]\}\}\{2\}\textbackslash\{\}right) = [[N]] \textbackslash\{\}left($\frac{[[A]] + [[A + D * (N - 1)]]}{2}$\textbackslash\{\}right) = [[SUM]].
  \item First, let's find the explicit formula for the terms of the arithmetic series. We can see that the first term is [[A]] and the common difference is [[D]].
                    Thus, the explicit formula for this sequence is a\_i = [[A]] + [[D]](i - 1).
\end{itemize}
\end{document}
