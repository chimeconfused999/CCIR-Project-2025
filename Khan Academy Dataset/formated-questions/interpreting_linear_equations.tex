% Auto-converted from khan-exercises
\documentclass{article}
\usepackage{amsmath,amssymb}
\usepackage[T1]{fontenc}
\usepackage{textcomp}
\newcommand{\abs}[1]{\lvert #1\rvert}

\begin{document}
\section*{Interpreting linear relationships}
\textbf{Question.} How does Y change as X increases?
                    How does the cost of producing [[plural\_form(UNIT)]] change as the number of [[plural\_form(UNIT)]] increases?

\textbf{Answer.} Increases

\textbf{Hints.}
\begin{itemize}
  \item Looking at the graph, we see that as x increases (\textbackslash\{\}text\{[[BLACK\_ARROW]]\}), y also increases (\textbackslash\{\}green\{\textbackslash\{\}text\{[[GREEN\_ARROW]]\}\}).
  \item We can say that the slope of the line is positive, or that the variables have a direct relationship.
  \item Thus, as X increases, Y also increases.
  \item Thus, as the number of [[plural\_form(UNIT)]] increases, the price of [[plural\_form(UNIT)]] also increases.
  \item Looking at the graph, we see that as x increases (\textbackslash\{\}text\{[[BLACK\_ARROW]]\}), y decreases (\textbackslash\{\}red\{\textbackslash\{\}text\{[[RED\_ARROW]]\}\}).
  \item We can say that the slope of the line is negative, or that the variables have an inverse relationship.
  \item Thus, as X increases, Y decreases.
  \item Thus, as the number of [[plural\_form(UNIT)]] increases, the price of [[plural\_form(UNIT)]] decreases.
  \item Looking at the graph, we see that as x increases, there is no change in y.
  \item We can say that the slope of the line is zero, or that the variables have no correlation.
  \item Thus, as X increases, Y stays the same.
  \item Thus, as the number of [[plural\_form(UNIT)]] increases, the price of [[plural\_form(UNIT)]] stays the same.
\end{itemize}
\end{document}
