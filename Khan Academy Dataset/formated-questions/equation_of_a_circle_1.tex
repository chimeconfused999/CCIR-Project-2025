% Auto-converted from khan-exercises
\documentclass{article}
\usepackage{amsmath,amssymb}
\usepackage[T1]{fontenc}
\usepackage{textcomp}
\newcommand{\abs}[1]{\lvert #1\rvert}

\begin{document}
\section*{Equation of a circle in factored form}
\textbf{Question.} The equation of a circle C is +,\textasciicircum{},+,x,-1,2,\textasciicircum{},+,y,2,2 = 9.

                What are its center (h, k) and its radius r?

\textbf{Answer.} (h, k) = (1, -2)
                r = \{\}3

\textbf{Hints.}
\begin{itemize}
  \item The equation of a circle with center (h, k) and radius r is (x - h)\textasciicircum{}2 + (y - k)\textasciicircum{}2 = r\textasciicircum{}2.
  \item We can rewrite the given equation as (x - [[negParens(H)]])\textasciicircum{}2 + (y - [[negParens(K)]])\textasciicircum{}2 = 3\textasciicircum{}2.
  \item Thus, (h, k) = (1, -2) and r = 3.
\end{itemize}
\end{document}
