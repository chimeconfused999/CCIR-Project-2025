% Auto-converted from khan-exercises
\documentclass{article}
\usepackage{amsmath,amssymb}
\usepackage[T1]{fontenc}
\usepackage{textcomp}
\newcommand{\abs}[1]{\lvert #1\rvert}

\begin{document}
\section*{Multiplying and dividing rational expressions 4}
\textbf{Question.} 

\textbf{Answer.} ([[NUMERSOL.toString()]])/([[DENOMSOL.toString()]])
                ([[NUMERSOL.toStringFactored()]])/([[DENOMSOL.toString()]])
                ([[NUMERSOL.toString()]])/([[DENOMSOL.toStringFactored()]])
                ([[NUMERSOL.toStringFactored()]])/([[DENOMSOL.toStringFactored()]])
                [[NUMERSOL.toString()]]
                [[NUMERSOL.toStringFactored()]]
            

            
                [[-B]]
                [[SOL]]
                
                    [[X]] \textbackslash\{\}neq \textbackslash\{\}space
                    
                        [[X]] \textbackslash\{\}neq \textbackslash\{\}space

\textbf{Hints.}
\begin{itemize}
  \item Dividing by an expression is the same as multiplying by its inverse.
                \textbackslash\{\}qquad
                    $\frac{[[NUMERATORS[ORDER[0]]]]}{[[DENOMINATORS[ORDER[0]]]]}$ $\times$
                    $\frac{[[NUMERATORS[ORDER[1]]]]}{[[DENOMINATORS[ORDER[1]]]]}$
  \item First factor the quadratic.
  \item \textbackslash\{\}qquad
                $\frac{[[NUMER_STRINGS[ORDER[0]]]]}{[[DENOM_STRINGS[ORDER[0]]]]}$ $\times$
                $\frac{[[NUMER_STRINGS[ORDER[1]]]]}{[[DENOM_STRINGS[ORDER[1]]]]}$
  \item Then factor out any other terms.

                \textbackslash\{\}qquad
                    $\frac{[[NUMER_STRINGS2[ORDER[0]]]]}{[[DENOM_STRINGS2[ORDER[0]]]]}$ $\times$
                    $\frac{[[NUMER_STRINGS2[ORDER[1]]]]}{[[DENOM_STRINGS2[ORDER[1]]]]}$
  \item Then multiply the two numerators and multiply the two denominators.
  \item \textbackslash\{\}qquad \textbackslash\{\}dfrac\{
                ([[NUMER\_STRINGS2[ORDER[0]]]])
                [[NUMER\_STRINGS2[ORDER[0]]]] $\times$
                ([[NUMER\_STRINGS2[ORDER[1]]]])
                [[NUMER\_STRINGS2[ORDER[1]]]] \} \{
                ([[DENOM\_STRINGS2[ORDER[0]]]])
                [[DENOM\_STRINGS2[ORDER[0]]]] $\times$
                ([[DENOM\_STRINGS2[ORDER[1]]]])
                [[DENOM\_STRINGS2[ORDER[1]]]] \}
  \item \textbackslash\{\}qquad $\frac{
                [[getProduct(NUMER_PRODUCT[0], NUMER_PRODUCT[1])]]}{
                [[getProduct(DENOM_PRODUCT[0], DENOM_PRODUCT[1])]]}$
  \item Notice that
                
                    ([[CANCEL]]) and ([[TERM\_B]]) appear
                
                ([[TERM\_B]]) appears twice
                in both the numerator and denominator so we can cancel them.
  \item \textbackslash\{\}qquad $\frac{
                    [[getProduct(NUMER_PRODUCT[0], NUMER_PRODUCT[1], CANCEL_ORDER[0].slice(0, 1))]]}{
                    [[getProduct(DENOM_PRODUCT[0], DENOM_PRODUCT[1], CANCEL_ORDER[1].slice(0, 1))]]}$
                

                We are dividing by [[TERM\_B]], so [[TERM\_B]] \textbackslash\{\}neq 0.
                Therefore, [[X]] \textbackslash\{\}neq [[-B]].
  \item \textbackslash\{\}qquad $\frac{
                    [[getProduct(NUMER_PRODUCT[0], NUMER_PRODUCT[1], CANCEL_ORDER[0])]]}{
                    [[getProduct(DENOM_PRODUCT[0], DENOM_PRODUCT[1], CANCEL_ORDER[1])]]}$
                

                We are dividing by [[CANCEL]], so [[CANCEL]] \textbackslash\{\}neq 0.
                Therefore, [[X]] \textbackslash\{\}neq [[-CANCEL.terms[1]]].
  \item \textbackslash\{\}qquad
                    $\frac{[[NUMERSOL.multiply(COMMON_FACTOR).toStringFactored()]]}{[[DENOMSOL.multiply(COMMON_FACTOR).toStringFactored()]]}$
  \item \textbackslash\{\}qquad
                [[NUMERSOL.toStringFactored()]]
                [[NUMERSOL.multiply(-1).toStringFactored()]]
                $\frac{[[NUMERSOL.toStringFactored()]]}{[[DENOMSOL.toStringFactored()]]}$
                ; \textbackslash\{\}space [[X]] \textbackslash\{\}neq [[-B]]
                ; \textbackslash\{\}space [[X]] \textbackslash\{\}neq [[-CANCEL.terms[1]]]
\end{itemize}
\end{document}
