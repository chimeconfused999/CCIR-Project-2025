% Auto-converted from khan-exercises
\documentclass{article}
\usepackage{amsmath,amssymb}
\usepackage[T1]{fontenc}
\usepackage{textcomp}
\newcommand{\abs}[1]{\lvert #1\rvert}

\begin{document}
\section*{Identifying numerators and denominators}
\textbf{Question.} What is the numerator of the fraction $\frac{9}{5}$?

\textbf{Answer.} 9

\textbf{Hints.}
\begin{itemize}
  \item Fractions represent parts of a whole.
  \item The fraction $\frac{9}{5}$
                            could represent cutting pies into 5 slices and
                            taking 9 of those slices.
  \item The numerator is the number of slices we take, and it is written above the fraction line.
  \item So the numerator is 9.
  \item The denominator is the number of slices per pie, and it is written below the line.
  \item So the denominator is 5.
\end{itemize}
\end{document}
