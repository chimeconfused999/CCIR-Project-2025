% Auto-converted from khan-exercises
\documentclass{article}
\usepackage{amsmath,amssymb}
\usepackage[T1]{fontenc}
\usepackage{textcomp}
\newcommand{\abs}[1]{\lvert #1\rvert}

\begin{document}
\section*{Exploring mean and median}
\textbf{Question.} Arrange the 10 orange points on the
                    number line so the arithmetic mean
                    is 0 and the
                    median is
                    0.
                    The distance between adjacent tick marks is 1.

\textbf{Answer.} $.map(graph.points, function(el) {
                            return el.coord[0];
                        })
                    
                        if (roundTo(1, mean(guess)) === MEAN &&
                                roundTo(1, median(guess)) === MEDIAN) {
                            return true;
                        } else if (graph.moved) {
                            return false;
                        } else {
                            return "";
                        }
                    
                    
                        $.each(guess, function(i, x) \{
                            onMovePoint(graph.points[i], x, 0);
                        \});
                        updateMeanAndMedian();

\textbf{Hints.}
\begin{itemize}
  \item The median is the middle number. In other words there
                        are always as many points to the right of the median
                        as to the left.
  \item Try dragging the points so that half of them are to
                        the left of 
                        0 and half of them
                        are to the right of 
                        0.
                        
                            The two points in the middle should be the same
                            distance from 
                            0.
                        
                        
                            The middle point should be at
                            
                            0.
                        
                        
                            Show me an example
  \item As long as there are as many points to the left and to
                        the right of the median, the median will stay the same.
                        But the arithmetic mean is calculated using the value
                        of every point. Try moving the points on either side
                        of the median closer and further from the median to
                        see how the mean is affected.
  \item There are a number of different ways to arrange the
                        points so the mean is 
                        0 and the median is
                        
                        0.
                        
                            Show me an example
\end{itemize}
\end{document}
