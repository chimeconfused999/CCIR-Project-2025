% Auto-converted from khan-exercises
\documentclass{article}
\usepackage{amsmath,amssymb}
\usepackage[T1]{fontenc}
\usepackage{textcomp}
\newcommand{\abs}[1]{\lvert #1\rvert}

\begin{document}
\section*{Volume with unit cubes 2}
\textbf{Question.} What is the volume of this rectangular prism?

\textbf{Answer.} [[N * NUMERATOR3 / DENOMINATOR3]] 3

\textbf{Hints.}
\begin{itemize}
  \item The volume of a cube is equal to its side length cubed.
  \item The side length of one cube is $\frac{NaN}{NaN}$ [[i18n.\_("cubic ") + plural\_form(UNIT)]],
                    so the volume of one cube is \textbackslash\{\}left($\frac{NaN}{NaN}$\textbackslash\{\}right)\textasciicircum{}3 = 
                    NaN [[plural\_form(UNIT)]].
  \item The volume of a shape is measured by counting the number of cubic [[plural\_form(UNIT)]].
  \item Carefully count the cubes. Some of the cubes might be hidden behind other cubes. Try to visualize all of the cubes.
  \item There are a total of [[max(max(LENGTH, WIDTH), HEIGHT)]] cubes.
  \item The volume is [[max(max(LENGTH, WIDTH), HEIGHT)]] $\times$ NaN = 
                    NaN, 1 3.
  \item The volume of a cube is equal to its side length cubed.
  \item The side length of one cube is $\frac{NaN}{NaN}$ [[i18n.\_("cubic ") + plural\_form(UNIT)]],
                    so the volume of one cube is \textbackslash\{\}left($\frac{NaN}{NaN}$\textbackslash\{\}right)\textasciicircum{}3 = 
                    NaN [[plural\_form(UNIT)]].
  \item The volume of a shape is measured by counting the number of cubic [[plural\_form(UNIT)]].
  \item Carefully count the cubes. Some of the cubes might be hidden behind other cubes. Try to visualize all of the cubes.
  \item There are a total of [[max(max(LENGTH, WIDTH), HEIGHT)]] cubes.
  \item The volume is [[max(max(LENGTH, WIDTH), HEIGHT)]] $\times$ NaN = 
                    NaN, 1 3.
\end{itemize}
\end{document}
