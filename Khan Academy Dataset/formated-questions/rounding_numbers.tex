% Auto-converted from khan-exercises
\documentclass{article}
\usepackage{amsmath,amssymb}
\usepackage[T1]{fontenc}
\usepackage{textcomp}
\newcommand{\abs}[1]{\lvert #1\rvert}

\begin{document}
\section*{Rounding numbers}
\textbf{Question.} Round 85192.6743 to the nearest [[decimalPlaceNames[PLACE]]].

\textbf{Answer.} [[roundTo( PLACE, NUM )]]

\textbf{Hints.}
\begin{itemize}
  \item Because we want to round to the [[plural\_form(TPLACE, 2)]] place, we need to look at the digit in the [[plural\_form(decimalPlaceNames[PLACE + 1], 2)]] place.
  \item The digit in the [[plural\_form(decimalPlaceNames[PLACE + 1], 2)]] place is 9.
  \item Because 9 is more than 5, we round up to [[roundTo( PLACE, NUM )]].
  \item Because the [[plural\_form(decimalPlaceNames[PLACE + 1], 2)]] place digit is 9,
                    we round up to [[roundTo( PLACE, NUM )]].
  \item Because 9 is less than 5, we round down to [[roundTo( PLACE, NUM )]].
  \item Because we want to round to the [[plural\_form(TPLACE, 2)]] place, we need to look at the digit in the [[plural\_form(decimalPlaceNames[PLACE + 1], 2)]] place.
  \item The digit in the [[plural\_form(decimalPlaceNames[PLACE + 1], 2)]] place is 9.
  \item Because 9 is more than 5, we round up to [[roundTo( PLACE, NUM )]].
  \item Because the [[plural\_form(decimalPlaceNames[PLACE + 1], 2)]] place digit is 9,
                    we round up to [[roundTo( PLACE, NUM )]].
  \item Because 9 is less than 5, we round down to [[roundTo( PLACE, NUM )]].
\end{itemize}
\end{document}
