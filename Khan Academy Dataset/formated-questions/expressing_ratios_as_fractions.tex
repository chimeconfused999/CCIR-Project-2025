% Auto-converted from khan-exercises
\documentclass{article}
\usepackage{amsmath,amssymb}
\usepackage[T1]{fontenc}
\usepackage{textcomp}
\newcommand{\abs}[1]{\lvert #1\rvert}

\begin{document}
\section*{Expressing ratios as fractions}
\textbf{Question.} Alex has 16 [[plural\_form(fruit( 0 ), NUM)]] for every  6 [[plural\_form(fruit( 1 ), DENOM)]].
                Write the ratio of [[fruit( 0 )]]s to [[fruit( 1 )]]s as a simplified fraction.

\textbf{Answer.} 2.6666666666666665

\textbf{Hints.}
\begin{itemize}
  \item Ratios can be written in a few different ways that mean the same thing.
  \item You can express a ratio with a colon separating the two numbers.
            16:6
  \item You can write it out as a phrase like this.
            
            16 \textbackslash\{\}text\{ to \} 6
  \item Or, you can express a ratio as a fraction.
            $\frac{16}{6}$=8, 3
  \item Therefore, 8, 3 is the ratio of [[fruit( 0 )]]s to [[fruit( 1 )]]s written as a simplified fraction.
\end{itemize}
\end{document}
