% Auto-converted from khan-exercises
\documentclass{article}
\usepackage{amsmath,amssymb}
\usepackage[T1]{fontenc}
\usepackage{textcomp}
\newcommand{\abs}[1]{\lvert #1\rvert}

\begin{document}
\section*{Permutations and combinations}
\textbf{Question.} How many unique ways are there to arrange the letters in the word [[WORD]]?

\textbf{Answer.} [[ANSWER]]

\textbf{Hints.}
\begin{itemize}
  \item Let's try building the arrangements (or permutations) letter by letter. The word is [[WORD.length]]
                        letters long:
                        [[\_.map(\_.range(WORD.length), function(l)\{ return "\_ "; \}).join("")]]
                        Now, for the first blank, we have [[WORD.length]] choices of letters to put in.
  \item After we put in the first letter, let's say it's [[PERM[0]]], we have [[WORD.length-1]] blanks left.
                        [[PERM[0]+" "+\_.map(\_.range(WORD.length-1), function(l)\{ return "\_ "; \}).join("")]]
                        For the second blank, we only have [[WORD.length-1]] choices of letters left to put in. So far, there were
                            [[WORD.length]] $\cdot$ [[WORD.length-1]] unique choices we could have made.
  \item We can continue in this fashion to put in a third letter, then a fourth, and so on. At each step, we have one fewer choice to
                        make, until we get to the last letter, and there's only one we can put in.
  \item Using this method, the total number of arrangements is [[\_.map(\_.range(WORD.length).reverse(), function(l)\{ return (++l);\}).join("\textbackslash\{\}$\cdot$")]] = [[factorial(WORD.length)]]. Another way of writing this is [[WORD.length]]!,
                        or [[WORD.length]] factorial, but this isn't quite the right answer.
  \item Using the above method, we assumed that all the letters were unique. But they're not! There are [[REPTIMES]]
                        [[REPLETTER]]s, so we're counting every permutation multiple times. So every time we have these
                        [[factorial(REPTIMES)]] permutations:
                            
                                [[PERM\_UNIQUE.join("</br>")]]
                            
                        We actually should have only one permutation:
                            
                                [[PERM]]
  \item Notice that we've overcounted our arrangements by [[REPTIMES]]!. This is not a coincidence!
                        This is exactly the number of ways to permute [[REPTIMES]] objects, which we were doing with the
                        non-unique [[REPLETTER]]s. To address this overcounting, we need to divide the number of arrangements we counted
                        before by [[REPTIMES]]!.
  \item When we divide the number of permutations we got by the number of times we're overcounting each permutation, we get
                         $\frac{[[WORD.length]]!}{[[REPTIMES]]!}$ = $\frac{[[factorial(WORD.length)]]}{[[factorial(REPTIMES)]]}$ = [[ANSWER]]
  \item Forget about the reindeer that can't be together for a second, and let's try to
                        figure out how many ways we can arrange the reindeer if we don't have to worry about that.Forget about the reindeer that have to be together for a second, and let's try to
                        figure out how many ways we can arrange the reindeer if we don't have to worry about that.
                        We can build our line of reindeer one by one: there are [[NAMES.length]] slots,
                        and we have [[NAMES.length]] different reindeer we can put in the first slot.
  \item Once we fill the first slot, we only have [[NAMES.length-1]] reindeer left, so we only have
                        [[NAMES.length-1]]
                        choices for the second slot. So far, there are [[NAMES.length]] $\cdot$ [[NAMES.length-1]] = [[NAMES.length*(NAMES.length-1)]] unique choices we can make.
  \item We can continue in this way for the third reindeer, then the fourth, and so on, until we reach the last slot, where we only have
                        one reindeer left and so we can only make one choice.
  \item So, the total number of unique choices we could make to get to an arrangement of reindeer is [[\_.map(\_.range(NAMES.length).reverse(), function(l)\{ return (++l);\}).join("\textbackslash\{\}$\cdot$")]] = [[factorial(NAMES.length)]]. Another way of writing this is [[NAMES.length]]!,
                        or [[NAMES.length]] factorial. But we haven't thought about the two reindeer who can't be together yet.
  \item So, the total number of unique choices we could make to get to an arrangement of reindeer is [[\_.map(\_.range(NAMES.length).reverse(), function(l)\{ return (++l);\}).join("\textbackslash\{\}$\cdot$")]] = [[factorial(NAMES.length)]]. Another way of writing this is [[NAMES.length]]!,
                        or [[NAMES.length]] factorial. But we haven't thought about the two reindeer who have to be together yet.
  \item There are [[factorial(NAMES.length)]] different arrangements of reindeer altogether,
                            so we just need to subtract
                            all the arrangements where [[PAIR[0]]] and [[PAIR[1]]] are together. How many of these are there?
                        
                        
                            We can count the number of arrangements where [[PAIR[0]]] and [[PAIR[1]]] are together by treating them as one double-reindeer. Now we can use the same idea as before to come up with [[\_.map(\_.range(NAMES.length-1).reverse(), function(l)\{ return (++l);\}).join("\textbackslash\{\}$\cdot$")]] = [[factorial(NAMES.length-1)]] different arrangements. But that's not quite right.
                        
                        
                            Why? Because you can arrange the double-reindeer with [[PAIR[0]]] in front or with
                            [[PAIR[1]]] in front, and those are different arrangements! So the actual number of arrangements with [[PAIR[0]]]
                            and [[PAIR[1]]] together is [[factorial(NAMES.length-1)]] $\cdot$ 2 =
                            [[factorial(NAMES.length-1)*2]]
                        
                        
                            So, subtracting the number of arrangements where [[PAIR[0]]] and [[PAIR[1]]] are together from the total number
                            of arrangements, we get [[ANSWER]] arrangements of reindeer where they will fly.
  \item We can count the number of arrangements where [[PAIR[0]]] and [[PAIR[1]]] are together by treating them as one double-reindeer. Now we can use the same idea as before to come up with [[\_.map(\_.range(NAMES.length-1).reverse(), function(l)\{ return (++l);\}).join("\textbackslash\{\}$\cdot$")]] = [[factorial(NAMES.length-1)]] different arrangements. But that's not quite right.
                        
                        
                            Why? Because you can arrange the double-reindeer with [[PAIR[0]]] in front or with
                            [[PAIR[1]]] in front, and those are different arrangements! So the actual number of arrangements with [[PAIR[0]]]
                            and [[PAIR[1]]] together is [[factorial(NAMES.length-1)]] $\cdot$ 2 =
                            [[factorial(NAMES.length-1)*2]]
  \item There are [[FAC1\_TIMES]] numbers divisible by [[FAC1]] between 1
                            and 100, and [[FAC2\_TIMES]] numbers divisible by
                            [[FAC2]] between 1 and 100.
                            [Show me why]
                        
                        
                            
                                One way to see this is to take  $\frac{100}{[[FAC1]]}$ = [[FAC1\_TIMES]]
                            
                            
                                One way to see this is to take  $\frac{100}{[[FAC1]]}$, which is between [[FAC1\_TIMES]] and [[FAC1\_TIMES+1]]. So, starting at 0 and adding [[FAC1]]
                                at each step, you have to take between [[FAC1\_TIMES]] and [[FAC1\_TIMES+1]] steps to get
                                to 100. Since if you take a fraction of a step you won't get to a number divisible by [[FAC1]],
                                you'll hit [[FAC1\_TIMES]] numbers divisible by [[FAC1]] before you get to above 100.
                                So, there are [[FAC1\_TIMES]] numbers between 1 and 100 divisible by [[FAC1]].
                            
                            
                            
                                Using similar logic for the second factor, we see that  $\frac{100}{[[FAC2]]}$ = [[FAC2\_TIMES]]
                            
                            
                                Using similar logic for the second factor, take  $\frac{100}{[[FAC2]]}$, which is between [[FAC2\_TIMES]] and [[FAC2\_TIMES+1]]. So, starting at 0 and adding [[FAC2]]
                                at each step, you have to take between [[FAC2\_TIMES]] and [[FAC2\_TIMES+1]] steps to get
                                to 100. Since if you take a fraction of a step you won't get to a number divisible by [[FAC2]],
                                you'll hit [[FAC2\_TIMES]] numbers divisible by [[FAC2]] before you get to above 100.
                                So, there are [[FAC2\_TIMES]] numbers between 1 and 100 divisible by [[FAC2]].
  \item So, you might think there are [[FAC1\_TIMES]] + [[FAC2\_TIMES]] = [[FAC1\_TIMES+FAC2\_TIMES]]
                        numbers divisible by one or the other, but this is overcounting something.
  \item We're counting every number which is divisible by both [[FAC1]] and [[FAC2]] twice.
                        So, for example, [[FAC1*FAC2]] is counted once as a number divisible by [[FAC1]], and then again as a number divisible by [[FAC2]].
  \item So, we need to count how many numbers are divisible by both [[FAC1]] and [[FAC2]]
                        and subtract this from what we had before.
  \item Being divisible by both [[FAC1]] and [[FAC2]] is the same thing as being divisible by
                        [[FAC1*FAC2]], so there is [[BOTH\_TIMES]] number between 1 and 100 divisible by both.
  \item Being divisible by both [[FAC1]] and [[FAC2]] is the same thing as being divisible by
                        [[FAC1*FAC2]], so there are [[BOTH\_TIMES]] numbers between 1 and 100 divisible by both.
  \item Subtracting, there are [[FAC1\_TIMES + FAC2\_TIMES]] - [[BOTH\_TIMES]] = [[FAC1\_TIMES + FAC2\_TIMES - BOTH\_TIMES]] numbers divisible by [[FAC1]] or [[FAC2]].
\end{itemize}
\end{document}
