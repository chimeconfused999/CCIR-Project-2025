% Auto-converted from khan-exercises
\documentclass{article}
\usepackage{amsmath,amssymb}
\usepackage[T1]{fontenc}
\usepackage{textcomp}
\newcommand{\abs}[1]{\lvert #1\rvert}

\begin{document}
\section*{Equation of a circle in non-factored form}
\textbf{Question.} The equation of a circle C is
                    [[expr(["+",
                        "x\textasciicircum{}2",
                        "y\textasciicircum{}2",
                        D === 0 ? null : ["*", D, "x"],
                        E === 0 ? null : ["*", E, "y"],
                        F === 0 ? null : F
                    ])]] = 0.

                What is its center (h, k) and its radius r?

\textbf{Answer.} (h, k) = ([[H]], [[K]])
                r = \{\}[[R]]

\textbf{Hints.}
\begin{itemize}
  \item To find the equation in standard form, complete the square.
  \item ([[expr(["+", "x\textasciicircum{}2", ["*", D, "x"]])]]) + ([[expr(["+", "y\textasciicircum{}2", ["*", E, "y"]])]]) = [[-F]]
  \item ([[expr(["+", "x\textasciicircum{}2", ["*", D, "x"], H * H])]]) + ([[expr(["+", "y\textasciicircum{}2", ["*", E, "y"], K * K])]]) = [[-F]] + [[H * H]] + [[K * K]]
  \item [[X2T]] + [[Y2T]] = [[R * R]] = [[R]]\textasciicircum{}2
  \item Thus, (h, k) = ([[H]], [[K]]) and r = [[R]].
\end{itemize}
\end{document}
