% Auto-converted from khan-exercises
\documentclass{article}
\usepackage{amsmath,amssymb}
\usepackage[T1]{fontenc}
\usepackage{textcomp}
\newcommand{\abs}[1]{\lvert #1\rvert}

\begin{document}
\section*{Congruency postulates}
\textbf{Question.} 

\textbf{Answer.} Yes

\textbf{Hints.}
\begin{itemize}
  \item With these constraints, there are two ways to construct a triangle. See if you can find both ways.
  \item Both of these triangles have sides with lengths \textbackslash\{\}overline\{AB\} and \textbackslash\{\}overline\{BC\},
                            and an angle with the same measure as \textbackslash\{\}angle C.
                            However, angles \textbackslash\{\}angle A and \textbackslash\{\}angle B are not the same.
  \item So, yes, we can construct a triangle with the same angles as \textbackslash\{\}triangle ABC but different side lengths.
  \item With these constraints, there is more than one way to construct a triangle. See if you can find some different ways.
  \item Both of these triangles have the same three angles, but they have different side lengths:
  \item So, yes, we can construct a triangle different from \textbackslash\{\}triangle ABC with these constraints.
\end{itemize}
\end{document}
