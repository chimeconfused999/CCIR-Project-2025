% Auto-converted from khan-exercises
\documentclass{article}
\usepackage{amsmath,amssymb}
\usepackage[T1]{fontenc}
\usepackage{textcomp}
\newcommand{\abs}[1]{\lvert #1\rvert}

\begin{document}
\section*{Evaluating logarithms}
\textbf{Question.} \textbackslash\{\}large\{\textbackslash\{\}log\_\{3\}\}27 = \{?\}

\textbf{Answer.} 3

\textbf{Hints.}
\begin{itemize}
  \item If \textbackslash\{\}log\_\{b\}x = y, then b\textasciicircum{}y=x.
  \item First, try to write 27, the number we are taking the logarithm of, as a power of 3, the base of the logarithm.
  \item 27 can be expressed as [[get\_power\_string(number, base)]].
  \item 27 can be expressed as 3\textasciicircum{}3.
  \item 3\textasciicircum{}3=27, so \textbackslash\{\}log\_\{3\}27=3.
\end{itemize}
\end{document}
