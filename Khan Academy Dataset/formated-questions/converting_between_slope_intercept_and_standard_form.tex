% Auto-converted from khan-exercises
\documentclass{article}
\usepackage{amsmath,amssymb}
\usepackage[T1]{fontenc}
\usepackage{textcomp}
\newcommand{\abs}[1]{\lvert #1\rvert}

\begin{document}
\section*{Converting between slope-intercept and standard form}
\textbf{Question.} Convert the following equation from standard form to slope intercept form.

\textbf{Answer.} m = [[SLOPE]]
                b = [[Y\_INTERCEPT]]

\textbf{Hints.}
\begin{itemize}
  \item Move the x term to the other side of the equation.
                    [[expr([ "*", B, "y" ])]] = [[expr([ "*", -1 * A, "x"])]] + [[C]]
  \item Divide both sides by [[B]].
                    y = [[fractionReduce( -1 * A, B)]]-x + [[fractionReduce( C, B )]]
  \item Inspecting the equation in slope intercept form, we see the following.
                    \textbackslash\{\}begin\{align*\}m \&= [[fractionReduce( -1 * A, B)]]\textbackslash\{\}\textbackslash\{\}
                        b \&= [[fractionReduce( C, B )]]\textbackslash\{\}end\{align*\}
  \item Behold! The magic of math, that both equations could represent the same line!
  \item Move the x term to the same side as the y term.
                    [[coefficient(fractionReduce(A, B))]]x + y = [[fractionReduce(C, B)]]
  \item To get integers, multiply all the terms by [[B]].
                    
                        [[coefficient(A)]]x + 
                        [[B]]y = [[C]]
  \item Since the slope is 0 and there is no x term, the equation is already in slope intercept form.
                    y = [[Y\_INTERCEPT]]
  \item So we have \textbackslash\{\}blue\{[[A]]\}
                         lot of 
                         lots of 
                        x, \textbackslash\{\}green\{[[B]]\}
                         lot of 
                         lots of 
                        y, and a \textbackslash\{\}pink\{[[C]]\}.
                    
                    \textbackslash\{\}blue\{[[A]]\}x + \textbackslash\{\}green\{[[B]]\}y = \textbackslash\{\}pink\{[[C]]\}
  \item \textbackslash\{\}begin\{align*\}
                    \textbackslash\{\}blue\{A\} \&= \textbackslash\{\}blue\{[[A]]\}\textbackslash\{\}\textbackslash\{\}
                    \textbackslash\{\}green\{B\} \&= \textbackslash\{\}green\{[[B]]\}\textbackslash\{\}\textbackslash\{\}
                    \textbackslash\{\}pink\{C\} \&= \textbackslash\{\}pink\{[[C]]\}\textbackslash\{\}end\{align*\}
  \item Behold! The magic of math, that both equations could represent the same line!
\end{itemize}
\end{document}
