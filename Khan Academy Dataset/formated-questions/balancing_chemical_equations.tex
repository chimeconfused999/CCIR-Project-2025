% Auto-converted from khan-exercises
\documentclass{article}
\usepackage{amsmath,amssymb}
\usepackage[T1]{fontenc}
\usepackage{textcomp}
\newcommand{\abs}[1]{\lvert #1\rvert}

\begin{document}
\section*{Balancing chemical equations}
\textbf{Question.} Balance the following chemical equation:

\textbf{Answer.} \textbackslash\{\}qquad
                1\textbackslash\{\}text\{H\}\_2 +
                3\textbackslash\{\}text\{O\}\_2 \textbackslash\{\}rightarrow
                2\textbackslash\{\}text\{H\}\_2\textbackslash\{\}text\{O\}

\textbf{Hints.}
\begin{itemize}
  \item There are 2 \textbackslash\{\}text\{ O\} on the left and only
                        1 on the right, so multiply
                        \textbackslash\{\}text\{H\}\_2\textbackslash\{\}text\{O\} by \textbackslash\{\}blue\{2\}.
                    
                    \textbackslash\{\}qquad
                        \textbackslash\{\}text\{H\}\_2 + \textbackslash\{\}text\{O\}\_2 \textbackslash\{\}rightarrow \textbackslash\{\}blue\{2\}\textbackslash\{\}text\{H\}\_2\textbackslash\{\}text\{O\}
  \item That gives us 4 \textbackslash\{\}text\{ H\} on the right and
                        only 2 on the left, so multiply
                        \textbackslash\{\}text\{H\}\_2 by \textbackslash\{\}red\{2\}.
                    
                    \textbackslash\{\}qquad
                        \textbackslash\{\}red\{2\}\textbackslash\{\}text\{H\}\_2 + \textbackslash\{\}text\{O\}\_2 \textbackslash\{\}rightarrow 2\textbackslash\{\}text\{H\}\_2\textbackslash\{\}text\{O\}
  \item The balanced equation is:
                    \textbackslash\{\}qquad
                        2\textbackslash\{\}text\{H\}\_2 + \textbackslash\{\}text\{O\}\_2 \textbackslash\{\}rightarrow 2\textbackslash\{\}text\{H\}\_2\textbackslash\{\}text\{O\}
  \item \textbackslash\{\}text\{C\} is already balanced.
  \item There are 4 \textbackslash\{\}text\{ H\} on the left and only
                        1 on the right, so multiply
                        \textbackslash\{\}text\{HCl\} by \textbackslash\{\}blue\{4\}.
                    
                    \textbackslash\{\}qquad
                        \textbackslash\{\}text\{CH\}\_4 + \textbackslash\{\}text\{Cl\}\_2 \textbackslash\{\}rightarrow \textbackslash\{\}text\{CCl\}\_4 + \textbackslash\{\}blue\{4\}\textbackslash\{\}text\{HCl\}
  \item That gives us 8 \textbackslash\{\}text\{ Cl\} on the right and
                        only 2 on the left, so multiply
                        \textbackslash\{\}text\{Cl\}\_2 by \textbackslash\{\}red\{4\}.
                    
                    \textbackslash\{\}qquad
                        \textbackslash\{\}text\{CH\}\_4 + \textbackslash\{\}red\{4\}\textbackslash\{\}text\{Cl\}\_2 \textbackslash\{\}rightarrow \textbackslash\{\}text\{CCl\}\_4 + 4\textbackslash\{\}text\{HCl\}
  \item The balanced equation is:
                    \textbackslash\{\}qquad
                        \textbackslash\{\}text\{CH\}\_4 + 4\textbackslash\{\}text\{Cl\}\_2 \textbackslash\{\}rightarrow \textbackslash\{\}text\{CCl\}\_4 + 4\textbackslash\{\}text\{HCl\}
  \item There are 2 \textbackslash\{\}text\{ O\} on the left and
                        3 on the right. The lowest common denominator
                        is 6, so multiply
                        \textbackslash\{\}text\{O\}\_2 by \textbackslash\{\}blue\{3\} and
                        \textbackslash\{\}text\{Al\}\_2\textbackslash\{\}text\{O\}\_3 by \textbackslash\{\}red\{2\}.
                    
                    \textbackslash\{\}qquad
                        \textbackslash\{\}text\{Al\} + \textbackslash\{\}blue\{3\}\textbackslash\{\}text\{O\}\_2 \textbackslash\{\}rightarrow \textbackslash\{\}red\{2\}\textbackslash\{\}text\{Al\}\_2\textbackslash\{\}text\{O\}\_3
  \item That gives us 4 \textbackslash\{\}text\{ Al\} on the right and
                        only 1 on the left, so multiply
                        \textbackslash\{\}text\{Al\} by \textbackslash\{\}pink\{4\}.
                    
                    \textbackslash\{\}qquad
                        \textbackslash\{\}pink\{4\}\textbackslash\{\}text\{Al\} + 3\textbackslash\{\}text\{O\}\_2 \textbackslash\{\}rightarrow 2\textbackslash\{\}text\{Al\}\_2\textbackslash\{\}text\{O\}\_3
  \item The balanced equation is:
                    \textbackslash\{\}qquad
                        4\textbackslash\{\}text\{Al\} + 3\textbackslash\{\}text\{O\}\_2 \textbackslash\{\}rightarrow 2\textbackslash\{\}text\{Al\}\_2\textbackslash\{\}text\{O\}\_3
  \item \textbackslash\{\}text\{C\} is already balanced.
  \item There are 4 \textbackslash\{\}text\{ H\} on the left and
                        2 on the right, so multiply
                        \textbackslash\{\}text\{H\}\_2\textbackslash\{\}text\{O\} by \textbackslash\{\}blue\{2\}.
                    
                    \textbackslash\{\}qquad
                        \textbackslash\{\}text\{CH\}\_4 + \textbackslash\{\}text\{O\}\_2 \textbackslash\{\}rightarrow \textbackslash\{\}text\{CO\}\_2 + \textbackslash\{\}blue\{2\}\textbackslash\{\}text\{H\}\_2\textbackslash\{\}text\{O\}
  \item That gives us 4 \textbackslash\{\}text\{ O\} on the right and
                        only 2 on the left, so multiply
                        \textbackslash\{\}text\{O\}\_2 by \textbackslash\{\}red\{2\}.
                        (Since oxygen is by itself on the left, it should be done
                        at the end because you can give it a coefficient without
                        affecting another element.)
                    
                    \textbackslash\{\}qquad
                        \textbackslash\{\}text\{CH\}\_4 + \textbackslash\{\}red\{2\}\textbackslash\{\}text\{O\}\_2 \textbackslash\{\}rightarrow \textbackslash\{\}text\{CO\}\_2 + 2\textbackslash\{\}text\{H\}\_2\textbackslash\{\}text\{O\}
  \item The balanced equation is:
                    \textbackslash\{\}qquad
                        \textbackslash\{\}text\{CH\}\_4 + 2\textbackslash\{\}text\{O\}\_2 \textbackslash\{\}rightarrow \textbackslash\{\}text\{CO\}\_2 + 2\textbackslash\{\}text\{H\}\_2\textbackslash\{\}text\{O\}
  \item There is 1 \textbackslash\{\}text\{ Br\} on the left and
                        2 on the right, so multiply
                        \textbackslash\{\}text\{NaBr\} by \textbackslash\{\}blue\{2\}.
                    
                    \textbackslash\{\}qquad
                        \textbackslash\{\}blue\{2\}\textbackslash\{\}text\{NaBr\} + \textbackslash\{\}text\{Cl\}\_2 \textbackslash\{\}rightarrow \textbackslash\{\}text\{NaCl\} + \textbackslash\{\}text\{Br\}\_2
  \item There are 2 \textbackslash\{\}text\{ Cl\} on the left and
                        1 on the right, so multiply
                        \textbackslash\{\}text\{NaCl\} by \textbackslash\{\}red\{2\}.
                    
                    \textbackslash\{\}qquad
                        2\textbackslash\{\}text\{NaBr\} + \textbackslash\{\}text\{Cl\}\_2 \textbackslash\{\}rightarrow \textbackslash\{\}red\{2\}\textbackslash\{\}text\{NaCl\} + \textbackslash\{\}text\{Br\}\_2
  \item Now \textbackslash\{\}text\{Na\} is balanced again.
  \item The balanced equation is:
                    \textbackslash\{\}qquad
                        2\textbackslash\{\}text\{NaBr\} + \textbackslash\{\}text\{Cl\}\_2 \textbackslash\{\}rightarrow 2\textbackslash\{\}text\{NaCl\} + \textbackslash\{\}text\{Br\}\_2
  \item There are 2 \textbackslash\{\}text\{ Cl\} on the right and
                        only 1 on the left, so multiply
                        \textbackslash\{\}text\{HCl\} by \textbackslash\{\}blue\{2\}.
                    
                    \textbackslash\{\}qquad
                        \textbackslash\{\}text\{Mg\} + \textbackslash\{\}blue\{2\}\textbackslash\{\}text\{HCl\} \textbackslash\{\}rightarrow \textbackslash\{\}text\{MgCl\}\_2 + \textbackslash\{\}text\{H\}\_2
  \item Now all atoms are balanced; the balanced equation is:
                    \textbackslash\{\}qquad
                        \textbackslash\{\}text\{Mg\} + 2\textbackslash\{\}text\{HCl\} \textbackslash\{\}rightarrow \textbackslash\{\}text\{MgCl\}\_2 + \textbackslash\{\}text\{H\}\_2
  \item Start with the compound that has the most elements
                    (\textbackslash\{\}text\{NH\}\_4\textbackslash\{\}text\{NO\}\_3).
  \item There are 2 \textbackslash\{\}text\{N\} on the left and
                    2 \textbackslash\{\}text\{N\} on the right, so \textbackslash\{\}text\{N\}
                    is already balanced.
  \item There are 4 \textbackslash\{\}text\{ H\} on the left and
                        2 on the right, so multiply
                        \textbackslash\{\}text\{H\}\_2\textbackslash\{\}text\{O\} by \textbackslash\{\}blue\{2\}.
                    
                    \textbackslash\{\}qquad
                        \textbackslash\{\}text\{NH\}\_4\textbackslash\{\}text\{NO\}\_3 \textbackslash\{\}rightarrow \textbackslash\{\}text\{N\}\_2 + \textbackslash\{\}text\{O\}\_2 + \textbackslash\{\}blue\{2\}\textbackslash\{\}text\{H\}\_2\textbackslash\{\}text\{O\}
  \item That gives us 3 \textbackslash\{\}text\{ O\} on the left and
                        4 on the right. If we try giving
                        \textbackslash\{\}text\{O\}\_2 a coefficient of \textbackslash\{\}red\{$\frac{1}{2}$\},
                        it gives us 3 \textbackslash\{\}text\{ O\} on both sides.
                    
                    \textbackslash\{\}qquad
                        \textbackslash\{\}text\{NH\}\_4\textbackslash\{\}text\{NO\}\_3 \textbackslash\{\}rightarrow \textbackslash\{\}text\{N\}\_2 + \textbackslash\{\}red\{$\frac{1}{2}$\}\textbackslash\{\}text\{O\}\_2 + 2\textbackslash\{\}text\{H\}\_2\textbackslash\{\}text\{O\}
  \item Since fractions are not usually used as coefficients, multiply everything by 2 to get rid of the fraction.
                    
                    \textbackslash\{\}qquad
                        2\textbackslash\{\}text\{NH\}\_4\textbackslash\{\}text\{NO\}\_3 \textbackslash\{\}rightarrow 2\textbackslash\{\}text\{N\}\_2 + 1\textbackslash\{\}text\{O\}\_2 + 4\textbackslash\{\}text\{H\}\_2\textbackslash\{\}text\{O\}
  \item The balanced equation is:
                    \textbackslash\{\}qquad
                        2\textbackslash\{\}text\{NH\}\_4\textbackslash\{\}text\{NO\}\_3 \textbackslash\{\}rightarrow 2\textbackslash\{\}text\{N\}\_2 + \textbackslash\{\}text\{O\}\_2 + 4\textbackslash\{\}text\{H\}\_2\textbackslash\{\}text\{O\}
  \item For a combustion reaction, it is usually easiest to start
                    with \textbackslash\{\}text\{C\}.
  \item There are 2 \textbackslash\{\}text\{ C\} on the left and
                        1 on the right, so multiply
                        \textbackslash\{\}text\{CO\}\_2 by \textbackslash\{\}blue\{2\}.
                    
                    \textbackslash\{\}qquad
                        \textbackslash\{\}text\{C\}\_2\textbackslash\{\}text\{H\}\_6\textbackslash\{\}text\{O\} + \textbackslash\{\}text\{O\}\_2 \textbackslash\{\}rightarrow \textbackslash\{\}blue\{2\}\textbackslash\{\}text\{CO\}\_2 + \textbackslash\{\}text\{H\}\_2\textbackslash\{\}text\{O\}
  \item There are 6 \textbackslash\{\}text\{ H\} on the left and
                        2 on the right, so multiply
                        \textbackslash\{\}text\{H\}\_2\textbackslash\{\}text\{O\} by \textbackslash\{\}red\{3\}.
                    
                    \textbackslash\{\}qquad
                        \textbackslash\{\}text\{C\}\_2\textbackslash\{\}text\{H\}\_6\textbackslash\{\}text\{O\} + \textbackslash\{\}text\{O\}\_2 \textbackslash\{\}rightarrow 2\textbackslash\{\}text\{CO\}\_2 + \textbackslash\{\}red\{3\}\textbackslash\{\}text\{H\}\_2\textbackslash\{\}text\{O\}
  \item That gives us 7 \textbackslash\{\}text\{ O\} on the right and
                        3 on the left, so multiply
                        \textbackslash\{\}text\{O\}\_2 by \textbackslash\{\}pink\{3\}.
                    
                    \textbackslash\{\}qquad
                        \textbackslash\{\}text\{C\}\_2\textbackslash\{\}text\{H\}\_6\textbackslash\{\}text\{O\} + \textbackslash\{\}pink\{3\}\textbackslash\{\}text\{O\}\_2 \textbackslash\{\}rightarrow 2\textbackslash\{\}text\{CO\}\_2 + 3\textbackslash\{\}text\{H\}\_2\textbackslash\{\}text\{O\}
  \item The balanced equation is:
                    \textbackslash\{\}qquad
                        \textbackslash\{\}text\{C\}\_2\textbackslash\{\}text\{H\}\_6\textbackslash\{\}text\{O\} + 3\textbackslash\{\}text\{O\}\_2 \textbackslash\{\}rightarrow 2\textbackslash\{\}text\{CO\}\_2 + 3\textbackslash\{\}text\{H\}\_2\textbackslash\{\}text\{O\}
  \item There are 2 \textbackslash\{\}text\{ O\} on the left and only
                        1 on the right, so multiply
                        \textbackslash\{\}text\{MgO\} by \textbackslash\{\}blue\{2\}.
                    
                    \textbackslash\{\}qquad
                        \textbackslash\{\}text\{Mg\} + \textbackslash\{\}text\{O\}\_2 \textbackslash\{\}rightarrow \textbackslash\{\}blue\{2\}\textbackslash\{\}text\{MgO\}
  \item That gives us 2 \textbackslash\{\}text\{ Mg\} on the right and
                        only 1 on the left, so multiply
                        \textbackslash\{\}text\{Mg\} by \textbackslash\{\}red\{2\}.
                    
                    \textbackslash\{\}qquad
                        \textbackslash\{\}red\{2\}\textbackslash\{\}text\{Mg\} + \textbackslash\{\}text\{O\}\_2 \textbackslash\{\}rightarrow 2\textbackslash\{\}text\{MgO\}
  \item The balanced equation is:
                    \textbackslash\{\}qquad
                        2\textbackslash\{\}text\{Mg\} + \textbackslash\{\}text\{O\}\_2 \textbackslash\{\}rightarrow 2\textbackslash\{\}text\{MgO\}
  \item There is 1 \textbackslash\{\}text\{ H\} and 1 \textbackslash\{\}text\{ Cl\} on the left and
                        3 \textbackslash\{\}text\{ Cl\} and 2 \textbackslash\{\}text\{ H\} on the right.
                        The lowest common denominator for the right is 6, so multiply
                        \textbackslash\{\}text\{AlCl\}\_3 by \textbackslash\{\}blue\{2\} and
                        \textbackslash\{\}text\{H\}\_2 by \textbackslash\{\}red\{3\}.
                    
                    \textbackslash\{\}qquad
                        \textbackslash\{\}text\{Al\} + \textbackslash\{\}text\{HCl\} \textbackslash\{\}rightarrow \textbackslash\{\}blue\{2\}\textbackslash\{\}text\{AlCl\}\_3 + \textbackslash\{\}red\{3\}\textbackslash\{\}text\{H\}\_2
  \item That gives us 6 \textbackslash\{\}text\{ H\} and 6 \textbackslash\{\}text\{ Cl\}
                        on the right, so multiply
                        \textbackslash\{\}text\{HCl\} by \textbackslash\{\}pink\{6\}.
                    
                    \textbackslash\{\}qquad
                        \textbackslash\{\}text\{Al\} + \textbackslash\{\}pink\{6\}\textbackslash\{\}text\{HCl\} \textbackslash\{\}rightarrow 2\textbackslash\{\}text\{AlCl\}\_3 + 3\textbackslash\{\}text\{H\}\_2
  \item That gives us 2 \textbackslash\{\}text\{ Al\} on the right and
                        only 1 on the left, so multiply
                        \textbackslash\{\}text\{Al\} by \textbackslash\{\}green\{2\}.
                    
                    \textbackslash\{\}qquad
                        \textbackslash\{\}green\{2\}\textbackslash\{\}text\{Al\} + 6\textbackslash\{\}text\{HCl\} \textbackslash\{\}rightarrow 2\textbackslash\{\}text\{AlCl\}\_3 + 3\textbackslash\{\}text\{H\}\_2
  \item The balanced equation is:
                    \textbackslash\{\}qquad
                        2\textbackslash\{\}text\{Al\} + 6\textbackslash\{\}text\{HCl\} \textbackslash\{\}rightarrow 2\textbackslash\{\}text\{AlCl\}\_3 + 3\textbackslash\{\}text\{H\}\_2
  \item We can treat the phosphate polyatomic ion \textbackslash\{\}text\{(PO\}\_4\textbackslash\{\}text\{)\} as an atom, symbolized by \textbackslash\{\}green\{X\}:
                    
                    \textbackslash\{\}qquad
                        \textbackslash\{\}text\{CaCl\}\_2 + \textbackslash\{\}text\{Na\}\_3\textbackslash\{\}green\{\textbackslash\{\}text\{PO\}\_4\} \textbackslash\{\}rightarrow \textbackslash\{\}text\{Ca\}\_3(\textbackslash\{\}green\{\textbackslash\{\}text\{PO\}\_4\textbackslash\{\}text\{\}\})\_2 + \textbackslash\{\}text\{NaCl\}
                    
                    \textbackslash\{\}qquad
                        \textbackslash\{\}text\{CaCl\}\_2 + \textbackslash\{\}text\{Na\}\_3\textbackslash\{\}green\{X\} \textbackslash\{\}rightarrow \textbackslash\{\}text\{Ca\}\_3\textbackslash\{\}green\{X\}\_2 + \textbackslash\{\}text\{NaCl\}
  \item There is 1 \textbackslash\{\}space X on the left and 2 \textbackslash\{\}space X
                        on the right, so multiply
                        \textbackslash\{\}text\{Na\}\_3X by \textbackslash\{\}blue\{2\}.
                    
                    \textbackslash\{\}qquad
                        \textbackslash\{\}text\{CaCl\}\_2 + \textbackslash\{\}blue\{2\}\textbackslash\{\}text\{Na\}\_3X \textbackslash\{\}rightarrow \textbackslash\{\}text\{Ca\}\_3X\_2 + \textbackslash\{\}text\{NaCl\}
  \item There are 6 \textbackslash\{\}text\{ Na\} on the left and only
                        1 on the right, so multiply
                        \textbackslash\{\}text\{NaCl\} by \textbackslash\{\}red\{6\}.
                    
                    \textbackslash\{\}qquad
                        \textbackslash\{\}text\{CaCl\}\_2 + 2\textbackslash\{\}text\{Na\}\_3X \textbackslash\{\}rightarrow \textbackslash\{\}text\{Ca\}\_3X\_2 + \textbackslash\{\}red\{6\}\textbackslash\{\}text\{NaCl\}
  \item That gives us 6 \textbackslash\{\}text\{ Cl\} on the right and
                        only 2 on the left, so multiply
                        \textbackslash\{\}text\{CaCl\}\_2 by \textbackslash\{\}pink\{3\}.
                    
                    \textbackslash\{\}qquad
                        \textbackslash\{\}pink\{3\}\textbackslash\{\}text\{CaCl\}\_2 + 2\textbackslash\{\}text\{Na\}\_3X \textbackslash\{\}rightarrow \textbackslash\{\}text\{Ca\}\_3X\_2 + 6\textbackslash\{\}text\{NaCl\}
  \item Now \textbackslash\{\}text\{Ca\} is balanced too.
  \item Replacing \textbackslash\{\}text\{PO\}\_4 for X, the balanced equation is:
                    \textbackslash\{\}qquad
                        3\textbackslash\{\}text\{CaCl\}\_2 + 2\textbackslash\{\}text\{Na\}\_3\textbackslash\{\}text\{PO\}\_4 \textbackslash\{\}rightarrow \textbackslash\{\}text\{Ca\}\_3\textbackslash\{\}text\{(PO\}\_4\textbackslash\{\}text\{)\}\_2 + 6\textbackslash\{\}text\{NaCl\}
  \item There are 2 \textbackslash\{\}text\{ N\} on the left and
                        only 1 on the right, so multiply
                        \textbackslash\{\}text\{NO\}\_2 by \textbackslash\{\}blue\{2\}.
                    
                    \textbackslash\{\}qquad
                        \textbackslash\{\}text\{N\}\_2\textbackslash\{\}text\{H\}\_4 + \textbackslash\{\}text\{O\}\_2 \textbackslash\{\}rightarrow \textbackslash\{\}blue\{2\}\textbackslash\{\}text\{NO\}\_2 + \textbackslash\{\}text\{H\}\_2\textbackslash\{\}text\{O\}
  \item There are 4 \textbackslash\{\}text\{ H\} on the left and
                        only 2 on the right, so multiply
                        \textbackslash\{\}text\{H\}\_2\textbackslash\{\}text\{O\} by \textbackslash\{\}red\{2\}.
                    
                    \textbackslash\{\}qquad
                        \textbackslash\{\}text\{N\}\_2\textbackslash\{\}text\{H\}\_4 + \textbackslash\{\}text\{O\}\_2 \textbackslash\{\}rightarrow 2\textbackslash\{\}text\{NO\}\_2 + \textbackslash\{\}red\{2\}\textbackslash\{\}text\{H\}\_2\textbackslash\{\}text\{O\}
  \item That gives us 6 \textbackslash\{\}text\{ O\} on the right and
                        only 2 on the left, so multiply
                        \textbackslash\{\}text\{O\}\_2 by \textbackslash\{\}pink\{3\}.
                        (Since oxygen is by itself on the left, it should be done
                        at the end because you can give it a coefficient without
                        affecting another element.)
                    
                    \textbackslash\{\}qquad
                        \textbackslash\{\}text\{N\}\_2\textbackslash\{\}text\{H\}\_4 + \textbackslash\{\}pink\{3\}\textbackslash\{\}text\{O\}\_2 \textbackslash\{\}rightarrow 2\textbackslash\{\}text\{NO\}\_2 + 2\textbackslash\{\}text\{H\}\_2\textbackslash\{\}text\{O\}
  \item The balanced equation is:
                    \textbackslash\{\}qquad
                        \textbackslash\{\}text\{N\}\_2\textbackslash\{\}text\{H\}\_4 + 3\textbackslash\{\}text\{O\}\_2 \textbackslash\{\}rightarrow 2\textbackslash\{\}text\{NO\}\_2 + 2\textbackslash\{\}text\{H\}\_2\textbackslash\{\}text\{O\}
  \item There are 3 \textbackslash\{\}text\{ Fe\} on the right and
                        only 1 on the left, so multiply
                        \textbackslash\{\}text\{Fe\} by \textbackslash\{\}blue\{3\}.
                    
                    \textbackslash\{\}qquad
                        \textbackslash\{\}blue\{3\}\textbackslash\{\}text\{Fe\} + \textbackslash\{\}text\{H\}\_2\textbackslash\{\}text\{O\} \textbackslash\{\}rightarrow \textbackslash\{\}text\{Fe\}\_3\textbackslash\{\}text\{O\}\_4 + \textbackslash\{\}text\{H\}\_2
  \item There are 4 \textbackslash\{\}text\{ O\} on the right and
                        only 1 on the left, so multiply
                        \textbackslash\{\}text\{H\}\_2\textbackslash\{\}text\{O\} by \textbackslash\{\}red\{4\}.
                    
                    \textbackslash\{\}qquad
                        3\textbackslash\{\}text\{Fe\} + \textbackslash\{\}red\{4\}\textbackslash\{\}text\{H\}\_2\textbackslash\{\}text\{O\} \textbackslash\{\}rightarrow \textbackslash\{\}text\{Fe\}\_3\textbackslash\{\}text\{O\}\_4 + \textbackslash\{\}text\{H\}\_2
  \item That gives us 8 \textbackslash\{\}text\{ H\} on the left and
                        only 2 on the right, so multiply
                        \textbackslash\{\}text\{H\}\_2 by \textbackslash\{\}pink\{4\}.
                        (Since hydrogen is by itself on the right, it should be done
                        at the end because you can give it a coefficient without
                        affecting another element.)
                    
                    \textbackslash\{\}qquad
                        3\textbackslash\{\}text\{Fe\} + 4\textbackslash\{\}text\{H\}\_2\textbackslash\{\}text\{O\} \textbackslash\{\}rightarrow \textbackslash\{\}text\{Fe\}\_3\textbackslash\{\}text\{O\}\_4 + \textbackslash\{\}pink\{4\}\textbackslash\{\}text\{H\}\_2
  \item The balanced equation is:
                    \textbackslash\{\}qquad
                        3\textbackslash\{\}text\{Fe\} + 4\textbackslash\{\}text\{H\}\_2\textbackslash\{\}text\{O\} \textbackslash\{\}rightarrow \textbackslash\{\}text\{Fe\}\_3\textbackslash\{\}text\{O\}\_4 + 4\textbackslash\{\}text\{H\}\_2
  \item There are 2 \textbackslash\{\}text\{ N\} on the right and
                    2 on the left, so \textbackslash\{\}text\{N\}
                    is already balanced.
  \item There are 4 \textbackslash\{\}text\{ H\} on the left and
                        only 2 on the right, so multiply
                        \textbackslash\{\}text\{H\}\_2\textbackslash\{\}text\{O\} by \textbackslash\{\}blue\{2\}.
                    
                    \textbackslash\{\}qquad
                        \textbackslash\{\}text\{NH\}\_4\textbackslash\{\}text\{NO\}\_3 \textbackslash\{\}rightarrow \textbackslash\{\}text\{N\}\_2\textbackslash\{\}text\{O\} + \textbackslash\{\}blue\{2\}\textbackslash\{\}text\{H\}\_2\textbackslash\{\}text\{O\}
  \item There are 3 \textbackslash\{\}text\{ O\} on the right and
                    3 on the left, so \textbackslash\{\}text\{O\}
                    is already balanced.
  \item The balanced equation is:
                    \textbackslash\{\}qquad
                        \textbackslash\{\}text\{NH\}\_4\textbackslash\{\}text\{NO\}\_3 \textbackslash\{\}rightarrow \textbackslash\{\}text\{N\}\_2\textbackslash\{\}text\{O\} + 2\textbackslash\{\}text\{H\}\_2\textbackslash\{\}text\{O\}
  \item There are 2 \textbackslash\{\}text\{ O\} on the right and
                        only 1 on the left, so multiply
                        \textbackslash\{\}text\{HgO\} by \textbackslash\{\}blue\{2\}.
                    
                    \textbackslash\{\}qquad
                        \textbackslash\{\}blue\{2\}\textbackslash\{\}text\{HgO\} \textbackslash\{\}rightarrow \textbackslash\{\}text\{Hg\} + \textbackslash\{\}text\{O\}\_2
  \item Now there are 2 \textbackslash\{\}text\{ Hg\} on the left and
                        only 1 on the right, so multiply
                        \textbackslash\{\}text\{Hg\} by \textbackslash\{\}red\{2\}.
                    
                    \textbackslash\{\}qquad
                        2\textbackslash\{\}text\{HgO\} \textbackslash\{\}rightarrow \textbackslash\{\}red\{2\}\textbackslash\{\}text\{Hg\} + \textbackslash\{\}text\{O\}\_2
  \item The balanced equation is:
                    \textbackslash\{\}qquad
                        2\textbackslash\{\}text\{HgO\} \textbackslash\{\}rightarrow 2\textbackslash\{\}text\{Hg\} + \textbackslash\{\}text\{O\}\_2
  \item There is 1 \textbackslash\{\}text\{ Si\} on the right and
                    1 on the left, so \textbackslash\{\}text\{Si\}
                    is already balanced.
  \item There are 2 \textbackslash\{\}text\{ O\} on the left and
                        only 1 on the right, so multiply
                        \textbackslash\{\}text\{H\}\_2\textbackslash\{\}text\{O\} by \textbackslash\{\}blue\{2\}.
                    
                    \textbackslash\{\}qquad
                        \textbackslash\{\}text\{SiO\}\_2 + \textbackslash\{\}text\{HF\} \textbackslash\{\}rightarrow \textbackslash\{\}text\{SiF\}\_4 + \textbackslash\{\}blue\{2\}\textbackslash\{\}text\{H\}\_2\textbackslash\{\}text\{O\}
  \item Now there are 4 \textbackslash\{\}text\{ H\} on the right and
                        only 1 on the left, so multiply
                        \textbackslash\{\}text\{HF\} by \textbackslash\{\}red\{4\}.
                    
                    \textbackslash\{\}qquad
                        \textbackslash\{\}text\{SiO\}\_2 + \textbackslash\{\}red\{4\}\textbackslash\{\}text\{HF\} \textbackslash\{\}rightarrow \textbackslash\{\}text\{SiF\}\_4 + 2\textbackslash\{\}text\{H\}\_2\textbackslash\{\}text\{O\}
  \item Now \textbackslash\{\}text\{F\} is balanced too.
  \item The balanced equation is:
                    \textbackslash\{\}qquad
                        \textbackslash\{\}text\{SiO\}\_2 + 4\textbackslash\{\}text\{HF\} \textbackslash\{\}rightarrow \textbackslash\{\}text\{SiF\}\_4 + 2\textbackslash\{\}text\{H\}\_2\textbackslash\{\}text\{O\}
  \item There is 1 \textbackslash\{\}text\{ Mg\} on the right and
                    1 on the left, so \textbackslash\{\}text\{Mg\}
                    is already balanced.
  \item There are 2 \textbackslash\{\}text\{ Cl\} on the right and
                        only 1 on the left, so multiply
                        \textbackslash\{\}text\{HCl\} by \textbackslash\{\}blue\{2\}.
                    
                    \textbackslash\{\}qquad
                        \textbackslash\{\}text\{Mg(OH)\}\_2 + \textbackslash\{\}blue\{2\}\textbackslash\{\}text\{HCl\} \textbackslash\{\}rightarrow \textbackslash\{\}text\{MgCl\}\_2 + \textbackslash\{\}text\{H\}\_2\textbackslash\{\}text\{O\}
  \item Now there are 4 \textbackslash\{\}text\{ H\} on the left and
                        only 2 on the right, so multiply
                        \textbackslash\{\}text\{H\}\_2\textbackslash\{\}text\{O\} by \textbackslash\{\}red\{2\}.
                    
                    \textbackslash\{\}qquad
                        \textbackslash\{\}text\{Mg(OH)\}\_2 + 2\textbackslash\{\}text\{HCl\} \textbackslash\{\}rightarrow \textbackslash\{\}text\{MgCl\}\_2 + \textbackslash\{\}red\{2\}\textbackslash\{\}text\{H\}\_2\textbackslash\{\}text\{O\}
  \item Now \textbackslash\{\}text\{O\} is balanced too.
  \item The balanced equation is:
                    \textbackslash\{\}qquad
                        \textbackslash\{\}text\{Mg(OH)\}\_2 + 2\textbackslash\{\}text\{HCl\} \textbackslash\{\}rightarrow \textbackslash\{\}text\{MgCl\}\_2 + 2\textbackslash\{\}text\{H\}\_2\textbackslash\{\}text\{O\}
  \item We can treat the sulfate polyatomic ion \textbackslash\{\}text\{(SO\}\_4\textbackslash\{\}text\{)\} as an atom, symbolized by \textbackslash\{\}green\{X\}:
                    
                    \textbackslash\{\}qquad
                        \textbackslash\{\}text\{H\}\_2\textbackslash\{\}green\{\textbackslash\{\}text\{SO\}\_4\} + \textbackslash\{\}text\{Pb(OH)\}\_4 \textbackslash\{\}rightarrow \textbackslash\{\}text\{Pb(\}\textbackslash\{\}green\{\textbackslash\{\}text\{SO\}\_4\}\textbackslash\{\}text\{)\}\_2 + \textbackslash\{\}text\{H\}\_2\textbackslash\{\}text\{O\}
                    
                    \textbackslash\{\}qquad
                        \textbackslash\{\}text\{H\}\_2\textbackslash\{\}green\{X\} + \textbackslash\{\}text\{Pb(OH)\}\_4 \textbackslash\{\}rightarrow \textbackslash\{\}text\{Pb\}\textbackslash\{\}green\{X\}\_2 + \textbackslash\{\}text\{H\}\_2\textbackslash\{\}text\{O\}
  \item There is 1 \textbackslash\{\}space X on the left and 2 \textbackslash\{\}space X
                        on the right, so multiply
                        \textbackslash\{\}text\{H\}\_2X by \textbackslash\{\}blue\{2\}.
                    
                    \textbackslash\{\}qquad
                        \textbackslash\{\}blue\{2\}\textbackslash\{\}text\{H\}\_2X + \textbackslash\{\}text\{Pb(OH)\}\_4 \textbackslash\{\}rightarrow \textbackslash\{\}text\{Pb\}X\_2 + \textbackslash\{\}text\{H\}\_2\textbackslash\{\}text\{O\}
  \item That gives us 8 \textbackslash\{\}text\{ H\} on the left and
                        only 2 on the right, so multiply
                        \textbackslash\{\}text\{H\}\_2\textbackslash\{\}text\{O\} by \textbackslash\{\}red\{4\}.
                    
                    \textbackslash\{\}qquad
                        2\textbackslash\{\}text\{H\}\_2X + \textbackslash\{\}text\{Pb(OH)\}\_4 \textbackslash\{\}rightarrow \textbackslash\{\}text\{Pb\}X\_2 + \textbackslash\{\}red\{4\}\textbackslash\{\}text\{H\}\_2\textbackslash\{\}text\{O\}
  \item Everything is now balanced. Replacing \textbackslash\{\}text\{SO\}\_4 for X, the balanced equation is:
                    \textbackslash\{\}qquad
                        2\textbackslash\{\}text\{H\}\_2\textbackslash\{\}text\{SO\}\_4 + \textbackslash\{\}text\{Pb(OH)\}\_4 \textbackslash\{\}rightarrow \textbackslash\{\}text\{Pb(\}\textbackslash\{\}text\{SO\}\_4\textbackslash\{\}text\{)\}\_2 + 4\textbackslash\{\}text\{H\}\_2\textbackslash\{\}text\{O\}
  \item \textbackslash\{\}text\{As\} is already balanced.
  \item There are 6 \textbackslash\{\}text\{ S\} on the left and
                        only 1 on the right, so multiply
                        \textbackslash\{\}text\{SO\}\_2 by \textbackslash\{\}blue\{6\}.
                    
                    \textbackslash\{\}qquad
                        \textbackslash\{\}text\{As\}\_4\textbackslash\{\}text\{S\}\_6 + \textbackslash\{\}text\{O\}\_2 \textbackslash\{\}rightarrow \textbackslash\{\}text\{As\}\_4\textbackslash\{\}text\{O\}\_6 + \textbackslash\{\}blue\{6\}\textbackslash\{\}text\{SO\}\_2
  \item That gives us 18 \textbackslash\{\}text\{ O\} on the right and
                        only 2 on the left, so multiply
                        \textbackslash\{\}text\{O\}\_2 by \textbackslash\{\}red\{9\}.
                        (Since oxygen is by itself on the left, it should be done
                        at the end because you can give it a coefficient without
                        affecting another element.)
                    
                    \textbackslash\{\}qquad
                        \textbackslash\{\}text\{As\}\_4\textbackslash\{\}text\{S\}\_6 + \textbackslash\{\}red\{9\}\textbackslash\{\}text\{O\}\_2 \textbackslash\{\}rightarrow \textbackslash\{\}text\{As\}\_4\textbackslash\{\}text\{O\}\_6 + 6\textbackslash\{\}text\{SO\}\_2
  \item The balanced equation is:
                    \textbackslash\{\}qquad
                        \textbackslash\{\}text\{As\}\_4\textbackslash\{\}text\{S\}\_6 + 9\textbackslash\{\}text\{O\}\_2 \textbackslash\{\}rightarrow \textbackslash\{\}text\{As\}\_4\textbackslash\{\}text\{O\}\_6 + 6\textbackslash\{\}text\{SO\}\_2
  \item There are 2 \textbackslash\{\}text\{ Cr\} on the left and
                        only 1 on the right, so multiply
                        \textbackslash\{\}text\{Cr\} by \textbackslash\{\}blue\{2\}.
                    
                    \textbackslash\{\}qquad
                        \textbackslash\{\}text\{Cr\}\_2\textbackslash\{\}text\{O\}\_3 + \textbackslash\{\}text\{Mg\} \textbackslash\{\}rightarrow \textbackslash\{\}blue\{2\}\textbackslash\{\}text\{Cr\} + \textbackslash\{\}text\{MgO\}
  \item There are 3 \textbackslash\{\}text\{ O\} on the left and
                        only 1 on the right, so multiply
                        \textbackslash\{\}text\{MgO\} by \textbackslash\{\}red\{3\}.
                    
                    \textbackslash\{\}qquad
                        \textbackslash\{\}text\{Cr\}\_2\textbackslash\{\}text\{O\}\_3 + \textbackslash\{\}text\{Mg\} \textbackslash\{\}rightarrow 2\textbackslash\{\}text\{Cr\} + \textbackslash\{\}red\{3\}\textbackslash\{\}text\{MgO\}
  \item That gives us 3 \textbackslash\{\}text\{ Mg\} on the right and
                        only 1 on the left, so multiply
                        \textbackslash\{\}text\{Mg\} by \textbackslash\{\}pink\{3\}.
                    
                    \textbackslash\{\}qquad
                        \textbackslash\{\}text\{Cr\}\_2\textbackslash\{\}text\{O\}\_3 + \textbackslash\{\}pink\{3\}\textbackslash\{\}text\{Mg\} \textbackslash\{\}rightarrow 2\textbackslash\{\}text\{Cr\} + 3\textbackslash\{\}text\{MgO\}
  \item The balanced equation is:
                    \textbackslash\{\}qquad
                        \textbackslash\{\}text\{Cr\}\_2\textbackslash\{\}text\{O\}\_3 + 3\textbackslash\{\}text\{Mg\} \textbackslash\{\}rightarrow 2\textbackslash\{\}text\{Cr\} + 3\textbackslash\{\}text\{MgO\}
\end{itemize}
\end{document}
