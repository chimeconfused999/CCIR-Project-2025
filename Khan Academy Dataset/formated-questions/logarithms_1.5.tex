% Auto-converted from khan-exercises
\documentclass{article}
\usepackage{amsmath,amssymb}
\usepackage[T1]{fontenc}
\usepackage{textcomp}
\newcommand{\abs}[1]{\lvert #1\rvert}

\begin{document}
\section*{Evaluating logarithms 2}
\textbf{Question.} \textbackslash\{\}large\{\textbackslash\{\}log\_\{9\}\} 4 = \textbackslash\{\}text\{?\}

\textbf{Answer.} 3

\textbf{Hints.}
\begin{itemize}
  \item If b\textasciicircum{}y = x, then \textbackslash\{\}log\_\{b\} x = y.
  \item Therefore, we want to find the value y such that 9\textasciicircum{}\{y\} = 4.
  \item Any non-zero number raised to the power 0 is simply 1, so 9\textasciicircum{}0 = 1 and thus \textbackslash\{\}log\_\{9\} 1 = 0.
  \item Any number raised to the power -1 is its reciprocal, so 9\textasciicircum{}\{-1\} = $\frac{1}{9}$ and thus \textbackslash\{\}log\_\{9\} \textbackslash\{\}left($\frac{1}{9}$\textbackslash\{\}right) = -1.
  \item In this case, 9\textasciicircum{}\{3\} = 4,
                            so \textbackslash\{\}log\_\{9\} 4 = 3.
  \item If b\textasciicircum{}y = x, then \textbackslash\{\}log\_\{b\} x = y.
  \item Notice that 9 is the cube root of 759375.
  \item That is, \textbackslash\{\}sqrt[3]\{759375\} = 759375\textasciicircum{}\{1/3\} = 9.
  \item Thus, \textbackslash\{\}log\_\{759375\} 9 = $\frac{1}{3}$.
\end{itemize}
\end{document}
