% Auto-converted from khan-exercises
\documentclass{article}
\usepackage{amsmath,amssymb}
\usepackage[T1]{fontenc}
\usepackage{textcomp}
\newcommand{\abs}[1]{\lvert #1\rvert}

\begin{document}
\section*{Fractional exponents}
\textbf{Question.} \textbackslash\{\}Large\{[[fracParens( BASE\_N, BASE\_D )]]\textasciicircum{}\{\textasciicircum{}\{[[fracSmall( EXP\_NEG ? -1 : 1, EXP\_D )]]\}\} = \{?\}\}

\textbf{Answer.} [[SOL\_N / SOL\_D]]

\textbf{Hints.}
\begin{itemize}
  \item = [[fracParens( BASEF\_N, BASEF\_D )]]\textasciicircum{}\{[[fracSmall( 1, EXP\_D )]]\}
  \item Figure out what goes in the blank:\textbackslash\{\}Big(? \textbackslash\{\}Big)\textasciicircum{}\{[[VALS.root]]\}=[[frac( BASEF\_N, BASEF\_D )]]
  \item Figure out what goes in the blank:\textbackslash\{\}Big(\textbackslash\{\}color\{blue\}\{[[frac( SOL\_N, SOL\_D )]]\}\textbackslash\{\}Big)\textasciicircum{}\{[[VALS.root]]\}=[[frac( BASEF\_N, BASEF\_D )]]
  \item So [[fracParens( BASE\_N, BASE\_D )]]\textasciicircum{}\{[[fracSmall( EXP\_NEG ? -1 : 1, EXP\_D )]]\}=[[fracParens( BASEF\_N, BASEF\_D )]]\textasciicircum{}\{[[fracSmall( 1, EXP\_D )]]\}=$\frac{NaN}{NaN}$
  \item So [[fracParens( BASEF\_N, BASEF\_D )]]\textasciicircum{}\{[[fracSmall( 1, EXP\_D )]]\}=$\frac{NaN}{NaN}$
\end{itemize}
\end{document}
