% Auto-converted from khan-exercises
\documentclass{article}
\usepackage{amsmath,amssymb}
\usepackage[T1]{fontenc}
\usepackage{textcomp}
\newcommand{\abs}[1]{\lvert #1\rvert}

\begin{document}
\section*{Volume 2}
\textbf{Question.} What is the volume of this box? Drag on the box to rotate it.

\textbf{Answer.} [[NUMERPRODUCT / DENOMPRODUCT]] 2, 2, 4\textasciicircum{}3

\textbf{Hints.}
\begin{itemize}
  \item The volume of a shape is measured by the number of one \textbackslash\{\}text\{ [[randFromArray(metricUnits)]]\} cubes which make up the shape.
  \item The volume of a box equals width $\times$ height $\times$ length.
  \item Therefore, \textbackslash\{\}text\{NaN, NaN\} = [ $\cdot$ [ $\cdot$ L
  \item \textbackslash\{\}qquad = 
                    $\frac{[[NUMERPRODUCT / GCD]]}{[[DENOMPRODUCT / GCD]]}$ = 
                    NaN, NaN 2, 2, 4\textasciicircum{}3
  \item The volume of a box equals width $\times$ height $\times$ length.
  \item Therefore [[VOL]] = [ $\cdot$ L $\cdot$ x.
  \item \textbackslash\{\}qquadNaN, 1x = [[VOL]]
  \item \textbackslash\{\}qquad x = [[VOL]] \textbackslash\{\}div
                    NaN, 1
  \item \textbackslash\{\}qquad x = [[VOL]] $\cdot$
                    1, NaN
                    = [
  \item \textbackslash\{\}qquad x = [
\end{itemize}
\end{document}
