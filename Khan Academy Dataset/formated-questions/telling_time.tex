% Auto-converted from khan-exercises
\documentclass{article}
\usepackage{amsmath,amssymb}
\usepackage[T1]{fontenc}
\usepackage{textcomp}
\newcommand{\abs}[1]{\lvert #1\rvert}

\begin{document}
\section*{Telling time}
\textbf{Question.} What time is it?

\textbf{Answer.} The time is 5 : 45 [[icu.getDateFormatSymbols().am\_pm[HOUR >= 7 ? 0 : 1]]]

\textbf{Hints.}
\begin{itemize}
  \item The small hand is for the hour, and the big hand is for the minutes.
  \item The hour hand is between 5 and 6, so the hour is 5.
                    The hour hand is close to but hasn't passed 6, so the hour is still 5.
  \item The minute hand starts pointing straight up for 0 minutes, and it makes a complete circle in one hour (passing by all 12 numbers in 60 minutes).
                    For each number that the minute hand passes, add $\frac{60}{12}$ = 5 minutes.
  \item The minute hand is pointing at 9, which represents 5 $\times$ 9 = 45 minutes.
  \item The time is 5:45.
\end{itemize}
\end{document}
