% Auto-converted from khan-exercises
\documentclass{article}
\usepackage{amsmath,amssymb}
\usepackage[T1]{fontenc}
\usepackage{textcomp}
\newcommand{\abs}[1]{\lvert #1\rvert}

\begin{document}
\section*{Recognizing fractions 0.5}
\textbf{Question.} This circle represents one whole.
                            
                        
                        
                            What fraction is shaded blue below?

\textbf{Answer.} 2

\textbf{Hints.}
\begin{itemize}
  \item Each blue slice is \textbackslash\{\}blue\{$\frac{1}{2}$\} of the whole.
  \item \textbackslash\{\}pink\{4\} slices are shaded blue, so we add
                            \textbackslash\{\}blue\{$\frac{1}{2}$\} a total of \textbackslash\{\}pink\{4\} times.
                        
                        [[\_.times(NUM\_1, function() \{ return FRACTION; \}).join(" + ")]] = ?
  \item The fraction shaded blue is $\frac{4}{2}$.
  \item Each blue piece is \textbackslash\{\}blue\{$\frac{1}{2}$\} of the whole.
  \item \textbackslash\{\}pink\{4\} piece are shaded blue, so we add
                            \textbackslash\{\}blue\{$\frac{1}{2}$\} a total of \textbackslash\{\}pink\{4\} times.
                        
                        [[\_.times(NUM\_1, function() \{ return FRACTION; \}).join(" + ")]] = ?
  \item The fraction shaded blue is $\frac{4}{2}$.
\end{itemize}
\end{document}
