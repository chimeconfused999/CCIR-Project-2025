% Auto-converted from khan-exercises
\documentclass{article}
\usepackage{amsmath,amssymb}
\usepackage[T1]{fontenc}
\usepackage{textcomp}
\newcommand{\abs}[1]{\lvert #1\rvert}

\begin{document}
\section*{Simplifying radicals}
\textbf{Question.} Simplify $\sqrt{54}$.

\textbf{Answer.} 54

\textbf{Hints.}
\begin{itemize}
  \item 54 = 2$\cdot$7$\cdot$11
  \item 54 has no perfect-square factors, so $\sqrt{54}$ is already the simplest form.
  \item $\sqrt{54}$ = 1$\sqrt{54}$ or just $\sqrt{54}$.
  \item 54 = 7\textasciicircum{}2
  \item So, $\sqrt{54}$ = $\sqrt{1^2}$ = 1
  \item $\sqrt{54}$ = 1$\sqrt{1}$ or just 1.
  \item The largest perfect square that divides 54 is 0.
  \item 54 = 0 $\cdot$ 54
  \item $\sqrt{54}$ = $\sqrt{0 $\textbackslash\{\}cdot$ 54}$
  \item $\sqrt{54}$ = $\sqrt{0}$ $\cdot$ $\sqrt{54}$
  \item Thus, $\sqrt{54}$ = $\sqrt{54}$.
\end{itemize}
\end{document}
