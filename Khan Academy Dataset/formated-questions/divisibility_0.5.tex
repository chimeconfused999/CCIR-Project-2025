% Auto-converted from khan-exercises
\documentclass{article}
\usepackage{amsmath,amssymb}
\usepackage[T1]{fontenc}
\usepackage{textcomp}
\newcommand{\abs}[1]{\lvert #1\rvert}

\begin{document}
\section*{Divisibility 0.5}
\textbf{Question.} Which of the following numbers is a factor of [[A]]?

\textbf{Answer.} [[B]]

\textbf{Hints.}
\begin{itemize}
  \item By definition, a factor of a number will divide evenly into that number. We can start by dividing [[A]] by each of our answer choices.
  \item [[A]] \textbackslash\{\}div [[WRONG]] = [[floor( A / WRONG )]]\textbackslash\{\}text\{ R \}[[( A \% WRONG )]]
  \item The only answer choice that divides into \textbackslash\{\}blue\{[[A]]\} with no remainder is \textbackslash\{\}pink\{[[B]]\}.
                        \textbackslash\{\}quad[[FACTOR]] $\times$ \textbackslash\{\}pink\{[[B]]\} = \textbackslash\{\}blue\{[[A]]\}.
  \item We can check our answer by looking at the prime factorization of both numbers. Notice that the prime factors of [[B]] are contained within the prime factors of [[A]].
                        [[A]] = [[FACTORIZATION\_A.join( "\textbackslash\{\}$\times$" )]]\textbackslash\{\}qquad\textbackslash\{\}qquad[[B]] = [[FACTORIZATION\_B.join( "\textbackslash\{\}$\times$" )]]
  \item Therefore, \textbackslash\{\}pink\{[[B]]\} is a factor of \textbackslash\{\}blue\{[[A]]\}.
  \item The multiples of [[B]] are [[B]], [[B*2]], [[B*3]], [[B*4]]...
                        In general, any number that leaves no remainder when divided by [[B]] is considered a multiple of [[B]].
  \item We can start by dividing each of our answer choices by [[B]].
                        [[WRONG]] \textbackslash\{\}div [[B]] = [[floor( WRONG / B )]]\textbackslash\{\}text\{ R \}[[( WRONG \% B )]]
  \item The only answer choice that leaves no remainder after the division is \textbackslash\{\}blue\{[[A]]\}.
                        \textbackslash\{\}quad[[FACTOR]] $\times$ \textbackslash\{\}pink\{[[B]]\} = \textbackslash\{\}blue\{[[A]]\}.
  \item We can check our answer by looking at the prime factorization of both numbers. Notice that the prime factors of [[B]] are contained within the prime factors of [[A]].
                        [[A]] = [[FACTORIZATION\_A.join( "\textbackslash\{\}$\times$" )]]\textbackslash\{\}qquad\textbackslash\{\}qquad[[B]] = [[FACTORIZATION\_B.join( "\textbackslash\{\}$\times$" )]]
  \item Therefore, \textbackslash\{\}blue\{[[A]]\} is a multiple of \textbackslash\{\}pink\{[[B]]\}.
\end{itemize}
\end{document}
